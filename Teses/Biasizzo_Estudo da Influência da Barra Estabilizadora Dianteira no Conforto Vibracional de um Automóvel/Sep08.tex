
\documentclass{sebase}
\usepackage{amssymb}

%%%%%%%%%%%%%%%%%%%%%%%%%%%%%%%%%%%%%%%%%%%%%%%%%%%%%%%%%%%%%%%%%%%%%%%%%%%%%%%%%%%%%%%%%%%%%%%%%%%%
\usepackage{tesepuc}

%TCIDATA{OutputFilter=LATEX.DLL}
%TCIDATA{Created=Thu Jun 07 21:50:19 2001}
%TCIDATA{LastRevised=Tue Sep 11 16:34:41 2001}
%TCIDATA{<META NAME="GraphicsSave" CONTENT="32">}
%TCIDATA{<META NAME="DocumentShell" CONTENT="Style Editor\tesepuc #1">}
%TCIDATA{CSTFile=tesepuc.cst}

\input{tcilatex}

\begin{document}


\begin{NFThesistitle}
An\'{a}\c{c}ise de Sensibilidade da Barra Estabilizadora sobre Estudo do
Conforto de um Autom\'{o}vel Utilizando a paranormalidade comparada
\end{NFThesistitle}

\begin{NFThesisauthor}
Mauro Bruno Biasizzo
\end{NFThesisauthor}

\begin{NFDessertationreq}
Disserta\c{c}\~{a}o de Mestrado
\end{NFDessertationreq}

\begin{NFDegreename}
Departamento de Engenharia Mec\^{a}nica
\end{NFDegreename}

\begin{NFMajorsubject}
Belo Horizonte, Minas Gerais
\end{NFMajorsubject}

\begin{NFMinorsubject}
LaTeX
\end{NFMinorsubject}

\begin{NFUniversity}
Pontif\'{i}cia Universidade Cat\'{o}lica de Minas Gerais
\end{NFUniversity}

\begin{NFThesisdate}
Julho de 2001
\end{NFThesisdate}

\begin{NFTcopyright}
Copyright 1994 by C.W. Casey
\end{NFTcopyright}

%TCIMACRO{
%\TeXButton{Dissertation With Major Minor}{\TitleDisMajorMinor{}%
%}}%
%BeginExpansion
\TitleDisMajorMinor{}%
%
%EndExpansion

\begin{NFDissertation}
Mestre em Engenharia Mec\^{a}nica
\end{NFDissertation}

\begin{NFCommitteesigna}
Clovis Sperb de Barcellos

Dean of Scientific Word
\end{NFCommitteesigna}

\begin{NFCommitteesignb}
Juan Manuel Fangio

Chair of the Examining Committee
\end{NFCommitteesignb}

\begin{NFCommitteesignc}
Enzo\ Ferrari

Co-Chair of the Examining Committee
\end{NFCommitteesignc}

\begin{NFDate}
Julho de 2001
\end{NFDate}

\begin{NFCommittee}
Committee in charge:

Dr. Luca Cordero di Montezemolo

Dr. Jean Tod
\end{NFCommittee}

%TCIMACRO{
%\TeXButton{Sign Page}{\SignPage{}%
%}}%
%BeginExpansion
\SignPage{}%
%
%EndExpansion

\begin{NFChairperson}
Dr. San Martin, Chair
\end{NFChairperson}

%TCIMACRO{
%\TeXButton{Abstract for Dissertation}{\AbstractDis{}%
%}}%
%BeginExpansion
\AbstractDis{}%
%
%EndExpansion
\newpage

\QTP{contents}
Sum\'{a}rio

%TCIMACRO{
%\TeXButton{gerar sumario}{\TableOfContents%
%}}%
%BeginExpansion
\TableOfContents%
%
%EndExpansion
\newpage

\chapter{Introdu\c{c}\~{a}o}

%TCIMACRO{
%\TeXButton{pagina numera��o =1}{\setcounter{page}{1}%
%}}%
%BeginExpansion
\setcounter{page}{1}%
%
%EndExpansion

\bigskip

As suspens\~{o}es tradicionais de autom\'{o}veis, ou as chamadas
suspens\~{o}es passivas, s\~{a}o projetadas de maneira a obter um
compromisso no que diz respeito \`{a}s caracter\'{i}sticas de
estabilidade/manobrabilidade (handling) e conforto vibracional dos
autom\'{o}veis. As caracter\'{i}sticas de ride s\~{a}o associadas
primariamente com a capacidade do sistema de suspens\~{a}o em absorver esfor%
\c{c}os verticais, ou seja, minimizar a transmiss\~{a}o de vibra\c{c}\~{o}es
para os ocupantes do ve\'{i}culo. Handling representa a capacidade e o esfor%
\c{c}o do conjunto ve\'{i}culo/motorista para realizar manobras com sucesso,
o estudo de como isso ocorre e o estudo da percep\c{c}\~{a}o do motorista
quanto ao comportamento do ve\'{i}culo em manobras\cite{teste01}. Desta
maneira as caracter\'{i}sticas de Handling est\~{a}o ligadas \`{a}s for\c{c}%
as horizontais que atuam sobre o CG do ve\'{i}culo, \`{a} transfer\^{e}ncia
de peso devido \`{a}s acelera\c{c}\~{o}es e \`{a}s caracter\'{i}sticas dos
pneum\'{a}ticos.

As suspens\~{o}es ativas e semi-ativas permitem que se alcancem boas condi\c{%
c}\~{o}es de conforto sem o detrimento das caracter\'{i}sticas de handling,
e vice-versa. Para obter bons par\^{a}metros de conforto/handling, as
suspens\~{o}es ativas utilizam , geralmente, complexos algoritmos e sistemas
de controle associados a sensores e atuadores
hidr\'{a}ulicos/pneum\'{a}ticos de alta press\~{a}o. Alterando as
caracter\'{i}sticas de rigidez e amortecimento dos elementos das
suspens\~{a}o de acordo com as condi\c{c}\~{o}es de tr\'{a}fego do
ve\'{i}culo ou atuando diretamente na resposta din\^{a}mica do ve\'{i}culo,
os par\^{a}metros propostos s\~{a}o alcan\c{c}ados. Apesar de realmente
eficientes, o alto custo dos sensores e atuadores ainda limita a utiliza\c{c}%
\~{a}o deste tipo de suspens\~{a}o aos ve\'{i}culos topo de linha.

Existem, ainda, sistemas que tem como principal finalidade diminuir o
rolamento da carroceria em curvas atrav\'{e}s se um sistema de controle da
rigidez da barra estabilizadora. Em compara\c{c}\~{a}o com os sistemas
passivos, este sistema oferece melhoras substanciais em termos de conforto e
handling. Comparando-o com os sistemas ativos, possui uma excelente rela\c{c}%
\~{a}o custo/beneficio, j\'{a} que o n\'{u}mero de atuadores e a press\~{a}o
de opera\c{c}\~{a}o destes \'{e} consideravelmente menor\cite{teste03}.
Embora n\~{a}o exercendo um controle total sobre o sistema de suspens\~{a}o
do ve\'{i}culo (atua somente minimizando o rolamento da carroceria) este
tipo de sistema \'{e} capaz de melhorar substancialmente algumas
caracter\'{i}sticas de handling e conforto.

Alguns estudos experimentais e num\'{e}ricos sobre este tipo de sistema
foram realizados concentrando-se exclusivamente nos benef\'{i}cios alcan\c{c}%
ados para as caracter\'{i}sticas de handling\cite{teste03}. Mesmo tendo-se
percebido uma consider\'{a}vel melhora no isolamento de vibra\c{c}\~{o}es,
ou seja , nos par\^{a}metros de conforto, todas as considera\c{c}\~{o}es a
esse respeito foram puramente subjetivas.

\section{Justificativa}

Nos \'{u}ltimos anos, com a libera\c{c}\~{a}o das importa\c{c}\~{o}es de
autom\'{o}veis, temos acompanhado uma substancial melhora tecnol\'{o}gica e,
consequentemente, da qualidade dos autom\'{o}veis fabricados no Brasil.
Frente a um mercado cada vez mais competitivo, as montadoras instaladas no
pa\'{i}s procuram fazer com que os autom\'{o}veis nacionais atinjam
n\'{i}veis de qualidade equiparados aos n\'{i}veis dos produtos
comercializados na Europa e EUA.

Os autom\'{o}veis nacionais, que t\^{e}m seus projetos desenvolvidos nas
matrizes das montadoras na Europa ou EUA, s\~{a}o adaptados \`{a}s condi\c{c}%
\~{o}es clim\'{a}ticas e \`{a}s caracter\'{i}sticas das estradas e relevos
do pa\'{i}s. Um dos principais sistemas adaptados para os mercados
sul-americanos, juntamente com sistemas de arrefecimento, \'{e} o de
suspens\~{a}o que deve proporcionar aos ocupantes do ve\'{i}culo bons
n\'{i}veis de conforto e handling.

Embora os autom\'{o}veis nacionais tenham alcan\c{c}ado n\'{i}veis de
sofistica\c{c}\~{a}o t\~{a}o altos, a determina\c{c}\~{a}o dos
par\^{a}metros que devem ser modificados para a adapta\c{c}\~{a}o destes
\`{a}s condi\c{c}\~{o}es brasileiras \'{e}, na maioria dos casos, puramente
emp\'{i}rica, ou seja, pilotos experientes relatam as suas impress\~{o}es ao
dirigirem o autom\'{o}vel aos t\'{e}cnicos e engenheiros que atuam adaptando
os projetos. Esse tipo de procedimento, de fundamental import\^{a}ncia e que
produz bons resultados, fica condicionado \`{a}s habilidades/sensibilidade
deste ou outro piloto, habilidades estas que podem ser alteradas devido
\`{a}s condi\c{c}\~{o}es em que o piloto se encontra no momento da realiza\c{%
c}\~{a}o dos experimentos, al\'{e}m de n\~{a}o fornecer nenhum par\^{a}metro
mensur\'{a}vel a n\~{a}o ser a sensa\c{c}\~{o}es que o piloto \'{e} capaz de
transmitir ao pessoal t\'{e}cnico.

Dentre os v\'{a}rios componentes de um sistema de suspens\~{a}o de um
autom\'{o}vel, a barra estabilizadora cumpre um importante papel no
''afinamento'' das suspens\~{o}es. O estudo da influ\^{e}ncia da barra
estabilizadora em par\^{a}metros de conforto, ou seja, um estudo das
v\'{a}rias freq\"{u}\^{e}ncias de vibra\c{c}\~{a}o, provenientes da pista,
que atingem os ocupantes do ve\'{i}culo, possibilitar\'{a} identificar qual
a influ\^{e}ncia da rigidez e da forma da barra estabilizadora no conforto
de um autom\'{o}vel.

Utilizando a t\'{e}cnica de prototipagem virtual, que ''permite o
desenvolvimento de um produto sem a constru\c{c}\~{a}o de um prot\'{o}tipo
f\'{i}sico''\cite{teste05}, as altera\c{c}\~{o}es dos par\^{a}metros de
projetos podem ser efetuadas computacionalmente, verificando-se sua
efici\^{e}ncia atrav\'{e}s de uma simula\c{c}\~{a}o num\'{e}rica dos
experimentos realizados nas estradas e pistas. Portanto, a cria\c{c}\~{a}o
de um modelo num\'{e}rico de um autom\'{o}vel e o dom\'{i}nio da t\'{e}cnica
de prototipagem virtual resultar\~{a}o em uma economia de tempo, j\'{a} que
n\~{a}o ser\'{a} necess\'{a}ria a constru\c{c}\~{a}o de tantos
prot\'{o}tipos f\'{i}sicos, e proporcionar\~{a}o \`{a}s empresas uma maior
agilidade para a adequa\c{c}\~{a}o dos novos modelos \`{a}s condi\c{c}%
\~{o}es brasileiras o que pode-se traduzir em redu\c{c}\~{a}o de custos.

O desenvolvimento de um sistema de controle para variar a rigidez da barra
estabilizadora \'{e} uma alternativa, tratando-se de suspens\~{o}es ativas,
de baixo custo e com possibilidade de excelentes resultados pr\'{a}ticos nas
caracter\'{i}sticas de Handling\cite{teste03}. Atualmente pesquisadores
ingleses e americanos est\~{a}o desenvolvendo, para a DELPHI, sistemas de
controle similares para a utiliza\c{c}\~{a}o em utilit\'{a}rios
(ve\'{i}culos com C.G. alto que necessitam minimizar os efeitos do rolamento
da carro\c{c}eria e que ao mesmo tempo sofrem com a alta rigidez de suas
suspens\~{o}es).Esse estudo tem sido impulsionado pela grande procura por
este tipo de ve\'{i}culo na Europa e EUA. O n\'{u}mero de artigos que tratam
exclusivamente da influ\^{e}ncia deste tipo de suspens\~{a}o nos
par\^{a}metros de conforto \'{e} pequeno, o que d\'{a} a este trabalho
relevante import\^{a}ncia.

\bigskip \newpage

\chapter{O autom\'{o}vel}

\section{Caracter\'{i}sticas construtivas do autom\'{o}vel.}

O autom\'{o}vel utilizado para o estudo dessa disserta\c{c}\~{a}o \'{e} um
ve\'{i}culo de tr\^{e}s volumes, de porte m\'{e}dio, com tra\c{c}\~{a}o no
eixo dianteiro e motor e c\^{a}mbio montados transversalmente.

A suspens\~{a}o dianteira utiliza o sistema McPherson com molas helicoidais,
amortecedores hidr\'{a}ulicos de dupla a\c{c}\~{a}o e barra estabilizadora e
a suspens\~{a}o traseira o sistema ''semi-trailing arm'' , com molas
helicoidais, amortecedores hidr\'{a}ulicos de dupla a\c{c}\~{a}o e barra
enrijecedora .

\section{Suspens\~{a}o Dianteira - MacPherson}

Atualmente a utiliza\c{c}\~{a}o das suspens\~{o}es McPherson est\'{a}
largamente difundida nos ve\'{i}culos de passeio, tanto para o eixo
dianteiro quanto para o traseiro. A sua grande difus\~{a}o entre os
autom\'{o}veis se d\'{a} principalmente devido as suas boas
caracter\'{i}sticas de montagem, ou seja, quase todas as partes componentes
de uma suspens\~{a}o McPherson podem ser agrupadas em um \'{u}nico conjunto.

A rigor o sistema de suspens\~{a}o McPherson, apresentado na figura \ref
{McPh_show}, \'{e} uma evolu\c{c}\~{a}o, ou melhor, uma continua\c{c}\~{a}o
do desenvolvimento da ''Double wishbone''(quadril\'{a}teros deform\'{a}veis)
figura \ref{dwb_forces}. O bra\c{c}o transversal superior da ''Double
wishbone'' foi substituido por um ponto de giro na torre do amortecedor. A
chamada torre do amortecedor recebe esse nome porque a extremidade da haste
do amortecedor e a mola helicoidal s\~{a}o alojadas no seu interior, figura 
\ref{McPh_show}.

\FRAME{fhFU}{2.7129in}{2.7795in}{0pt}{\Qcb{Sistema de suspens\~{a}o tipo
McPherson}}{\Qlb{McPh_show}}{mcpherson.jpg}{\special{language "Scientific
Word";type "GRAPHIC";maintain-aspect-ratio TRUE;display "USEDEF";valid_file
"F";width 2.7129in;height 2.7795in;depth 0pt;original-width
6.7291in;original-height 6.896in;cropleft "0";croptop "1";cropright
"1";cropbottom "0";filename 'Mcpherson.jpg';file-properties "XNPEU";}}

As vantagens do sistema de suspens\~{a}o Mcpherson s\~{a}o as seguintes:

1. For\c{c}as laterais aplicadas \`{a} carroceria reduzidas em rela\c{c}%
\~{a}o \`{a} suspens\~{a}o do tipo DWB, devido \`{a} grande dimens\~{a}o de
c, (ver fig.\ref{dwb_forces} e eq.\ref{eq.3.1}),

2. Pequena dist\^{a}ncia b, ( ver fig.\ref{McPh_Forces_b} e eq.\ref{eq.3.1}),

3. Possibilita utilizar mola que trabalha com longo curso,

4. Melhor op\c{c}\~{a}o de projeto para a zona de deforma\c{c}\~{a}o
programada,\cite{Reimpell}

5. Conjunto compacto permitindo um maior compartimento para o motor.

6. Devido ao maior compartimento, possibilita a montagem de motores
transversais.

Assim como qualquer outro sistema de suspens\~{a}o, o sistema McPherson
possui algumas desvantagens, s\~{a}o elas:

1. Caracter\'{i}sticas cinem\'{a}ticas n\~{a}o muito favor\'{a}veis,*

2. A introdu\c{c}\~{a}o de for\c{c}as e vibra\c{c}\~{o}es mec\^{a}nicas
diretamente na carroceria do ve\'{i}culo atrav\'{e}s da torre do amortecedor,

3. A maior dificuldade em isolar as vibra\c{c}\~{o}es provenientes das
irregularidades do piso, requer que seja utilizado na fixa\c{c}\~{a}o do
amortecedor e mola \`{a} carroceira um batente de borracha e uma estrutura
que seja ''desacoplada'' \footnote{%
O termo ''desacoplado'' \'{e} a tradu\c{c}\~{a}o de decoupled, e significa
que as for\c{c}as do amortecedor e das molas s\~{a}o absorvidas
separadamente pela estrutura de fixa\c{c}\~{a}o da suspens\~{a}o \`{a}
carroceria.}. (figura xx)

4. O atrito causado entre a haste amortecedor e a guia selo prejudica o
efeito de amortecimento e ocasiona um desgaste prematuro do
amortecedor.\bigskip \FRAME{fhFU}{2.3791in}{1.8377in}{0pt}{\Qcb{For\c{c}as
atuantes em uma suspens\~{a}o Double Wish Bone.}}{\Qlb{dwb_forces}}{%
dwb_forces.jpg}{\special{language "Scientific Word";type
"GRAPHIC";maintain-aspect-ratio TRUE;display "USEDEF";valid_file "F";width
2.3791in;height 1.8377in;depth 0pt;original-width 3.7706in;original-height
2.9066in;cropleft "0";croptop "1";cropright "1";cropbottom "0";filename
'DWB_forces.jpg';file-properties "XNPEU";}}\bigskip \FRAME{fhFU}{6.3439cm}{%
7.2994cm}{0pt}{\Qcb{For\c{c}as atuantes em um sistema de suspens\~{a}o do
tipo McPherson devido \`{a} carga vertical.}}{\Qlb{McPh_Forces_b}}{%
mcph_forces.jpg}{\special{language "Scientific Word";type
"GRAPHIC";maintain-aspect-ratio TRUE;display "USEDEF";valid_file "F";width
6.3439cm;height 7.2994cm;depth 0pt;original-width 3.6564in;original-height
4.2082in;cropleft "0";croptop "1.0017";cropright "1.0012";cropbottom
"0";filename 'Mcph_Forces.jpg';file-properties "XNPEU";}}

\subsection{As for\c{c}as atuantes:}

Considerando que a for\c{c}a $F_{z}$ \'{e} a \'{u}nica for\c{c}a externa que
atua no sistema, teremos, de acordo com a figura \ref{McPh_Forces_b} , que
as demais for\c{c}as necess\'{a}rias para equilibrar o sistema ser\~{a}o, $F$
e $F_{G}$.

Sendo $F$ a for\c{c}a de rea\c{c}\~{a}o do conjunto mola/amortecedor e $%
F_{G} $ a for\c{c}a de rea\c{c}\~{a}o do bra\c{c}o oscilante.

Quando a mola helicoidal \'{e} montada concentricamente com o amortecedor,
como na figura, ela s\'{o} \'{e} capaz de absorver esfor\c{c}os nessa dire\c{%
c}\~{a}o. Dessa forma teremos que as duas componentes da for\c{c}a $F$
ser\~{a}o:

- $F_{S}$ a for\c{c}a de rea\c{c}\~{a}o da mola helicoidal;

- $F_{E}$ \'{e} uma componente de $F$ que atua na haste do amortecedor.

A componente $F_{E}$ , como foi comentado anteriormente, faz com que a haste
do amortecedor trabalhe tamb\'{e}m a flex\~{a}o, fazendo-se necess\'{a}ria a
utiliza\c{c}\~{a}o de hastes de maior di\^{a}metro.

No diagrama de for\c{c}as para a haste do amortecedor, figura \ref
{McPh_Forces_b}, as for\c{c}as de rea\c{c}\~{a}o $F_{C,Y}$ e $F_{K,Y}$
s\~{a}o respectivamente as rea\c{c}\~{o}es na guia da haste e no \^{e}mbolo.
Quanto maiores forem essas for\c{c}as, maiores ser\~{a}o as for\c{c}as de
atrito entre a haste e a guia e entre o \^{e}mbolo e o cilindro do
amortecedor, fazendo com que sejam necess\'{a}rias maiores for\c{c}as
verticais no pneu para deslocar o amortecedor. Isso se traduz em uma maior
transmissibilidade de vibra\c{c}\~{o}es do sistema de suspens\~{a}o para a
carroceria do autom\'{o}vel, j\'{a} que para pequenas irregularidades do
piso que n\~{a}o s\~{a}o capazes de produzir uma rea\c{c}\~{a}o completa do
amortecedor, o amortecedor funciona como um elemento r\'{i}gido transmitindo
quase que completamente a for\c{c}a para a carroceria, reduzindo o conforto
do autom\'{o}vel.

Al\'{e}m do decr\'{e}scimo do conforto, o aumento da for\c{c}a de atrito
entre os elementos do amortecedor acelera o desgaste da guia da haste
provocando vazamentos prematuros e perda de efici\^{e}ncia do amortecedor
devido ao desgaste do embolo do amortecedor.

Para minimizar a componente $F_{E}$, atualmente as molas helicoidais s\~{a}o
montadas de forma alinhadas com a linha de a\c{c}\~{a}o da for\c{c}a $F$.
H\'{a} duas formas de se fazer com que a mola helicoidal seja coincidente
com eixo $EO$, s\~{a}o elas:

1 - Montar a mola de forma que a sua linha de a\c{c}\~{a}o n\~{a}o seja
coincidente com a linha de a\c{c}\~{a}o do amortecedor, figura xx. Na equa\c{%
c}\~{a}o \ref{eq.6.1}, a medida que $s$ aumenta $F_{E}$ diminui, ou na equa%
\c{c}\~{a}o \ref{eq.4.1} \`{a} medida que s aumenta, a diminui e em
consequ\^{e}ncia o seu seno, desta forma $F_{E}$ tamb\'{e}m diminui.

2 - A outra solu\c{c}\~{a}o \'{e} diminuir $a$ dist\^{a}ncia $b$,ver figura 
\ref{McPh_Forces_b}, que de acordo com a equa\c{c}\~{a}o \ref{eq.3.1} \'{e}
diretamente proporcional a $F_{E}$.

\bigskip

\[
\sum Fv=0 
\]

\begin{equation}
F_{Z}+F_{E}.\sin \sigma +F_{G}.\sin \beta =F_{S}\cos \sigma  \label{eq.1.1}
\end{equation}

\bigskip 
\[
\sum F_{H}=0 
\]
\begin{equation}
F_{G}.\cos \beta =F_{E}.\cos \sigma +F_{S}.\sin \sigma  \label{eq.2.1}
\end{equation}

\bigskip \bigskip 
\[
\sum M_{G}=0 
\]

\begin{equation}
F_{Z}.b=F_{E}.(c+o)\Longrightarrow F_{E}=\frac{F_{Z}.b}{(c+o)}
\label{eq.3.1}
\end{equation}

\bigskip

Da figura, temos que:

\bigskip 
\begin{equation}
F_{E}=F.\sin \alpha  \label{eq.4.1}
\end{equation}

\bigskip 
\begin{equation}
F_{S}=F.\cos \alpha  \label{eq.5.1}
\end{equation}

\bigskip

Para simplificar a an\'{a}lise, ser\'{a} considerado nulo o \^{a}ngulo entre
o bra\c{c}o oscilante e a horizontal, ou seja, b = 0, desta forma a equa\c{c}%
\~{a}o \ref{eq.1.1} pode ser escrita da forma:

\begin{equation}
F_{Z}+F_{E}.\sin \sigma =F_{S}.\cos \sigma \Longrightarrow F_{E}=\frac{%
F_{S}.\cos \sigma -F_{Z}}{\sin \sigma }  \label{eq.6.1}
\end{equation}

Entretanto, nessas duas solu\c{c}\~{o}es o espa\c{c}o dispon\'{i}vel na
caixa de roda n\~{a}o permite a montagem adequada do conjunto
mola/amortecedor, principalmente quando s\~{a}o utilizados pneum\'{a}ticos
mais largos. (figura xx)

Para contornar esse inconveniente, atualmente est\~{a}o sendo utilizadas,
por v\'{a}rias montadoras, molas que possuem uma leve curvatura no seu eixo
de a\c{c}\~{a}o, as comumente chamadas molas ''banana''. Esse tipo de mola
auxilia na redu\c{c}\~{a}o da componente $F_{E}$ na haste do amortecedor .

Pela equa\c{c}\~{a}o da dist\^{a}ncia B, mostrar que quanto maior a
dist\^{a}ncia C, menores ser\~{a}o aos esfor\c{c}os laterais.

Falar do acoplamento do amortecedor e da torre do amortecedor que pelo fato
de ser desacoplado, s\'{o} a mola suporta a carga vertical e que a For\c{c}a

\bigskip Fe atua no amortecedor j\'{a} que a mola s\'{o} pega carga no
sentido do seu eixo.\bigskip

\section{Suspens\~{a}o Traseira - Bra\c{c}os Arrastados Interconectados}

\bigskip Nos autom\'{o}veis com motor e tra\c{c}\~{a}o dianteiras, o sistema
de suspens\~{a}o do eixo traseiro torna-se relativamente simples. A solu\c{c}%
\~{a}o utilzando bra\c{c}os arrastados interconectados\footnote{%
O sistema \ e conhecido no idioma Ingl\^{e}s como Compound Crank Axel ou
Trailing Twist Axel. Em portugu\^{e}s normalmente \'{e} mencionado
simplesmente como Suspens\~{a}o Semi-Independente ou Bra\c{c}os Oscilantes
Longitudinais. O designa\c{c}\~{a}o utilizada neste trabalho \'{e} uma
tentativa de melhor descrever o sistema, embora devesse ser utilizada a
denomin\c{c}\~{a}o completa de Bra\c{c}os Arrastados Interconectados por
Viga Sujeita a Tor\c{c}\~{a}o.
\par
O termo Bra\c{c}os Arrastados vem do fato de o ponto de pivotamento dos bra%
\c{c}os oscilantes estar \`{a} frente dos centros das rodas.} \'{e}
largamente utilizada hoje em dia em autom\'{o}veis de pequeno e m\'{e}dio
porte. Segundo Bastow \cite{Bastow}, o Volkswagem Golf foi o pionerio na
utiliza\c{c}\~{a}o desse tipo de sistema.(ver figura \ref{Trailing Arm}\ )

\FRAME{fhFU}{4.9242in}{4.2237in}{0pt}{\Qcb{Suspens\~{a}o Traseira do
Volkswagem Golf e detalhe da bucha de fixa\c{c}\~{a}o dos bra\c{c}os
oscilantes \`{a} carroceria.}}{\Qlb{Trailing Arm}}{trseira.jpg}{\special%
{language "Scientific Word";type "GRAPHIC";maintain-aspect-ratio
TRUE;display "USEDEF";valid_file "F";width 4.9242in;height 4.2237in;depth
0pt;original-width 6.9998in;original-height 6.0001in;cropleft "0";croptop
"1";cropright "1";cropbottom "0";filename 'Trseira.jpg';file-properties
"XNPEU";}}

\bigskip As modernas suspens\~{o}es com bra\c{c}os arrastados
interconectados consistem em dois bra\c{c}os oscilantes arrastados
conectados entre si por meio de uma viga de sec\c{c}\~{a}o transversal em
forma de ''C ''ou ''U''\footnote{%
No in\'{i}cio do desenvolvimento desse tipo de suspens\~{a}o a viga de
conex\~{a}o dos bra\c{c}os oscilantes era contru\'{i}da com se\c{c}\~{a}o
transversal circular, por\'{e}m esa solu\c{c}\~{a}o carecia de uma junta que
permitisse a tor\c{c}\~{a}o da viga para que a suspens\~{a}o n\~{a}o fosse
muito rigida em Roll. A utiliza\c{c}\~{a}o de vigas com perfis abertos
permite que se tenha baixa rigidez torcional e alta rigidez a flex\~{a}o.}
que \'{e} rigidamente soldada aos bra\c{c}os oscilantes.

A simplicidade desse sistema de suspens\~{a}o que possui poucos componentes
e o pouco espa\c{c}o que ocupa na parte de tr\'{a}s do ve\'{i}culo,
possibilitanto que se aproveite ao m\'{a}ximo o porta malas, constituem as
principais caracteristicas \ que fazem o sistema de bra\c{c}os arrastados
interconectados ser t\~{a}o bem aceito e t\~{a}o empregado na industria
automobil\'{i}stica atual.

Do ponto de vista de projeto e ajuste da suspens\~{a}o, a grande vatagem
\'{e} o reduzido n\'{u}mero de caminhos de carga existentes para a rea\c{c}%
\~{a}o na carroceria dos esfor\c{c}os provenientes das rodas.\ Tipicamente,
as duas buchas transmitem para a carroceria todas as cargas laterais e
longitudinais enquanto as cargas verticais s\~{a}o absorvidas parte pelas
molas e parte pelas pr\'{o}prias buchas.

A posi\c{c}\~{a}o das buchas e das molas em rela\c{c}\~{a}o ao centro das
rodas \'{e} de grande influ\^{e}ncia na efici\^{e}ncia estrutural do
conjunto. A localiza\c{c}\~{a}o longitudinal das molas, por exemplo,
controla os momentos fletores do bra\c{c}o arrastado, enquanto a sua posi\c{c%
}\~{a}o transversal, afeta os momentos longitudinais dos mesmos.Um bom
exemplo de diferentes configura\c{c}\~{o}es do sistema de bra\c{c}os
arrastados interconectados est\'{a} na compara\c{c}\~{a}o das posi\c{c}%
\~{o}es das molas, buchas e amortecedores das suspens\~{o}es dos modelos de
m\'{e}dio porte das montadoras FIAT, GM e Volkswagem.

As v\'{a}rias posi\c{c}\~{o}es dos componentes da suspens\~{a}o s\~{a}o fun%
\c{c}\~{a}o principalmente do comportamento que se deseja do autom\'{o}vel,
tanto nos aspectos de conforto como de estabilidade, tra\c{c}\~{a}o,
frenagem e seguran\c{c}a.

No sistema de bra\c{c}os arrastados interconectados a viga que interconecta
os dois bra\c{c}os oscilantes, \'{e} segundo Stachell \cite{Satchell} , o
componente mais importante e cuja posi\c{c}\~{a}o tem grande influ\^{e}ncia
no comportamento cinem\'{a}tico da suspens\c{c}\~{a}o. A viga deve suportar
momentos fletores longitudinais e verticais com um m\'{i}nimo de
deflex\~{o}es e ao mesmo tempo permitir grandes tor\c{c}\~{o}es.Para cumprir
esses requisitos utilizam-se normalmente perfis abertos que possuem grandes
momentos de inercia \ e baixa rigidez \ torcional. Por\'{e}m a utiliza\c{c}%
\~{a}o de se\c{c}\~{o}es transversais n\~{a}o circulares para vigas sujeitas
\`{a} tor\c{c}\~{a}o acarreta no aparecimento de \textit{warping} \footnote{%
\bigskip \textit{Warping effect}, \'{e} efeito da perda de planicidade da se%
\c{c}\~{a}o transversal de uma barra n\~{a}o circular submetida a tor\c{c}%
\~{a}o. (Te\'{o}ria Cl\'{a}ssica da Tor\c{c}\~{a}o - Saint Venant)
\par
Em 1853 o engenheiro franc\^{e}s Adh\'{e}mar Jean Barr\'{e} de Saint-Venant
apresentou na Academia Francesa de Ci\^{e}ncias a teoria cl\'{a}ssica da tor%
\c{c}\~{a}o. Saint-Venant mostrou que quando uma barra n\~{a}o circular
\'{e} torcida, a se\c{c}\~{a}o transversal que era planar antes da tor\c{c}%
\~{a}o n\~{a}o mais permanecia planar ap\'{o}s a tor\c{c}\~{a}o.} o que faz
com que a regi\~{a}o da uni\~{a}o entre a viga e os bra\c{c}os oscilantes
esteja sujeita a tens\~{o}es mais altas se comparada as tens\~{o}es
decorrentes da utiliza\c{c}\~{a}o de vigas de se\c{c}\~{a}o transversal
circular.

Do ponto de vista cinem\'{a}tico a posi\c{c}\~{a}o da viga em rela\c{c}%
\~{a}o ao ponto de pivotamento da suspens\~{a}o na carroceria possue grande
influencia, e \'{e} trabalhando com essa vari\'{a}vel que os engenherios de
suspens\~{a}o podem ajustar o comportamento da suspens\~{a}o.

A figura (FIGURA) a seguir mostra as tr\^{e}s alternativas para o
posicionamento do centro de cisalhamento da viga ao longo do comprimento dos
bra\c{c}os oscilantes arrastados.

Posicionando-se \ o centro de cisalhamento da viga \ no mesmo alinhamento
das buchas (articula\c{c}\~{a}o dos bra\c{c}os oscilantes arrastados), ver
figura XXX-a, est\'{a} desempenhar\'{a} apenas o papel de uma barra
estabilizadora e o sistema se comporta de forma analoga a um sistema de
suspens\~{a}o independente por bra\c{c}os oscilantes paralelos. Quando a
viga \'{e} montada nessa posi\c{c}\~{a}o e a carroceria do ve\'{i}culo rola,
por exemplo ao realizar uma curva, o \'{u}nico carregamento a que est\'{a}
sujeita a viga \'{e} a tor\c{c}\~{a}o pura.

Com a viga posicionada entre as buchas e o centro das rodas (ver figura
XXX-b) o comportamento desse tipo de suspens\~{a}o, quando o autom\'{o}vel
rola, \ \'{e} semelhante ao comportamento de uma suspens\~{a}o Independende
de Bra\c{c}os Obl\'{i}quos Arrastados\footnote{%
A suspens\~{a}o aqui chamada de Independente de Bra\c{c}os Obl\'{i}quos
Arrastados \'{e} conhecida no idioma Ingl\^{e}s por Semi-Trailing Link
\par
Esse tipo de suspens\~{a}o traseira \'{e} encontrada no Omega da GM e nos
BMW s\'{e}rie 5, por exemplo.} (ver figura XXX1). Para essa configura\c{c}%
\~{a}o e com o ve\'{i}culo em rolamento, o ponto onde se cruzam o centro de
cisalhamento da viga de interconex\c{c}\~{a}o e a linha de centro
imagin\'{a}ria do ve\'{i}culo, encontra-se o segundo ponto de articula\c{c}%
\~{a}o para a correspond\^{e}ncia com o sistema \ por Bra\c{c}os
Obl\'{i}quos Arrastados.

A posi\c{c}\~{a}o intermedi\'{a}ria da viga faz com que esta esteja sujeita
a carregamentos mais severos e principalmente carregamentos combinados de
flex\~{a}o e tor\c{c}\~{a}o, o que aumenta a import\^{a}ncia de uma boa
an\'{a}lise de tens\~{o}es na regi\~{a}o da solda que une os bra\c{c}os
oscilantes \`{a} viga.

A terceira op\c{c}\~{a}o em rela\c{c}\~{a}o ao posicionamento da viga \'{e}
o alinhamento com os centro das rodas (ver figura XXX-c). Nesta posi\c{c}%
\~{a}o, segundo Stachell \cite{Satchell}, \'{e} necess\'{a}rio um componente
adicional com o objetivo de resistir aos esfor\c{c}os laterais que o sistema
original n\~{a}o \'{e} capaz de absorver, Satchell \cite{Satchell}.
Normalmente, utiliza-se a chamada barra Panhard como o elemento para
resistir aos esfor\c{c}os laterais. Um exemplo da utiliza\c{c}\~{a}o desse
tipo de combina\c{c}\~{a}o esteve presente nas primeiras vers\~{o}es do Ford
Fiesta.

\subsection{As for\c{c}as atuantes}

\bigskip

\pagebreak

\chapter{Conforto Vibracional}

\section{Introdu\c{c}\~{a}o}

Ao utilizar um autom\'{o}vel deseja-se que este isole os ocupantes das vibra%
\c{c}\~{o}es induzidas pelas irregularidades da pista, pelas massas
rotativas (conjunto rodas/pneus/montantes), motor e sistemas de
transmiss\~{a}o. Essas vibra\c{c}\~{o}es s\~{a}o percebidas pelos ocupantes
do ve\'{i}culo pelo tato, audi\c{c}\~{a}o e vis\~{a}o. Alguns autores
estabelecem uma divis\~{a}o bem definida entre as v\'{a}rias frequ\^{e}ncias
de vibra\c{c}\~{a}o experimentadas pelos ocupantes de autom\'{o}veis. Para
autores como Gillespe\cite{Gillespie} a divis\~{a}o \'{e} a seguinte:

-RIDE (0-25Hz) - Vibra\c{c}\~{o}es percebidas pelo tato e visualmente;

-NOISE (25-20000Hz) - Vibra\c{c}\~{o}es percebidas pela audi\c{c}\~{a}o.

Justifica-se essa divis\~{a}o do ponto de vista dos limites de audi\c{c}%
\~{a}o do ser humano, j\'{a} que o limite inferior da faixa de
frequ\^{e}ncia aud\'{i}vel \'{e} aproximadamente 25Hz. Embora a divis\~{a}o
te\'{o}rica seja clara \'{e} dific\'{i}l considerar uma ou outra
separadamente, pois normalmente as duas se apresentam simultaneamente, o
ruido (NOISE) est\'{a} presente normalmente quando as baixas frequ\^{e}ncias
s\~{a}o exitadas.

Em outras publica\c{c}\~{o}es o espectro de frequ\^{e}ncias que corresponde
a faixa mais impotante para o estudo do conforto de autom\'{o}veis est\'{a}
compreendido em uma faixa um pouco diferente da citada acima. De acordo com
Reimpell,\cite{Reimpell} os ocupandtes de um autom\'{o}vel est\~{a}o
sujeitos a uma combina\c{c}\~{a}o de acelera\c{c}\~{o}es e vibra\c{c}\~{o}es
mec\^{a}nicas que se apresentam em uma faixa bem ampla de frequ\^{e}ncia. A
faixa de frequ\^{e}ncia considerada cr\'{i}tica para o estudo de conforto de
um autom\'{o}vel est\'{a} entre 1 e 80 Hz. Dentro dessa faixa de
frequ\^{e}ncia tida como mais cr\'{i}tica ainda \'{e} sugerida uma separa\c{c%
}\~{a}o em RIDE\ ou SPRINGING COMFORT que engloba a faixa entre 1 e 4 Hz e
ROAD\ HARSHNESS que corresponde a frequ\^{e}ncias acima de 4Hz. A separa\c{c}%
\~{a}o das faixas de frequ\^{e}ncia \'{e} importante pois as frequ\^{e}ncias
s\~{a}o percebidas de forma diferente pelo corpo humano e al\'{e}m disso
torna mais f\'{a}cil a distin\c{c}\~{a}o da influ\^{e}ncia de cada
componente do sistema de suspens\~{a}o na resposta din\^{a}mica do
ve\'{i}culo completo.

A faixa de frequ\^{e}ncia entre 1 e 4 Hz \'{e} percebida principalmente pela
parte superior do corpo humano, enquanto as frequ\^{e}ncias acima de \ 4 Hz
s\~{a}o percebidas pelos ocupantes do ve\'{i}culo no assoalho e assentos do
ve\'{i}culo e o motorista pode sent\'{i}-las no volante.

O estudo do conforto vibracional de um ve\'{i}culo concentra-se
principalmente na faixa de vibra\c{c}\~{o}es de baixa frequ\^{e}ncia e
est\'{a} relacionado \`{a} resposta do o autom\'{o}vel a determinado tipo de
exita\c{c}\~{a}o. A resposta do autom\'{o}vel determina a magnitude e dire\c{%
c}\~{o}es das vibra\c{c}\~{o}es que ser\~{a}o impostas ao habit\'{a}culo e
finalmente a percep\c{c}\~{a}o dos passageiros do comportmanento do
autom\'{o}vel.

As fontes de exita\c{c}\~{a}o de um autom\'{o}vel podem ser divididas em
internas e externas:

\section{Fontes internas de Excita\c{c}\~{a}o}

As fontes internas de excita\c{c}\~{a}o s\~{a}o escencialmente tr\^{e}s: O
sistema de transmiss\~{a}o (caixa de c\^{a}mbio, diferencial, cardan) , o
motor e o conjunto rodas/pneus. A principal causa dessas excita\c{c}\~{o}es
s\~{a}o as partes rotantes que produzem excita\c{c}\~{o}es din\^{a}micas
devido a desbalanceamento. Um rotor est\'{a} perfeitamente balanceado quando
o seu eixo de rota\c{c}\~{a}o coincide com um dos seus eixos principais de
in\'{e}rcia, entretanto essa condi\c{c}\~{a}o \'{e} dif\'{i}cil de ser alcan%
\c{c}ada devido aos desvios de produ\c{c}\~{a}o e as toler\^{a}ncias de
projeto.

Como uma consequ\^{e}ncia do desbalanceamento o componente rotante exerce
sobre o seu suporte uma for\c{c}a cuja frequ\^{e}ncia \'{e} proporcional
\`{a} velocidade angular $\varpi $ e a amplitude proporcional ao quadrado
dessa velocidade $\varpi 
%TCIMACRO{\UNICODE[m]{0xb2}}%
%BeginExpansion
{{}^2}%
%EndExpansion
$. Na equa\c{c}\~{a}o a seguir tem-se que $m$ representa a massa
desbalanceada, $e$ a ecentricidade da massa e $\varpi $ a velocidade angular
do componente desbalanceado.

\begin{equation}
F_{0.}\sin \varpi .t=(m.e.\varpi ^{2})\sin \varpi .t  \label{eq.vibr01}
\end{equation}

\bigskip \bigskip Esse fato faz com que se possa caracterizar bem cada
componente j\'{a} que o motor, o sistema de transmiss\~{a}o e as rodas giram
a velocidades diferentes, dessa forma a frequ\^{e}ncia excitada por cada
componente est\'{a} em uma faixa diferente.

\subsection{Conjunto Pneum\'{a}ticos e Rodas}

Al\'{e}m das excita\c{c}\~{o}es devido a desbalanceamentos, h\'{a} outros
efeitos que s\~{a}o particulares de cada componente rotante. As rodas, por
exemplo, podem ter irregularidades geometricas. A ovaliza\c{c}\~{a}o da
forma dos pneus, por exemplo, excita frequ\^{e}ncias da ordem de $2\varpi $
e uma deforma\c{c}\~{a}o de forma triangular frequ\^{e}ncias de \ $3\varpi $%
, e assim sucessivmente. A banda de rodagem do pneum\'{a}tico, que possui
sulcos, excita altas frequ\^{e}ncias que se encontram normalmente na faixa
do espectro aud\'{i}vel (ruido). Alguns tipos de pneus s\~{a}o capazes de
produzir um ru\'{i}do t\~{a}o caracteristico que \'{e} facilmente
identificado por ouvidos experientes.De acordo com Genta\cite{Genta} os
sulcos da banda de rodagem s\~{a}o idealizados de forma irregular, com
elementos espa\c{c}ados aleat\'{o}riamente para evitar uma forte excita\c{c}%
\~{a}o com um periodo igual a passagem de um dos elementos da banda de
rodagem pela pista.

\bigskip As causas da n\~{a}o uniformidade dos pneus s\~{a}o resultado da
sua pr\'{o}pria caracteristica construtiva que apresenta varia\c{c}\~{o}es
de espessura e rigidez ao longo da sua estrutura. (figura) Aliado a isso
desvios causados pelos processos de fabrica\c{c}\~{a}o resultam em vaira\c{c}%
\~{o}es ao longo da circunfer\^{e}ncia do tipo:

-espessura

-massa

-rigidez

essas varia\c{c}\~{o}es s\~{a}o causadoras de desbalanceamentos, varia\c{c}%
\~{o}es nas for\c{c}as radiais, laterais e longitudinais e vari\c{c}\~{o}es
geometricas na banda de rodagem e nos flancos do pneu durante a rota\c{c}%
\~{a}o.

Dos fen\^{o}menos descritos acima o que aparece de forma mais significativa
\'{e} a varia\c{c}\~{a}o da for\c{c}a radial devido \`{a} varia\c{c}\~{a}o
de rigidez do pneum\'{a}tico ao longo da sua circunfer\^{e}ncia.(figura)

\bigskip \FRAME{fhFU}{3.4679in}{1.7988in}{0pt}{\Qcb{Varia\c{c}\~{a}o da
rigidez radial de um pneu ao longo da sua cirunfer\^{e}ncia.}}{}{tire.jpg}{%
\special{language "Scientific Word";type "GRAPHIC";maintain-aspect-ratio
TRUE;display "USEDEF";valid_file "F";width 3.4679in;height 1.7988in;depth
0pt;original-width 8.1768in;original-height 4.2289in;cropleft "0";croptop
"1";cropright "1";cropbottom "0";filename 'tire.jpg';file-properties
"XNPEU";}}

De acordo com Reimpell\cite{Reimpell}, uma varia\c{c}\~{a}o de $\pm 5\%$ \
na rigidez pode ser esperada em pneus radiais (ex.:175 R14 88 S). Essa varia%
\c{c}\~{a}o de rigidez gera uma varia\c{c}\~{a}o do raio din\^{a}mico de rota%
\c{c}\~{a}o do pneu que causa vari\c{c}\~{o}es da for\c{c}a vertical $\
\Delta Fz$. Normalmente essas varia\c{c}\~{o}es de for\c{c}a s\~{a}o
inferiores as for\c{c}as de atrito do sistema de suspens\~{a}o. Se o
autom\'{o}vel trafega em uma pista suficientemente lisa essa varia\c{c}%
\~{a}o de $Fz$ ser\'{a} percebida pelos ocupantes do ve\'{i}culo, j\'{a} que
a for\c{c}a vertical que atua nos pneus n\~{a}o ser\'{a} suficiente para
fazer a suspens\~{a}o trabalhar, com isso toda a varia\c{c}\~{a}o de $Fz$
ser\'{a} transmitida diretamente para os apoios da suspens\~{a}o na
carroceria.

\qquad \emph{Se o autom\'{o}vel trafegar por uma pista que induza varia\c{c}%
\~{o}es de }$Fz$\emph{\ maiores que as induzidas pelos pneus a vibra\c{c}%
\~{a}o causada por ambas pode ser filtrada pelo sistema de suspens\~{a}o do
ve\'{i}culo. ( Devo colocar essa observa\c{c}\~{a}o??????)}

As varia\c{c}\~{o}es de rigidez do pneum\'{a}tico ao longo da sua
circunferencia resultam em for\c{c}as din\^{a}micas com frequ\^{e}ncias
iguais \`{a} frequ\^{e}ncia de rota\c{c}\~{a}o e seus m\'{u}ltiplos.

Os desvios dimensionais existentes na fabrica\c{c}\~{a}o das rodas e pneus
n\~{a}o produzem grandes varia\c{c}\~{o}es de rigidez, por\'{e}m geram for\c{%
c}as e deslocamentos no eixo do ve\'{i}culo enquanto a roda gira. A montagem
pneu/roda \'{e} um sistema mec\^{a}nico complexo que filtra grande uma parte
das frequ\^{e}ncias produzidas na interface pneu/pista. As altas
frequ\^{e}ncias s\~{a}o firltradas primeiramente pelos pneus, depois pelo
sistema de suspens\~{a}o, e finalmente s\~{a}o percebidas a bordo
principalmente como ruido e vibra\c{c}\~{o}es do interior (paineis de porta,
painel de instrumentos, tampas do porta-luvas, etc).

\subsection{Comportamento din\^{a}mico dos pneus}

O comportamento din\^{a}mico dos pneus tem grande import\^{a}ncia no estudo
do conforto e estabiliade/manobrabilidade de ve\'{i}culos. Para o estudo do
conforto vibracional \ de um ve\'{i}culo, os pneus s\~{a}o os primeiros
filtros das irregularidades provenientes da \ pista e s\~{a}o eles ao mesmo
tempo fontes de exita\c{c}\~{a}o.

Sendo a rigidez dos pneus maior que a rigidez das suspens\~{o}es, o
comportamento dos mesmos geralmente n\~{a}o \'{e} muito importante em vibra%
\c{c}\~{o}es de baixas frequ\^{e}ncias (1a 3 Hz), para essas frequ\^{e}ncias
o pneu pode ser modelado como um corpo r\'{i}gido.

Para uma faixa de frequ\^{e}ncias um pouco mais altas que as anteriores (10
a 20 Hz) o pneu pode ser considerado como um elemento deform\'{a}vel com
massa desprez\'{i}vel.

Para aplica\c{c}\~{a}o na modelagem matem\'{a}tica de ve\'{i}culos , ''\'{e}
poss\'{i}vel mostrar que o pneu \ deve ser caracterizado por uma baixa
rigidez vertical, de forma a minimizar os deslocamentos verticais das massas
suspensas em uma superf\'{i}cie irregular, e uma rigidez transversal alta,
para reagir com pequenos deslocamentos a pequenas for\c{c}as aplicadas ao
ve\'{i}culo.''\cite{Genta}

H\'{a} diferen\c{c}as de rigidez entre os tipos construtivos de pneus, os
radiais que apresentam \ rigidez vertical menor que os diagonais.\ Portanto
os pneus radias possuem uma vantagem, do ponto de vista de conforto, em rela%
\c{c}\~{a}o aos diagonais. \emph{Hoje em dia a utiliza\c{c}\~{a}o de pneus
diagonais est\'{a} bastante reduzida e restrita a caminh\~{o}es, \^{o}nibus,
autom\'{o}veis de competi\c{c}\~{a}o (F1) etc....(Checar este
coment\'{a}rio...)}

A frequ\^{e}ncias mais altas, acima de 50 Hz, os pneus vibam na sua
frequ\^{e}ncia natural. Os pneus radias possuem frequ\^{e}ncias naturais que
variam de 60 a 90 Hz dependendo das suas dimens\~{o}es.\bigskip

\subsection{Sistema de Transmiss\~{a}o}

As exita\c{c}\~{o}es oriundas do sistema de transmiss\~{a}o podem ter
tr\^{e}s fontes diferentes, a caixa de c\^{a}mbio, o diferencial e o eixo de
transmiss\~{a}o.Essa divis\~{a}o aplica-se principalmente ao autom\'{o}veis
com a configura\c{c}\~{a}o motor e caixa dianteiros e tra\c{c}\~{a}o
traseira, entretanto pode ser aplicada tamb\'{e}m aos autom\'{o}veis com
motores dianteiros transversais e diferenciais montados dentro da mesma carca%
\c{c}a da caixa de c\^{a}mbio. Dentre esses componentes o eixo de
transmiss\~{a}o \'{e} o elemento que possui o maior potencial em \ gerar
exita\c{c}\~{o}es de frequ\^{e}ncias at\'{e} 20 Hz. Em carros de passeio com
tra\c{c}\~{a}o traseira ou tra\c{c}\~{a}o nas quatro rodas e em pequenos
caminh\~{o}es utilizam-se um \'{u}nico eixo, enquanto em caiminh\~{o}es
maiores e \^{o}nibus o eixo de transmiss\~{a}o \'{e} montado utilizando
v\'{a}rios segmentos suportados por mancais de rolamento. As exita\c{c}%
\~{o}es provenientes dos eixos podem ser geradas por dois fen\^{o}menos; um
poss\'{i}vel desbalanceamento de massa do eixo, e os momentos
secund\'{a}rios provocados pelo \^{a}ngulo formado entre dois semi-eixos
conectados por juntas universais.

As exita\c{c}\~{o}es, devido ao desbalanceamento, produzidas pelo eixo de
transmiss\~{a}o s\~{a}o normalmente devido \`{a} combina\c{c}\~{a}o dos
seguintes fatores:

-Assimetria das partes girat\'{o}rias;

-O eixo pode estar fora de centro em rela\c{c}\~{a}o aos flanges de
uni\~{a}o;

-O eixo pode n\~{a}o estar completamente reto;

-As toler\^{a}ncias de motagem podem fazer com que o eixo saia fora do
centro;

-O eixo, por ser um componente el\'{a}stico, pode defletir e causar o
desbalanceamento.

Nos ve\'{i}culos que utilizam juntas universais para o acoplamento dos eixos
de entrada e sa\'{i}da, o momento secund\'{a}rio gerado quando os eixos
trabalham em angulo \'{e} uma importante fonte de excita\c{c}\~{a}o. A
caracteriza\c{c}\~{a}o desse momento secund\'{e}rio \'{e} obtida com uma
simples soma vetorial dos torques de entrada e sa\'{i}da na junta universal,
como ilustrado a seguir.\cite{Gillespie}(figura)

\FRAME{fhFU}{1.7106in}{1.3543in}{0pt}{\Qcb{Momento secund\'{a}rio em uma
junta universal}}{}{hook_joint.jpg}{\special{language "Scientific Word";type
"GRAPHIC";maintain-aspect-ratio TRUE;display "USEDEF";valid_file "F";width
1.7106in;height 1.3543in;depth 0pt;original-width 4.9684in;original-height
3.9271in;cropleft "0";croptop "1";cropright "1";cropbottom "0";filename
'hook_joint.jpg';file-properties "XNPEU";}}

A magnitude \ do momento secund\'{a}rio \'{e} proporcional ao torque
aplicado ao eixo e ao \^{a}ngulo da junta universal, quando h\'{a} varia\c{c}%
\~{a}o do torque devido \`{a} pulsa\c{c}\~{a}o do motor o momento
secund\'{a}rio vaira proporcionalmente. \'{E} importante notar que o
\^{a}ngulo da junta universal pode variar de acordo com condi\c{c}\~{a}o de
carregamento do ve\'{i}culo, onde geralmente o eixo de entrada move-se
solid\'{a}rio \`{a} carroceria e o eixo de sa\'{i}da move-se junto com o
diferencial e conjunto de suspens\~{a}o. Em alguns conceitos mais modernos o
diferencial est\'{a} solid\'{a}rio a um sub-chassi e essa varia\c{c}\~{a}o
de \^{a}ngulo na junta universal n\~{a}o ocorre ou \ \'{e} bastante
reduzida.(figura)

Nos modernos ve\'{i}culos com motores e tra\c{c}\~{a}o dianteira a utiliza\c{%
c}\~{a}o de juntas homocin\'{e}ticas junto \'{a}s rodas tem reduzido
bastante as excita\c{c}\~{o}es devido a juntas, por\'{e}m ainda deve-se
tomar cuidado com os semi-eixos nas montagens motor e c\^{a}mbio
transversais.

\subsection{Motor}

Os motores alternativos utilizados nos ve\'{i}culos, onde a transmiss\c{c}%
\~{a}o de torque se d\'{a} de forma ciclica, as massas rotantes s\~{a}o
desbalanceadas e h\'{a} grandes for\c{c}as de in\'{e}rcia, s\~{a}o
considerados a principal fonte interna de vibra\c{c}\~{a}o e ruido dos
ve\'{i}culos. Essas vibra\c{c}\~{o}es tem uma frequ\^{e}ncia natural , em
motores de quatro cilindros, igual a metade da rota\c{c}\~{a}o do motor, mas
um grande n\'{u}mero de harmonicos tamb\'{e}m est\'{a} presente.\cite{Genta}

Para tentar minimizar as vibra\c{c}\~{o}es oriundas do motor v\'{a}rias solu%
\c{c}\~{o}es foram adotadas. Atualmente alguns motores de quatro e cinco
cilindros utilizam eixos que giram em sentido contrario ao do vibrabrequim \
e com o dobro da sua velocidade. Frente ao efeito das massa excentricas
presentes nos eixos contra-rotantes, s\~{a}o atenuadas consideravelmente as
vibra\c{c}\~{o}es devido \`{a}s tipicas for\c{c}as alternadas dos motores de
quatro cilindros.\cite{Alonso} (figura)

\FRAME{fhFU}{2.3289in}{1.8766in}{0pt}{\Qcb{Raio-X de um motor que apresenta
\'{a}rvores contra rotatnes}}{}{motor_came.jpg}{\special{language
"Scientific Word";type "GRAPHIC";maintain-aspect-ratio TRUE;display
"USEDEF";valid_file "F";width 2.3289in;height 1.8766in;depth
0pt;original-width 5.5936in;original-height 4.4996in;cropleft "0";croptop
"1";cropright "1";cropbottom "0";filename 'motor_came.jpg';file-properties
"XNPEU";}}

As vibra\c{c}\~{o}es produzidas pelo giro do eixo comando de v\'{a}lvulas,
ficam atenuadas consideravelmente mediante um projeto especial que consiste
em fazer excentricas as partes intermedi\'{a}rias do eixo entre os cames.

Com um projeto adequado dos coxins e suportes, a massa do grupo
moto-propurlsor pode ser utilizado como um absorvedor de vibra\c{c}\~{o}es
atenuando as vibra\c{c}\~{o}es \`{a}s quais o ve\'{i}culo est\'{a} sujeito.
Na maioria das vezes \'{e} utilizada para o controle de vibra\c{c}\~{o}es
verticais provenientes das excita\c{c}\~{o}es das rodas. Para esse
prop\'{o}sito\ o sistema de suporte do grupo moto-propulsor \'{e} projetado
para ter a frequ\^{e}ncia de resson\^{a}ncia vertical pr\'{o}xima da
frequ\^{e}ncia de resson\^{a}ncia das rodas (wheel hope) 12-15 Hz, desta
forma o moto-propulsor pode atuar como um amortecedor de vibra\c{c}\~{o}es
para esse modo de vibra\c{c}\~{a}o do ve\'{i}culo.\cite{Gillespie}\FRAME{fhFU%
}{2.808in}{1.8481in}{0pt}{\Qcb{Coxin e suporte do motor.}}{}{motor_suport.jpg%
}{\special{language "Scientific Word";type "GRAPHIC";maintain-aspect-ratio
TRUE;display "USEDEF";valid_file "F";width 2.808in;height 1.8481in;depth
0pt;original-width 11.2607in;original-height 7.3959in;cropleft "0";croptop
"1";cropright "1";cropbottom "0";filename 'motor_suport.jpg';file-properties
"XNPEU";}}

Os pontos de entrada das vibra\c{c}\~{o}es provenientes do grupo
moto-propulsor na carro\c{c}eria s\~{a}o os coxins dos suportes do motor e
c\^{a}mbio, as fixa\c{c}\~{o}es dos bra\c{c}os oscilantes \`{a} carro\c{c}%
eria e os suportes do sistema de exaust\~{a}o do motor.(figura) A liga\c{c}%
\~{a}o entre o coletor de escape dos motores e cano de escape \'{e}
normalmente feita por uma junta flex\'{i}vel com o intuito de minimizar a
transmiss\~{a}o de vibra\c{c}\~{o}es para o cano e por sua vez para a
carroceria.

\FRAME{fhFU}{3.1868in}{2.2762in}{0pt}{\Qcb{Pontos de entrada das vibra\c{c}%
\~{o}es provenientes do motor/c\^{a}mbio.}}{}{motor_input.jpg}{\special%
{language "Scientific Word";type "GRAPHIC";maintain-aspect-ratio
TRUE;display "USEDEF";valid_file "F";width 3.1868in;height 2.2762in;depth
0pt;original-width 11.4268in;original-height 8.1457in;cropleft "0";croptop
"1";cropright "1";cropbottom "0";filename 'motor_input.jpg';file-properties
"XNPEU";}}

\bigskip

\section{Fontes Externas de Vibra\c{c}\~{a}o}

\subsection{Excita\c{c}\~{a}o devido \`{a}s irregularidades da pista}

As excita\c{c}\~{o}es devido a irregularidades da pista em que trafega um
ve\'{i}culo \'{e} de extrema import\^{a}ncia no estudo do conforto de um
autom\'{o}vel. As irregularidades da superf\'{i}cie da estrada ou pista por
onde um autom\'{o}vel trafega se apresentam de modo aleat\'{o}rio, por\'{e}m
podem ser classificadas, para objeto de estudo, da seguinte forma segundo
Montiglio:\cite{Montiglio}

\textbf{Longas ondula\c{c}\~{o}es}: Varia\c{c}\~{o}es de altura do perfil da
estrada onde o comprimento de onda \'{e} superior \`{a} dist\^{a}ncia entre
eixos do ve\'{i}culo. O perfil varia de forma progressiva e
sim\'{e}trica.Esse tipo de irregularidade \'{e} principalmente devido ao
assentamento de parte do terreno onde foi construida a estrada ou em grandes
viadutos ou pontes.

\textbf{Aspereza}: Varia\c{c}\~{o}es de altura independentes do perfil da
estrada e que possuem dimens\~{o}es compat\'{i}veis com o raio das rodas do
ve\'{i}culo. O perfil varia de modo brusco e normalmente n\~{a}o \'{e}
sim\'{e}trico. S\~{a}o causados principalmente por jun\c{c}\~{o}es, danos na
pista (buracos) e passagens de n\'{i}vel.

\textbf{Descontinuidades}: Varia\c{c}\~{o}es cont\'{i}nuas do perfil da
pista em dimens\~{o}es inferiores \`{a}s do contato pneu/piso. O perfil
vaira de modo brusco e casual, normalmente ocorre por danos no manto de
asfalto ou a superf\'{i}cies n\~{a}o asfaltadas. Como exemplo pode ser
citado o pav\'{e} belga.

O quadro comparativo a seguir ilustra a faixa de frequ\^{e}ncia de cada uma
das classifica\c{c}\~{o}es das pistas.(figura)

Da classifica\c{c}\~{a}o dada por Montiglio as longas ondula\c{c}\~{o}es e a
aspereza s\~{a}o fontes de exita\c{c}\~{a}o deterministicas no tempo. Alguns
exemplos de fun\c{c}\~{o}es deterministicas que podem reproduzir esses tipos
de excita\c{c}\~{a}o s\~{a}o as fun\c{c}\~{o}es degrau, rampa e fun\c{c}%
\~{o}es harm\^{o}nicas.

As irregularidades do manto asf\'{a}ltico s\~{a}o inerentes aos m\'{e}todos
de constru\c{c}\~{a}o da pista e \`{a} manuten\c{c}\~{a}o da mesma, esse
tipo de desvio \'{e} caracterizado como n\~{a}o deterministico no tempo.

As irreguladidades da superf\'{i}cie da pista s\~{a}o descritas pelas eleva%
\c{c}\~{o}es do perfil \ pelo qual os pneus de um ve\'{i}culo passam. O
perfil da superf\'{i}cie varia de forma aleat\'{o}ria tanto na dira\c{c}%
\~{a}o longitudinal como na transversal. O perfil de pista, por suas varia\c{%
c}\~{o}es de eleva\c{c}\~{a}o aleat\'{o}rias, pode ser incluido,
matematicamente, na categoria de sinal rand\^{o}mico de banda larga
(broad-band random signals) e pode ser descrito pela pr\'{o}pria eleva\c{c}%
\~{a}o do perfil \ ou por suas propriedades estat\'{i}sticas.

A partir de experimentos, utilizando \ ''perfilometros '', o perfil da pista
\'{e} medido e a sua PSD \'{e} obtida fazendo-se a Transformada de Fourier
do perfil medido tomando-se como base uma medida de comprimento como
refer\^{e}ncia. A figura \ref{gentagraph01} a seguir mostra algumas PSDs
t\'{i}picas de pistas de asfalto, concreto e ''pav\'{e}s''\footnote{%
O termo''Pav\'{e}'' vem de''Belgian Pav\'{e}''. O ''Pav\'{e}'' original
\'{e} do produto da falta de manuten\c{c}\~{a}o e esquecimento, durante a
segunda grande Guerra, das t\'{i}picas estradas constu\'{i}das de blocos de
pedra encontradas na B\'{e}lgica, Fran\c{c}a, Paises Baixos e Alemanha.\cite
{Bastow}} e as curvas propostas pela ISO para classificar os perfis de pista.%
\FRAME{fhFU}{3.8035in}{3.5129in}{0pt}{\Qcb{PSD t\'{i}picas de pistas de
asfalto, concreto e ''paves''.}}{\Qlb{gentagraph01}}{grapf01_genta.jpg}{%
\special{language "Scientific Word";type "GRAPHIC";maintain-aspect-ratio
TRUE;display "USEDEF";valid_file "F";width 3.8035in;height 3.5129in;depth
0pt;original-width 8.6663in;original-height 8.0004in;cropleft "0";croptop
"1";cropright "1";cropbottom "0";filename
'grapf01_genta.jpg';file-properties "XNPEU";}}

Mesmo que para cada pista as PSDs correspondentes sejam diferentes
observa-se um comportamento t\'{i}pico que \'{e} a queda na amplitude a
medida que aumenta o valor do n\'{u}mero de onda. '' Isso reflete
simplesmente o fato de que os desvios na superf\'{i}cie da pista da ordem de
milhares de p\'{e}s de comprimento podem ter amplitudes de polegadas.''\cite
{Gillespie}

O fato de as v\'{a}rias PSDs de pistas diferentes possuirem um comportamento
caracter\'{i}sto de queda da amplitude com o aumento do n\'{u}mero de onda
nos mostra que o conceito de propriedades m\'{e}dias para determinar perfis
de pista pode ser muito \'{u}til no estudo da resposta de autm\'{o}veis
\`{a}s excita\c{c}\~{o}es da pista.

Os modelos de perfis de pista s\~{a}o usualmente modelados como uma
amplitude que diminui com a frequ\^{e}ncia na segunda ou quarta ordem. Os
modelos, usualmente, utilizam a combina\c{c}\~{a}o de uma sequ\^{e}ncia de
n\'{u}meros aleat\'{o}rios e uma curva aproximada por polin\^{o}mios\cite
{Sayers01},\cite{Gillespie}, \cite{Montiglio}, \cite{Genta}, \cite{ElBeheiry
and Karnopp}, \cite{X.P.Lu}, \cite{Dodds} embora existam propostas que se
baseiam na obten\c{c}\~{a}o da PSD utilizando uma fun\c{c}\~{a}o exponencial 
\cite{Tamboli and Joshi} e \cite{Kozin}.

Na literatura especializada encontram-se normalmente duas formas
anal\'{i}ticas distintas para a constru\c{c}\~{a}o mat\'{e}matica de uma PSD
representativa para pistas t\'{i}picas de conforto. Autores como Sayers \cite
{Sayers01}, Genta \cite{Genta}, Montiglio \cite{Montiglio}, ElBeheiry e
Karnopp \cite{ElBeheiry and Karnopp}, Lu e Segel \cite{X.P.Lu} propuseram e
desenvolveram trabalhos com seguinte equa\c{c}\~{a}o:

\begin{equation}
S(\gamma )=G_{o}.\gamma ^{-n}  \label{PSD01}
\end{equation}

Por outro lado, Gillespie \cite{Gillespie} e Montiglio \cite{Montiglio}
apresentam a seguinte formula\c{c}\~{a}o para a representa\c{c}\~{a}o da
pista.

\begin{equation}
S\left( \gamma \right) =\frac{\left( G_{o}.\gamma ^{n}+\gamma
_{o}^{n}\right) }{\gamma ^{n}.\left( 2.\pi .\gamma \right) ^{n}}
\label{merda}
\end{equation}

Onde tem-se:

$\S(\gamma )$\ = Amplitude da PSD [$m^{2}/ciclos/m$]$;$

$\gamma $ = N\'{u}mero de onda [$cilos/m$]$;$

$G_{o}$=Coeficiente de Rugosidade [$m^{2}\times ciclos/m$]$;$

$\gamma _{o}$= Corte do N\'{u}mero de onda [$cilos/m$]$;$

$n$ = Expoente adimensional.

Dos autores citados acima, Gillespie\cite{Gillespie}, Lu\cite{X.P.Lu},
ElBeheiry e Karnopp\cite{ElBeheiry and Karnopp} e Sayers\cite{Sayers01}
prop\~{o}e a utiliza\c{c}\~{a}o do expoente $n$ igual a 2, enquanto Genta 
\cite{Genta}, Dodds\cite{Dodds} e Montiglio\cite{Montiglio} apresentam o
expoente $n$ variando entre 1.36 e 3.8.

De acordo com\ Genta\cite{Genta}, as normas ISO sugerem que se utiliza $n=2$
para irregularidades com comprimento de onda maiores que $6m$ e $n=1.37$
para irregularidades com comprimentos de onda menores que o limite
mencionado.

Alguns valores para o Coeficiente de Rugosidade $G_{o}$ utilizados por Lu 
\cite{X.P.Lu01}, ElBeheiry e Karnopp\cite{ElBeheiry and Karnopp} est\~{a}o
apresentados na tabela a seguir.

\begin{equation}
\begin{tabular}{||l||l||}
\hline\hline
Tipo de Pista & Coeficiente de Rugosidade da Pista $\left[ m^{2}\times
ciclos/m\right] $ \\ \hline\hline
A (Muito Boa) & \multicolumn{1}{||c||}{$0.001\times 10^{-4}$} \\ \hline
B (Boa) & \multicolumn{1}{||c||}{$0.004\times 10^{-4}$} \\ 
C (M\'{e}dia) & \multicolumn{1}{||c||}{$0.016\times 10^{-4}$} \\ 
D (Irregular) & \multicolumn{1}{||c||}{$0.064\times 10^{-4}$} \\ 
E (Muito Irregular) & \multicolumn{1}{||c||}{$0.259\times 10^{-4}$} \\ 
\hline\hline
\end{tabular}
\label{TabelaGo}
\end{equation}

O gr\'{a}fico a seguir mostra PSDs t\'{i}picas para cinco tipos de pista
utilizando a classifica\c{c}\~{a}o e os valores de $G_{o}$ propostos por Lu
e ElBeheiry\cite{X.P.Lu},\cite{ElBeheiry and Karnopp} e apresentados na
tabela\ref{TabelaGo}.\FRAME{fhFU}{3.122in}{3.5743in}{0pt}{\Qcb{PSD de
algumas pistas t\'{i}picas segundo a classifica\c{c}\~{a}o apresentada na
tabela anterior.}}{}{psd01.jpg}{\special{language "Scientific Word";type
"GRAPHIC";maintain-aspect-ratio TRUE;display "USEDEF";valid_file "F";width
3.122in;height 3.5743in;depth 0pt;original-width 3.9998in;original-height
4.5835in;cropleft "0";croptop "1";cropright "1";cropbottom "0";filename
'PSD01.jpg';file-properties "XNPEU";}}

Da forma como foi apresentada at\'{e} agora a PSD, ela \'{e} uma fun\c{c}%
\~{a}o do espa\c{c}o e n\~{a}o uma fun\c{c}\~{a}o do tempo. A representa\c{c}%
\~{a}o de uma PSD de pista pode ser mais \'{u}til se apresentada em fun\c{c}%
\~{a}o do tempo.

A intensidade e a forma como as vibra\c{c}\~{o}es provenientes de uma pista
s\~{a}o impostas a um autom\'{o}vel que por ela trafega s\~{a}o,
basicamente, fun\c{c}\~{o}es da rugosidade da pista e da velocidade com que
o autom\'{o}vel trafega. Pode-se dessa forma obter PSDs diferentes para uma
mesma pista variando-se a velocidade do ve\'{i}culo, ou seja, a quantidade
de energia que ser\'{a} imposta ao ve\'{i}culo ser\'{a} fun\c{c}\~{a}o da
velocidade que ele trafega pela pista. Para obter PSDs que sejam fun\c{c}%
\~{a}o do tempo e n\~{a}o do espa\c{c}o, deve-se multiplicar a velocidade
com que trafega o ve\'{i}culo pelo n\'{u}mero de onda\cite{Gillespie}, como
mostra a seguinte equa\c{c}\~{a}o\cite{Genta}.

\begin{equation}
S(\lambda )=\frac{S(\gamma )}{V}  \label{PSDhz}
\end{equation}

Sendo a frequ\^{e}ncia definida como,

\begin{equation}
\lambda =V.\gamma  \label{Freq_PSD}
\end{equation}

tem-se que a PSD temporal de uma pista, pode ser definifa pela equa\c{c}%
\~{a}o seguinte.

\begin{equation}
S(\lambda )=G_{o}.V^{n-1}.\lambda ^{-n}  \label{PSDhz01}
\end{equation}

$S(\lambda )$= PSD do deslocamento [$m^{2}/Hz$]$;$

$V$ = Velocidade horizontal do ve\'{i}clulo [$m/s$]$;$

$\lambda $ = Frequ\^{e}ncia [$Hz$]$;$

\emph{Para ondula\c{c}\~{o}es com comprimentos de onda maiores que 6m, ou
seja, }$n=2$\emph{\ a PSD do deslocamento \'{e} proporcional a }$\lambda
^{-2}$\emph{\ e a PSD da acelera\c{c}\~{a}o \'{e} constante, dessa forma a
PSD da acelera\c{c}\~{a}o vertical pode ser representada por um ru\'{i}do
branco.} (Deve ou n\~{a}o ser comentado?)

Os gr\'{a}ficos a seguir mostram PSDs para o deslocamento, a velocidade e a
acelera\c{c}\~{a}o verticais de uma pista classificada como m\'{e}dia, ver
tabela\ref{TabelaGo} , para um autom\'{o}vel trafegando a $80$ e $100km/h$%
.\bigskip

\FRAME{fhF}{4.8274in}{2.1309in}{0pt}{}{}{Grafpsdhz}{\special{language
"Scientific Word";type "GRAPHIC";maintain-aspect-ratio TRUE;display
"USEDEF";valid_file "F";width 4.8274in;height 2.1309in;depth
0pt;original-width 9.0935in;original-height 3.9998in;cropleft "0";croptop
"1";cropright "1";cropbottom "0";filename 'PSD02hz.jpg';file-properties
"XNPEU";}}

Pode-se observar que o espectro de acelera\c{c}\~{a}o possui um aumento com
o aumento da frequ\^{e}ncia. Esse comportamento mostra que a altas
frequ\^{e}ncias as irregularidades da pista ser\~{a}o sentidas
predominantemente na forma de acelera\c{c}\~{o}es.

A influ\^{e}ncia da velocidade pode ser vista nos graficos da figura acima.
Nota-se que aumentando a velocidade causa o deslocamento das curvas no
sentido do aumento das amplitudes do deslocamento, velocidade e acelera\c{c}%
\~{a}o.

Sendo as irregularidades das pistas apresentadas de modo aleat\'{o}rio
pode-se prever que haja tamb\'{e}m diferen\c{c}as em termos de eleva\c{c}%
\~{a}o entre os caminhos percorridos pelos pneus de um lado e de outro de um
ve\'{i}culo.Essa diferen\c{c}a \'{e} respons\'{a}vel por excitar o modo de
''roll'' do ve\'{i}culo que comumente \'{e} desprezado nos estudos de
conforto vibracional. Segundo Gillespie\cite{Gillespie}, em grande parte dos
ve\'{i}culos a resson\^{a}ncia em ''roll'' ocorre a uma frequ\^{e}ncia mais
baixa que a resson\^{a}ncia em ''bounce'', por\'{e}m, para baixas
frequ\^{e}ncias a magnitude da exita\c{c}\~{a}o do modo de ''roll'' e
inferior \`{a} exita\c{c}\~{a}o dos modos de ''bounce'' e ''pitch''. J\'{a}
nas frequ\^{e}ncias mais altas quando as magnitudes das excita\c{c}\~{o}es
dos modos de ''roll'' e ''bounce'' s\~{a}o da mesma ordem de grandeza os
ve\'{i}culos n\~{a}o possuem boa resposta para o modo de ''roll''. \emph{(ou
n\~{a}o s\~{a}o t\~{a}o responsivos ao modo de roll)}

A densidade espectral \'{e} a distribui\c{c}\~{a}o da varian\c{c}a do perfil
em fun\c{c}\~{a}o do n\'{u}mero de onda (m\'{e}dia quadr\'{a}tica da eleva\c{%
c}\~{a}o=varian\c{c}a quando a m\'{e}dia da eleva\c{c}\~{a}o \'{e} igual a
zero).

\section{A Resposta Din\^{a}mica de um Autom\'{o}vel}

O comportamento din\^{a}mico de um ve\'{i}culo pode ser caracterizado
principalmente pelas rela\c{c}\~{o}es entre as excita\c{c}\~{o}es a ele
impostas e a forma como ele reage a estas exicta\c{c}\~{o}es, ou seja, a rela%
\c{c}\~{a}o entre as entradas e as sa\'{i}das. As entradas podem ser
qualquer tipo de excita\c{c}\~{a}o j\'{a} apresentada anteriormente ou uma
combiana\c{c}\~{a}o delas e as respostas normalmente s\~{a}o obtidas em
forma da acelera\c{c}\~{a}o da carroceria.

Um ve\'{i}culo n\~{a}o responde da mesma forma a todas as faixas de
frequ\^{e}ncia das excita\c{c}\~{o}es a que ele est\'{a} sujeito, ao
contr\'{a}rio, ele amplifica ou atenua as excita\c{c}\~{o}es provenientes
das fontes de excita\c{c}\~{a}o de forma diferente para cada faixa de
frequ\^{e}ncia.

A frequ\^{e}ncias muito baixas, pode-se afirmar que n\~{a}o se observa
nenhuma resposta do ve\'{i}culo, pois a carroceria do autom\'{o}vel move-se
solid\'{a}ria com os eixos do ve\'{i}culo. Na faixa de frequ\^{e}ncia de 1 a
2 Hz, ocorre a resson\^{a}ncia da carroceria em rela\c{c}\~{a}o \`{a}
suspens\~{a}o.Nessa faixa de frequ\^{e}ncia as irregularidades da pista
correspondentes a essa frequ\^{e}ncia s\~{a}o amplificadas pelo modo de
''bounce'' do ve\'{i}culo. A amplifica\c{c}\~{a}o nessas frequ\^{e}ncias
gira em torno de 1.5 a 3 vezes dependendo do amortecimento do sistema de
suspens\~{a}o do ve\'{i}culo. A segunda ress\^{o}nancia ocorre entre 8 e 12
Hz e corresponde a resson\^{a}ncia da massa n\~{a}o suspensa, sistema
pneu/roda/montante.

Acima dessa faixa de frequ\^{e}ncia, a amplitude das irregularidades da
pista \'{e} atenuada pela incapcidade da suspens\~{a}o do ve\'{i}culo em
seguir as irregularidades e nessa faixa as irregularidades s\~{a}o quase
totalmente absorvidas pelas deflex\~{o}es dos pneus\cite{Gillespie and
Sayers}.

\subsection{Comportamento Din\^{a}mico de um Autom\'{o}vel}

O principal objetivo de se ter um modelo mat\'{e}matico que represente,
mesmo que aproximadamente, o comportamento din\^{a}mico do ve\'{i}culo \'{e}
o de poder prever o comportamento deste sem a necessidade de se construir um
prot\'{o}tipo. Os modelos mat\'{e}maticos auxiliam aos projetitas desde o
in\'{i}cio do desenvolvimento de um autom\'{o}vel at\'{e} nos \'{u}ltimos
acertos ap\'{o}s at\'{e} da constru\c{c}\~{a}o de um prot\'{o}tipo.

Segundo Jolly\cite{Jolly}, existem duas classes de par\^{a}metros que
definem um modelo mat\'{e}matico para o estudo do conforto de um
autom\'{o}vel, s\~{a}o elas:

-par\^{a}metros estruturais, que s\~{a}o as massas suspensas e
n\~{a}o-suspensas, a rigidez dos pneus, a rigidez das suspens\~{o}es e os
seus coeficientes de amortecimento,que agem linearmente;e

-par\^{a}metros de ajustes finais, o atrito das suspens\~{o}es, as n\~{a}o
linearidades dos pneus e os batentes e buchas el\'{a}sticas onde s\~{a}o
montados os amortecedores e bra\c{c}os da suspens\~{a}o, que agem de forma
n\~{a}o-linear.

Essas caracter\'{i}sticas corroboram para o fato dos porjetistas utilizarem
no in\'{i}cio dos projetos modelos mais simples, pois deve-se determinar
neste momento as caracter\'{i}sticas macro do autom\'{o}vel, e modelos mais
complexos e refinados para os acertos finais do projeto.

Os modelos simples de um quarto de ve\'{i}culo e meio ve\'{i}culo s\~{a}o,
normalmente, utilizados no in\'{i}cio dos trabalhos para a determina\c{c}%
\~{a}o das caracteristicas principais do ve\'{i}culo. Esses dois modelos
permitem estudos da din\^{a}mica vertical, modo de ''bounce'' no modelo de
1/4 de ve\'{i}culo e os modos de ''bounce'' e ''pitch'' no modelo de meio
ve\'{i}culo.

Os modelos modelos completos com 7 graus de liberdade ou mais,
n\~{a}o-lineares e modelos utilizando t\'{e}cnicas de m\'{u}ltiplos corpos
podem ser utilizados para os ajustes finais do projeto, verificar a
influ\^{e}ncia de algum componente no comportamento din\^{a}mico do
autom\'{o}vel, ou mesmo para o estudo do modos de ''roll'' e ''yaw'' da
carroceria.

\subsection{ Modelo de 1/4 de ve\'{i}culo}

O modelo de 1/4 de ve\'{i}culo apesar de ser um modelo bastante simples
\'{e} de extrema importancia para compreender alguns fen\^{o}menos da
resposta do autom\'{o}vel as fontes de excita\c{c}\~{a}o a que ele est\'{a}
sujeito.

Carbon\footnote{%
Bourcier De Carbon Ch.: A theory and an effective design of the damped
suspension of ground vehicles, III FISITA Congress,1950.}.

Dois dos mais comuns modelos de 1/4 de autom\'{o}vel est\~{a}o apresentados
na figura abaixo.

\FRAME{fhFU}{3.0312in}{1.919in}{0pt}{\Qcb{Modelos de 1/4 de ve\'{i}culo com
1 e 2 graus de liberdade.}}{\Qlb{14veiculo}}{carbon.jpg}{\special{language
"Scientific Word";type "GRAPHIC";maintain-aspect-ratio TRUE;display
"USEDEF";valid_file "F";width 3.0312in;height 1.919in;depth
0pt;original-width 6.333in;original-height 3.9998in;cropleft "0";croptop
"1";cropright "1";cropbottom "0";filename 'Carbon.jpg';file-properties
"XNPEU";}}

O primeiro modelo \'{e} caracterizado por um sistema massa-mola-amortecedor,
portanto com um \'{u}nico grau de liberdade, com excita\c{c}\~{a}o feita
pela base, onde o pneu \'{e} considerado como corpo r\'{i}gido, a massa
representa a carroceria (massa suspensa) e a mola e o amortecedor
representam a rigidez e o amortecimeto do sistema de suspens\~{a}o. Esse
tipo de modelo pode ser utilizado quando as frequ\^{e}ncias de interesse
est\~{a}o pr\'{o}ximas a faixa de frequ\^{e}ncia de resson\^{a}ncia da massa
suspensa.

O\ segundo modelo tem dois graus de liberdade e considera,, al\'{e}m do
primeiro, a rigidez e o amortecimento do pneu e a massa n\~{a}o-suspensa.
Para frequ\^{e}ncias na faixa de 0Hz a 50Hz esse modelo \'{e} utilizado e
resultados satisfat\'{o}rios s\~{a}o obtidos.

Segundo Genta\cite{Genta}, um terceiro modelo com tr\^{e}s graus de
liberdade permite estudos em faixas de frequ\^{e}ncia que excedam a primeira
frequ\^{e}ncia natural dos pneus.

O modelo de 1/4 de ve\'{i}culo com dois graus de liberdade permite o estudo
da resposta do autom\'{o}vel de forma relativamente f\'{a}cil, com esse
modelo realizam-se v\'{a}rios estudos de conforto vibracional de
ve\'{i}culos.

Aplicando a 2%
%TCIMACRO{\UNICODE{0xaa} }%
%BeginExpansion
${{}^a}$%
%EndExpansion
lei de Newton para o sistema din\^{a}mico representado pelo modelo de dois
graus de liberdade da figura anterior, obtem-se o seguinte sistema de equa\c{%
c}\~{o}es diferenciais:

\begin{eqnarray}
Ms.\frac{dZ^{2}}{d^{2}t}+Cs.(\frac{dZ}{dt}-\frac{dZu}{dt})+Ks.(Z-Zu) &=&Fm
\label{14veiculoequa01} \\
mu.\frac{dZu^{2}}{d^{2}t}+Cs.(\frac{dZu}{dt}-\frac{dZ}{dt})+Ks.(Zu-Z)+Ct.(%
\frac{dZu}{dt}-\frac{dh}{dt})+Kp.(Zu-h) &=&Ft  \nonumber
\end{eqnarray}

Reescrevendo a segunda equa\c{c}\~{a}o tem-se:

\begin{eqnarray}
Ms.\frac{dZ^{2}}{d^{2}t}+Cs.(\frac{dZ}{dt}-\frac{dZu}{dt})+Ks.(Z-Zu) &=&Fm
\label{14veiculoequa02} \\
mu.\frac{dZu^{2}}{d^{2}t}+Cs.(\frac{dZu}{dt}-\frac{dZ}{dt})+Ks.(Zu-Z)+Ct.%
\frac{dZu}{dt}+Kp.Zu &=&
\end{eqnarray}

$\qquad \qquad \qquad \qquad \qquad \qquad \qquad \qquad \qquad \qquad
\qquad 
\begin{array}{c}
Ft+Ct.\frac{dh}{dt}+Kp.h
\end{array}
$

O sistema de equa\c{c}\~{o}es acima pode ser apresentado na forma matricial:

\bigskip 
\begin{equation}
\begin{array}{cc}
\begin{array}{cc}
Ms & 0 \\ 
0 & mu
\end{array}
.
\begin{array}{c}
\frac{dZ^{2}}{d^{2}t} \\ 
\frac{dZu^{2}}{d^{2}t}
\end{array}
+
\begin{array}{cc}
Cs & -Cs \\ 
-Cs & Cs+Ct
\end{array}
.
\begin{array}{c}
\frac{dZ}{dt} \\ 
\frac{dZu}{dt}
\end{array}
+
\begin{array}{cc}
Ks & -Ks \\ 
-Ks & Ks+Kp
\end{array}
.
\begin{array}{c}
Z \\ 
Zu
\end{array}
= & Fm \\ 
Ft+Ct.\frac{dh}{dt}+Kp.h & 
\end{array}
\end{equation}

\bigskip Onde:

$Ms$ \'{e} a massa suspensa;

$mu$ \'{e} a massa n\~{a}o suspensa;

$Cs$ \'{e} o coeficiente de amorteciemento do sistema de suspens\~{a}o;

$Ct$ \'{e} o coeficiente de amortecimento do pneu;

$Ks$ \'{e} a rigidez do sistema de suspens\~{a}o;

$Kp$ \'{e} a rigidez do pneu;

$Fm$ \'{e} uma for\c{c}a que atua na massa suspensa;

$Ft$ \'{e} uma for\c{c}a que atua na massa n\~{a}o suspensa;

$Z$ \'{e} o deslocamento da massa suspensa;

$Zu$ \'{e} o deslocamento da massa n\~{a}o suspensa;

$h$ \'{e} a eleva\c{c}\~{a}o do perfil da pista.

Considerando que o amortecimeto do pneum\'{a}tico \'{e} comumente desprezado
na obten\c{c}\~{a}o de modelos para o estudo de conforto\cite{Genta} \ e
visando obter a resposta das massas n\~{a}o suspensas e suspensas devido
unicamente as irregularidades da pista,ou seja considerando $Fm=Ft=0$, o
sistema de equa\c{c}\~{o}es diferenciais para o modelo acima pode ser
reescrito da seguinte forma:

\begin{eqnarray}
Ms.\frac{dZ^{2}}{d^{2}t}+Cs.(\frac{dZ}{dt}-\frac{dZu}{dt})+Ks.(Z-Zu) &=&0
\label{14veiculoequa03} \\
mu.\frac{dZu^{2}}{d^{2}t}+Cs.(\frac{dZu}{dt}-\frac{dZ}{dt})+Ks.(Zu-Z)+Kp.Zu
&=&Kp.h  \nonumber
\end{eqnarray}

A solu\c{c}\~{a}o para o sistema de equa\c{c}\~{o}es diferenciais acima
\'{e} amplamente difundido no estudo de sistemas din\^{a}micos e vibra\c{c}%
\~{o}es, podem ser citados os seguintes autores\cite{Ewins}\cite{Inman} \cite
{Pipes}\cite{Thomson} como refer\^{e}ncia \`{a} solu\c{c}\~{a}o de sistemas
com v\'{a}rios graus de liberdade excitados harmonicamente pela base.

As fun\c{c}\~{o}es de resposta em frequ\^{e}ncia para o deslocamento da
massa suspensa e da massa n\~{a}o suspensa devido \`{a} varia\c{c}\~{a}o da
eleva\c{c}\~{a}o do perfil da pista s\~{a}o apresentadas a seguir,com o
intuito de simplificar a apresenta\c{c}\~{a}o das fun\c{c}\~{o}es de
resposta em frequ\^{e}ncia, far-se-a a seguinte simplifica\c{c}\~{a}o:

\begin{eqnarray}
\frac{Z}{h} &=&\frac{Ks.Kp+i.\varpi .Cs.Kp}{(g(\varpi ).\varpi
.Cs).i+f(\varpi )}  \label{14veiculoequa04} \\
\frac{Zu}{h} &=&\frac{Ks.Kp-Ms.\varpi ^{2}.Kp+i.\varpi .Cs.Kp}{(g(\varpi
).\varpi .Cs).i+f(\varpi )}  \label{14veiculoequa05}
\end{eqnarray}

Onde:

\begin{eqnarray}
f(\varpi ) &=&Ms.mu.\varpi ^{4}-[(Kp+Ks).Ms+Ks.mu].\varpi ^{2}+Ks.Kp
\label{14veiculoequa06} \\
\bigskip g(\varpi ) &=&Kp-[\varpi ^{2}.(Ms+mu)]  \label{14veiculoequa07}
\end{eqnarray}

O ganho associado ao sistema acima \'{e} representado pela FRF, ou seja pela
raz\~{a}o entre a amplitude do deslocamento e amplitude da excita\c{c}%
\~{a}o, seja esta uma for\c{c}a, uma acelera\c{c}\~{a}o ou at\'{e} um
deslocamento.

\subsection{Resposta da massa suspensa e n\~{a}o suspensa.}

A equa\c{c}\~{a}o \ref{14veiculoequa04} representa a FRF da massa suspensa
devido a excita\c{c}\~{a}o decorrente da varia\c{c}\~{a}o de eleva\c{c}%
\~{a}o do perfil da pista e a equa\c{c}\~{a}o \ref{14veiculoequa05}
representa a FRF da massa n\~{a}o suspensa devido a mesma excita\c{c}\~{a}o.
A figura a seguir mostra o ganho da carroceria (massa suspensa) e da massa
n\~{a}o suspensa para um ve\'{i}culo com raz\~{a}o de amortecimento $\zeta
=0.24$.\ \ 

\FRAME{fhFU}{3.7144in}{3.3797in}{0pt}{\Qcb{Compara\c{c}\~{a}o entre FRF da
carroceria e da massa n\~{a}o suspensa devido as irregularidades da pista.}}{%
\Qlb{receptance00}}{2dofgraf01.jpg}{\special{language "Scientific Word";type
"GRAPHIC";maintain-aspect-ratio TRUE;display "USEDEF";valid_file "F";width
3.7144in;height 3.3797in;depth 0pt;original-width 3.6668in;original-height
3.333in;cropleft "0";croptop "1";cropright "1";cropbottom "0";filename
'2dofgraf01.jpg';file-properties "XNPEU";}}

\bigskip Observa-se que a baixas frequ\^{e}ncias o ganho \'{e} unit\'{a}rio
o que mostra que as massas suspensas e n\~{a}o suspensas tem amplitude de
deslocamento exatamente igual \`{a} da pista, o autom\'{o}vel ''copia''
exatamente o perfil da pista.

A frequ\^{e}ncias pr\'{o}ximas de $1Hz$ a massa suspensa entra em
resson\^{a}ncia, o que corresponde ao primeiro modo de vibrar do sistema de
2 graus de liberdade, em torno dessa frequ\^{e}ncia as irregularidades da
pista s\~{a}o amplificadas entre 1.5 e 3 vezes para autom\'{o}veis de
passeio.\cite{Gillespie and Sayers} De acordo com Gillespie\cite{Gillespie}
nos projetos classicos de autom\'{o}veis escolhe-se a resson\^{a}ncia da
carroceira pr\'{o}xima a $1Hz$, o que se reflete claramente no gr\'{a}fico 
\ref{receptance00}.

O segundo pico na curva de resposta da massa suspensa \'{e} devido a
resson\^{a}ncia da massa n\~{a}o suspensa que ocorre na faixa de
frequ\^{e}ncia entre $10$ e$12Hz$, o que corresponde ao segundo modo de
vibrar do sistema. Pode-se observar no gr\'{a}fico \ref{receptance00} \ que
a ocorr\^{e}ncia do segundo pico na curva de resposta da massa suspensa
corresponde ao maior ganho da curva da massa n\~{a}o suspensa.O mesmo ocorre
para a curva da massa n\~{a}o suspensa pr\~{o}ximo a frequ\^{e}ncia de $1Hz$.

O gr\'{a}fico \ref{Accelerance01} a seguir apresenta a FRF da acelera\c{c}%
\~{a}o da carroceria que \'{e} obtida multiplicando-se a FRF do
deslocamento, equa\c{c}\~{a}o\ref{14veiculoequa04}, por $\varpi ^{2}$,
j\'{a} que a resposta e a exita\c{c}\~{a}o do sistema s\~{a}o considerados
harmonicos para a solu\c{c}\~{a}o do sistema. Desta forma a resposta em
frequ\^{e}ncia da acelera\c{c}\~{a}o da carroceria, caracterizada pela
raz\~{a}o entre a acelera\c{c}\~{a}o da carroceria e a acelera\c{c}\~{a}o
imposta nos pneus devido a varia\c{c}\~{a}o da eleva\c{c}\~{a}o do perfil da
pista, pode ser escrita da seguinte forma:

\begin{equation}
\frac{\frac{dZ^{2}}{d^{2}t}}{\frac{dh^{2}}{d^{2}t}}=\frac{\varpi
^{2}Ks.Kp+i.\varpi ^{3}.Cs.Kp}{(g(\varpi ).\varpi .Cs).i+f(\varpi )}
\label{14veiculoequa15}
\end{equation}

\FRAME{fhFU}{4.0491in}{3.3797in}{0pt}{\Qcb{FRF da acelera\c{c}\~{a}o da
massa suspensa.}}{\Qlb{Accelerance01}}{2dofgraf02.jpg}{\special{language
"Scientific Word";type "GRAPHIC";maintain-aspect-ratio TRUE;display
"USEDEF";valid_file "F";width 4.0491in;height 3.3797in;depth
0pt;original-width 3.9998in;original-height 3.333in;cropleft "0";croptop
"1";cropright "1";cropbottom "0";filename '2dofgraf02.jpg';file-properties
"XNPEU";}}

A resposta da carroceira a excita\c{c}\~{o}es provenientes do grupo
moto-propulsor e de irregularidades e desbalanceamento do conjunto pneu/roda
tamb\'{e}m s\~{a}o de interesse para o estudo do conforto de ve\'{i}culos.
Para se obter a resposta da carroceria a excita\c{c}\~{o}es na pr\'{o}pria
carroceria, por exemplo vibra\c{c}\~{o}es provenientes do sistema
moto-propulsor, resolve-se novamente o sistema de equa\c{c}\~{o}es
diferenciais \ref{14veiculoequa02}, considerando agora $Ft=h=0$. Comumente
utiliza-se uma fun\c{c}\~{a}o de ganho um pouco diferente da obtida na solu%
\c{c}\~{a}o do sistema de equa\c{c}\~{o}es \ref{14veiculoequa02}, para se\
obter uma fun\c{c}\~{a}o de ganho adimensional multiplica-se a FRF da acelera%
\c{c}\~{a}o da massa suspensa pela pr\'{o}pria massa suspensa, o que resulta
em uma raz\~{a}o de for\c{c}as.Desta forma a fun\c{c}\~{a}o de resposta
\'{e} equivalente a uma for\c{c}a na massa suspensa necess\'{a}ria para
produzir as acelere\c{c}\~{o}es a que ela est\'{a} sujeita.

\begin{equation}
\frac{\frac{dZ^{2}}{d^{2}t}.Ms}{Fm}=\frac{(Ks+Kp-mu.\varpi ^{2}).Ms.\varpi
^{2}+i.\varpi ^{3}.Cs.Ms}{(g(\varpi ).\varpi .Cs).i+f(\varpi )}
\label{14veiculoequa13}
\end{equation}

\bigskip Para a obten\c{c}\~{a}o da resposta devido ao desbalanceamento do
conjunto pneu/roda, deve-se proceder da mesma forma, solucionado o sistema
de equa\c{c}\~{o}es diferenciais \ref{14veiculoequa02} para $Fm=h=0$. De
maneira an\'{a}loga a resposta devido as exita\c{c}\~{o}es provenientes do
moto-propulsor procura-se a uma fun\c{c}\~{a}o ganho adimensional e portanto
multiplica-se a FRF da acelera\c{c}\~{a}o da massa suspesnsa pela
pr\'{o}pria massa suspensa. Obtem-se dessa forma a raz\~{a}o entre a
amplitude da for\c{c}a necessaria para produzir as acelere\c{c}\~{o}es a que
est\'{a} sujeita a massa suspensa e a amplitude da for\c{c}a devido ao
desbalanceamento do conjunto pneu/roda, e pode ser escrito da seguinte
forma: 
\begin{equation}
\frac{\frac{dZ^{2}}{d^{2}t}.Ms}{Ft}=\frac{Ks.Ms.\varpi ^{2}+i.\varpi
^{3}.Cs.Ms}{(g(\varpi ).\varpi .Cs).i+f(\varpi )}  \label{14veiculoequa14}
\end{equation}

O gr\'{a}fico \ref{Accelerance02} a seguir mostra a resposta da carroceria
as excita\c{c}\~{o}es diretamente na carroceria e a excita\c{c}\~{o}es que
atuam no conjunto roda/pneu. A primeira curva que representa a resposta da
carroceria as excita\c{c}\~{o}es na pr\'{o}pria carroceria possui ganho nulo
a frequ\^{e}ncias pr\'{o}ximas de zero. Pr\'{o}ximo de $1Hz$ \ a curva
apresenta um pico com ganho da ordem de $2.5$ e em seguida a massa suspensa
responde com ganho aproximadamente igual a 1 o que mostra que para
frequ\^{e}ncias acima de $1Hz$ o ve\'{i}culo possui uma alta sensibilidade a
excita\c{c}\~{o}es atuantes diretamente na carroceria, pois os deslocamentos
produzidos pelas for\c{c}as s\~{a}o t\~{a}o pequenos que o sistema de
suspens\~{a}o n\~{a}o \'{e} mais capaz de absorv\'{e}-los e a for\c{c}a
\'{e} quase toda dissipada em acelera\c{c}\~{o}es da massa suspensa\cite
{Gillespie}.

A segunda curva mostra a resposta da carroceria \`{a} excita\c{c}\~{o}es
provenientes da massa n\~{a}o suspensa. Para valores de frequ\^{e}ncia
pr\'{o}ximos de zero o ganho da carroceria \'{e} igual a zero, pois as for\c{%
c}as na suspens\~{a}o s\~{a}o absorvidas pela rigidez dos pneus. O\ ganho
aumenta com o aumento da frequ\^{e}ncia e pr\'{o}ximo a $1Hz$ aparece o
primeiro pico devido a resson\^{a}ncia da massa suspensa, o ganho continua a
aumentar at\'{e} chegar a unidade em torno da faixa $10$ a $12Hz$.

\FRAME{fhFU}{3.7144in}{3.3797in}{0pt}{\Qcb{FRF da acelera\c{c}\~{a}o da
carroceria devido a excita\c{c}\~{o}es internas.}}{\Qlb{Accelerance02}}{%
2dofgraf03.jpg}{\special{language "Scientific Word";type
"GRAPHIC";maintain-aspect-ratio TRUE;display "USEDEF";valid_file "F";width
3.7144in;height 3.3797in;depth 0pt;original-width 3.6668in;original-height
3.333in;cropleft "0";croptop "1";cropright "1";cropbottom "0";filename
'2dofgraf03.jpg';file-properties "XNPEU";}}

\subsection{Influencia do amortecimento na resposta da massa suspensa.}

O gr\'{a}fico \ref{receptance01}, a seguir, apresenta o ganho da carroceria,
em termos de deslocamento\footnote{%
As fun\c{c}\~{o}es de resposta em frequ\^{e}ncia podem ser apresentadas como
a raz\~{a}o de amplitudes de deslocamento ( Receptance), de velocidade
(Mobility) e acelera\c{c}\~{a}o (Accelerance).\cite{Ewins}}, do modelo de
1/4 de ve\'{i}culo e dois graus de liberdade \`{a}s excita\c{c}\~{o}es
devido \`{a}s irregularidades da pista variando o coeficiente de
amortecimento da suspens\~{a}o.

\FRAME{fhFU}{4.6389in}{4.0534in}{0pt}{\Qcb{FRF da carroceira variando o
coeficiente de amortecimento da suspens\~{a}o.}}{\Qlb{receptance01}}{%
2dofgraf.jpg}{\special{language "Scientific Word";type
"GRAPHIC";maintain-aspect-ratio TRUE;display "USEDEF";valid_file "F";width
4.6389in;height 4.0534in;depth 0pt;original-width 6.3123in;original-height
5.5106in;cropleft "0";croptop "1";cropright "1";cropbottom "0";filename
'2dofgraf.jpg';file-properties "XNPEU";}}

Observando o gr\'{a}fico \ref{receptance01} nota-se que para $Cs=0$, um
sistema de dois graus de liberdade n\~{a}o amortecido, a resposta da massa
suspensa apresenta dois picos muito amplificados, com o ganho tendendo a
infinito pr\'{o}ximo as frequ\^{e}ncias de resson\^{a}ncia da carroceria. O
primeiro pico corresponde ao primeiro modo de vibrar do sistema; esse modo
corresponde aproximadamente \`{a} resson\^{a}ncia da massa suspensa e sua
frequ\^{e}ncia natural \'{e} aproximadamente igual a:

\begin{equation}
f1\cong \frac{1}{2.\pi }.\sqrt{(\frac{Kp}{Kp+Ks}.\frac{Ks}{Ms})}
\label{14veiculoequa08}
\end{equation}

Considerando que normalmente $Kp$ \'{e} muito maior que $Ks$ tem-se que a
raz\~{a}o; 
\begin{equation}
(\frac{Kp}{Kp+Ks})\cong 1  \label{14veiculoequa09}
\end{equation}
o que faz com que a primeira frequ\^{e}ncia natural do sistema se aproxime
de:

\begin{equation}
f1\cong \frac{1}{2.\pi }.\sqrt{\frac{Ks}{Ms}}  \label{14veiculoequa10}
\end{equation}

\bigskip O segundo pico apresentado no gr\'{a}fico \ref{receptance01},
est\'{a} relacionado com o segundo modo de vibrar do sistema; esse modo
corresponde aproximadamente \`{a} resson\^{a}ncia da massa n\~{a}o suspensa
e sua frequ\^{e}ncia \'{e} dada por:

\begin{equation}
f3\cong \frac{1}{2.\pi }.\sqrt{\frac{Kp+Ks}{mu}}  \label{14veiculoequa11}
\end{equation}

Quando $Cs=\infty $, o sistema original se degenera, utilizando as palavras
de Genta\cite{Genta}, e torna-se um sistema de um grau de liberdade n\~{a}o
amortecido, composto de uma massa equivalente a $(Ms+mu)$, apoiado em uma
\'{u}nica mola de constante de rigidez $Kp$. Esse fen\^{o}meno faz com que a
resson\^{a}ncia, para esse caso, \`{a} frequ\^{e}ncia:

\begin{equation}
f2=\frac{1}{2.\pi }.\sqrt{\frac{Kp}{Ms+mu}}  \label{14veiculoequa12}
\end{equation}

\bigskip Observando o gr\'{a}fico\ref{receptance01} com cuidado notam-se 4
pontos indicados pelas letras A, B, C e D onde a resposta da massa suspensa
independe do amortecimento Cs. Por esses quatro pontos passam todas as
curvas para os v\'{a}rios amortecimentos, Jolly\cite{Jolly},Genta\cite{Genta}%
,Inman\cite{Inman} e Thomson\cite{Thomson}, apresentam com mais detalhes as
propriedades desse sistema e esses pontos not\'{a}veis.

Nota-se que o amortecimento somente ser\'{a} benefico nas regi\~{o}es
correspondentes a resson\^{a}ncia da carroceria e a da massa n\~{a}o
suspensa. Nota-se no gr\'{a}fico\ref{receptance01} que ao aumentar-se o
coeficiente de amortecimento o ganho somente diminuir\'{a} nas regi\~{o}es
entre os pontos A e B e C e D. Entre os pontos B e C o aumento do
amortecimento faz com que o ganho aumente, o mesmo acontece para a
regi\~{a}o \`{a} direita do ponto D.

Normalmente, segundo Jolly\cite{Jolly}, os coeficientes de amortecimento
cr\'{i}tico para a resson\^{a}ncia da carroceria e da massa n\~{a}o suspensa
s\~{a}o muito pr\'{o}ximos. Dessa forma, o coeficiente de amortecimento $Cs$%
, que \'{e} adotado para os ajustes visando conforto vibracional visando
reduzir a resposta vertical da carroceria na frequ\^{e}ncia de
resson\^{a}ncia tamb\'{e}m ser\'{a} eficiente no amortecimento na
resson\^{a}ncia vertical da massa n\~{a}o suspensa. Por outro lado, esse
amortecimento ter\'{a} um efeito adverso na faixa de frequ\^{e}ncia entre 3
e 10 Hz, que abrange uma grande faixa no interesse do estudo do conforto de
autom\'{o}veis e na faixa acima de 12 Hz.

\ \ Segundo Gillespie \cite{Gillespie} \ para a obten\c{c}\~{a}o de boas
caracter\'{i}sticas de conforto os autom\'{o}veis modernos possuem
raz\~{o}es de amortecimento entre 0.2 e 0.4. Considerando um modelo de 1/4
de ve\'{i}culo onde $Ms=221Kg$, $mu=35kg$, $Kp=1.5\times 10^{5}N/m$, $%
Ks=1.25\times 10^{4}N/m$ e $Cs=800N.s/m$ \ a sua raz\~{a}o de amortecimento
ser\'{a} igual a $\zeta =0.241$, j\'{a} que a raz\~{a}o de amortecimento
para um sistema de dois graus de liberdade amortecido pode ser calculada da
seguinte maneira \cite{Inman}\cite{Thomson}\cite{Genta}\cite{Gillespie}.

\begin{equation}
\zeta =\frac{Cs}{\sqrt{4.Ks.Ms}}  \label{razao amortecimento}
\end{equation}

\subsection{A influ\^{e}ncia da rigidez na resposta da massa suspensa}

A predomin\^{a}ncia da rigidez da suspens\~{a}o no isolamento das vibra\c{c}%
\~{o}es pode ser observada nas equa\c{c}\~{o}es \ref{14veiculoequa08}\ref
{14veiculoequa09}\ref{14veiculoequa10}, j\'{a} que normalmente a rigidez do
penum\'{a}tico \'{e} muito maior que a rigidez das molas da suspens\~{a}o, e
isso faz com a rigidez da suspens\~{a}o seja a principal determinante da
frequ\^{e}ncia do modo de ''bounce'' do autom\'{o}vel. Para se obter um bom
isolamento das vibra\c{c}\~{o}es \'{e} sabido que quanto menor a rigidez do
sistema de suspens\~{a}o tanto melhor ser\'{a} o isolamento. Por\'{e}m a
rigidez das molas estar\'{a} limitada devido \`{a} deflex\~{a}o est\'{a}tica
e as caracteristicas de dirigibilidade ''Handling''. A figura a seguir
mostra claramente a influ\^{e}ncia da rigidez na resposta da carroceria de
um autom\'{o}vel, a resposta foi obtida utilizando um modelo de 1/4 de
ve\'{i}culo semelhante ao da figura \ref{14veiculo}.

\FRAME{fhFU}{3.7542in}{3.1047in}{0pt}{\Qcb{Influ\^{e}ncia da rigidez da
suspens\~{a}o na resposta da carroceira as acelera\c{c}\~{o}es devido as
irregularidades da pista.}}{\Qlb{Accelerance03}}{2dofgraf04.jpg}{\special%
{language "Scientific Word";type "GRAPHIC";maintain-aspect-ratio
TRUE;display "USEDEF";valid_file "F";width 3.7542in;height 3.1047in;depth
0pt;original-width 5.3333in;original-height 4.4062in;cropleft "0";croptop
"1";cropright "1";cropbottom "0";filename '2dofgraf04.jpg';file-properties
"XNPEU";}}

A figura \ref{Accelerance02} mostra a resposta em acelera\c{c}\~{a}o da
massa n\~{a}o suspensa de um modelo de 1/4 de ve\'{i}culo onde a amplitude
da excita\c{c}\~{a}o varia com o inverso da frequ\^{e}ncia. Considerando a
frequ\^{e}ncia de $1.2Hz$ a amplitude da acelera\c{c}\~{a}o da excita\c{c}%
\~{a}o \'{e} de aproximadamente $0.83m/s^{2}$, para $Ks=12500N/m$ a acelera%
\c{c}\~{a}o da carroceria ser\'{a} de aproximadamente $18.8m/s^{2}$ enquanto
para $Ks=16000N/m$ a acelera\c{c}\~{a}o da carroceria ser\'{a} de
aproximadamente $22m/s^{2}$. Nota-se que a frequ\^{e}ncia de resson\^{a}ncia
aumenta a medida que o valor de $Ks$ aumenta e a amplifica\c{c}\~{a}o das
acelera\c{c}\~{o}es alcan\c{c}a o valor de aproximadamente $25m/s^{2}$para $%
Ks=16000N/m$.

\subsection{A influ\^{e}ncia da resson\^{a}ncia da massa n\~{a}o suspensa na
resposta da massa suspensa.}

A massa n\~{a}o suspensa \'{e} ajustada nos projetos de autom\'{o}veis de
maneira que a sua frequ\^{e}ncia de resson\^{a}ncia esteja na faixa entre $%
10 $ e $12Hz$. Cada conjunto de roda/pneu/montante possui um modo vertical
de vibrar que \'{e} excitado pelas irregularidades da pista e pelas n\~{a}o
uniformidades do conjunto, essas vibra\c{c}\~{o}es s\~{a}o transmitidas para
a carroceria. O segundo pico presente na figura \ref{Accelerance03} mostra
influ\^{e}ncia da resson\^{a}ncia da massa n\~{a}o suspensa nas acelera\c{c}%
\~{o}es da massa suspensa devido as irregularidades da pista, enquanto a
curva pontilhada da figura \ref{Accelerance02} mostra a resposta da
carroceria devido a for\c{c}as atuando na massa n\~{a}o suspensa. Nota-se,
na figura \ref{Accelerance02}, que o ganho na faixa entre 10 e 12 Hz e muito
amplificado mostrando a sensibilidade da carroceria \`{a} massa n\~{a}o
suspensa nessa faixa de frequ\^{e}ncia.

A figura \ref{Accelerance04} mostra a acelera\c{c}\~{a}o da carroceria \
devido as irregularidades da pista variando-se a massa n\~{a}o suspensa e
considerando que a amplitude da acelera\c{c}\~{a}o imposta pela pista varia
com o inverso da frequ\^{e}ncia da excita\c{c}\~{a}o. Ao aumentar-se a massa
n\~{a}o suspensa a frequ\^{e}ncia de resson\^{a}ncia da mesma diminui
aumentando a transmissibilidade de vibra\c{c}\~{o}es na faixa entre 2Hz e 12
Hz.

Segundo Gillespie \cite{Gillespie}, como \'{e} mais f\'{a}cil isolar vibra\c{%
c}\~{o}es de alta frequ\^{e}ncia em qualquer parte do chassis, a menor massa
n\~{a}o suspensa produzir\'{a}, geralmente, melhor comportamento do
autom\'{o}vel em comforto ( ''Ride'').

\FRAME{fhFU}{3.5855in}{3.3477in}{0pt}{\Qcb{FRF da acelera\c{c}\~{a}o da
massa suspensa variando o valor da massa n\~{a}o suspensa.}}{\Qlb{%
Accelerance04}}{2dofgraf06.jpg}{\special{language "Scientific Word";type
"GRAPHIC";maintain-aspect-ratio TRUE;display "USEDEF";valid_file "F";width
3.5855in;height 3.3477in;depth 0pt;original-width 5.0004in;original-height
4.6665in;cropleft "0";croptop "1";cropright "1";cropbottom "0";filename
'2dofgraf06.jpg';file-properties "XNPEU";}}

\subsection{Os modos de ''Bounce'' e ''Pitch''}

\bigskip Um autom\'{o}vel trafegando por uma pista pode ser considerado um
sistema massa mola com m\'{u}ltiplas entradas de excita\c{c}\~{a}o que
responde de com movimentos de transla\c{c}\~{a}o e rota\c{c}\~{a}o da
carroceria. As excita\c{c}\~{o}es devido as irregularidades da pista excitam
v\'{a}rios modos de vibrar do ve\'{i}culo, sendo os principais modos de vibra%
\c{c}\~{a}o da carroceria, se considerada como um corpo r\'{i}gido, os
seguintes:

-A transla\c{c}\~{a}o ao longo do eixo Z (''Bounce'');

-A Rota\c{c}\~{a}o em rela\c{c}\~{a}o ao eixo Z (''Yaw'');

-A Rota\c{c}\~{a}o em rela\c{c}\~{a}o ao eixo X (''Roll'');

-A Rota\c{c}\~{a}o em rela\c{c}\~{a}o ao eixo Y (''Pitch'').

No estudo do conforto vibracional os modos de Bounce e Pitch s\~{a}o os mais
importantes. Segundo Gillespie \cite{Gillespie} os movimentos de ''pitch''
s\~{a}o importantes pois geralmente s\~{a}o considerados desagrad\'{a}veis e
s\~{a}o as fontes b\'{a}sicas de vibra\c{c}\~{o}es longitudinais em pontos
acima do CG da carroceria.O conhecimento dos movimentos de ''pitch'' e
''bounce'' \'{e} essencial porque a sua combina\c{c}\~{a}o \'{e} que
determina as vibra\c{c}\~{o}es verticais e longitudinais em qualquer ponto
do ve\'{i}culo.

Os modos de ''pitch'' e ''bounce'' s\~{a}o dificilemente percebidos
separadamente, normalmente os modos de vibrar da carroceria s\~{a}o ou
predominantemente ''pitch'' ou predominantemente ''bounce''. A excita\c{c}%
\~{a}o de um modo mais que o outro depende b\'{a}sicamente da velocidade em
que o autom\'{o}vel trafega, a dist\^{a}ncia entre eixos e o comprimento de
onda das irregularidades da pista.

Considerando, para efeito de visualiza\c{c}\~{a}o, que os modos de
''bounce'' e ''pitch'' s\~{a}o desacoplados, tem-se que para um comprimento
de onda das irregularidades da pista da ordem da dist\^{a}ncia entre eixos
somente o modo de ''bounce'' ser\'{a} excitado. O mesmo ocorrer\'{a} para
comprimentos de onda da muito maiores que a dist\^{a}ncia entre eixos e para
comprimentos de ondas menores que tiverem um m\'{u}ltiplo inteiro igual a
dist\^{a}ncia entre eixos. Em uma forma similar, somente o modo de ''pitch''
ser\'{a} excitado se o comprimento de onda for igual a duas vezes a
dist\^{a}ncia entre eixos do ve\'{i}culo ou for menor e tiver um
m\'{u}ltiplo inteiro impar igual a duas vezes a dist\^{a}ncia entre eixos. 
\emph{A figura} abaixo ilustra a chamada ''wheelbase filtering''\cite
{Gillespie}\cite{Genta}.

Para o estudo dos modos e frequ\^{e}ncias de ''pitch'' e ''bounce'' \'{e}
necess\'{a}rio utilizar modelos de 1/2 de ve\'{i}culo como o ilustrado na
figura \ref{12veiculo_figura} a seguir, ou modelos completos.

\FRAME{fhFU}{5.0419in}{2.4197in}{0pt}{\Qcb{Modelo de 1/2 ve\`{i}culo e
quatro graus de liberdade.}}{\Qlb{12veiculo_figura}}{12veiculo.jpg}{\special%
{language "Scientific Word";type "GRAPHIC";maintain-aspect-ratio
TRUE;display "USEDEF";valid_file "F";width 5.0419in;height 2.4197in;depth
0pt;original-width 7.6666in;original-height 3.6668in;cropleft "0";croptop
"1";cropright "1";cropbottom "0";filename '12veiculo.jpg';file-properties
"XNPEU";}}

\bigskip Este ponto parece oportuno definir uma grandeza conhecida como
\'{I}ndice din\^{a}mico da massa suspensa $Id$, como:

\begin{equation}
Id=\frac{r_{y}^{2}}{l_{1}.l_{2}}  \label{Dynamic Index}
\end{equation}

\bigskip onde, tem-se:

$r_{y}$ = Raio de gira\c{c}\~{a}o da massa suspensa em rela\c{c}\~{a}o ao
eixo $y$, que pode ser definido como:$\ \ r_{y}=\sqrt{\frac{Jy}{Ms}}$ ,\ \ [$%
m$];

$Jy$ = Momento de inercia de massa da massa suspensa.[$kg.m2$];

$l_{1}$ = Dist\^{a}ncia entre o eixo dianteiro e o CG. [$m$];

$l_{2}$ = Dist\^{a}ncia entre o CG e o eixo dianteiro. [$m$];

$L$ = Dist\^{a}ncia entre eixos. [$m$].

\bigskip Se o \'{I}ndice din\^{a}mico for igual a 1, i.e. se $%
Jy=Ms.l_{1}.l_{2}$ , as suspens\~{o}es dianteria e traseira estar\~{a}o
localizadas em centros de repercuss\~{a}o conjugados da massa suspensa e o
modelo da figura \ref{12veiculo_figura} se reduzir\'{a} a dois modelos de
1/4 de ve\'{i}culo como os da figura \ref{14veiculo}. Os movimentos das
suspens\~{o}es dianteira e traseira se tornam desacoplados e dessa forma
pode-se utilizar dois modelos de 1/4 de ve\'{i}culo de dois graus de
liberdade para o estudo do conforto. Com uma correta manipula\c{c}\~{a}o do
sistema de equa\c{c}\~{o}es \ref{12veiculo} a influ\^{e}ncia do \'{I}ndice
din\^{a}mico pode ser demonstrada \cite{Genta}.

Aplicando a 2%
%TCIMACRO{\UNICODE{0xaa} }%
%BeginExpansion
${{}^a}$%
%EndExpansion
lei de Newton para o sistema din\^{a}mico representado pelo modelo da figura 
\ref{12veiculo_figura} de quatro graus de liberdade, obtem-se o seguinte
sistema de equa\c{c}\~{o}es diferenciais. 
\[
\left[ 
\begin{array}{cccc}
Ms & 0 & 0 & 0 \\ 
0 & Jy & 0 & 0 \\ 
0 & 0 & mu_{1} & 0 \\ 
0 & 0 & 0 & mu_{2}
\end{array}
\right] \left( 
\begin{array}{c}
\frac{dZ^{2}}{d^{2}t} \\ 
\frac{d\theta ^{2}}{d^{2}t} \\ 
\frac{dZ_{u1}^{2}}{d^{2}t} \\ 
\frac{dZ_{u2}^{2}}{d^{2}t}
\end{array}
\right) +
\]

\[
+\left[ 
\begin{array}{cccc}
Cs_{2}+Cs_{1} & Cs_{2}.l_{2}-Cs_{1}.l_{1} & -Cs_{1} & -Cs_{2} \\ 
Cs_{2}.l_{2}-Cs_{1}.l_{1} & Cs_{2}.l_{2}^{2}+Cs_{1}.l_{1}^{2} & Cs_{1}.l_{1}
& -Cs_{2}.l_{2} \\ 
-Cs_{1} & Cs_{1}.l_{1} & Ct_{1}+Cs_{1} & 0 \\ 
-Cs_{2} & -Cs_{2}.l_{2} & 0 & Ct_{2}+Cs_{2}
\end{array}
\right] \left( 
\begin{array}{c}
\frac{dZ}{dt} \\ 
\frac{d\theta }{dt} \\ 
\frac{dZ_{u1}}{dt} \\ 
\frac{dZ_{u2}}{dt}
\end{array}
\right) +
\]

\[
+\left[ 
\begin{array}{cccc}
Ks_{2}+Ks_{1} & Ks_{2}.l_{2}-Ks_{1}.l_{1} & -Ks_{1} & -Ks_{2} \\ 
Ks_{2}.l_{2}-Ks_{1}.l_{1} & Ks_{2}.l_{2}^{2}+Ks_{1}.l_{1}^{2} & Ks_{1}.l_{1}
& -Ks_{2}.l_{2} \\ 
-Ks_{1} & Ks_{1}.l_{1} & Kt_{1}+Ks_{1} & 0 \\ 
-Ks_{2} & -Ks_{2}.l_{2} & 0 & Kt_{1}+Ks_{1}
\end{array}
\right] \left( 
\begin{array}{c}
Z \\ 
\theta  \\ 
Z_{u1} \\ 
Z_{u2}
\end{array}
\right) =
\]

\begin{equation}
=\left( 
\begin{array}{c}
0 \\ 
0 \\ 
Ct_{1}.\frac{dh_{1}}{dt}+Kt_{1}.h_{1} \\ 
Ct_{2}.\frac{dh_{2}}{dt}+Kt_{2}.h_{2}
\end{array}
\right)   \label{12veiculo}
\end{equation}

\bigskip Para o estudo dos modos de baixas frequ\^{e}ncias da massa
suspensa, os pneus podem ser considerados corpos r\'{i}gidos e o sistema de
equa\c{c}\~{o}es diferenciais acima fica reduzido ao seguinte sistema. 
\begin{equation}
\left[ 
\begin{array}{cc}
Ms & 0 \\ 
0 & Jy
\end{array}
\right] \left( 
\begin{array}{c}
\frac{dZ^{2}}{d^{2}t} \\ 
\frac{d\theta ^{2}}{d^{2}t}
\end{array}
\right) +\left[ 
\begin{array}{cc}
Cs_{2}+Cs_{1} & Cs_{2}.l_{2}-Cs_{1}.l_{1} \\ 
Cs_{2}.l_{2}-Cs_{1}.l_{1} & Cs_{2}.l_{2}^{2}+Cs_{1}.l_{1}^{2}
\end{array}
\right] \left( 
\begin{array}{c}
\frac{dZ}{dt} \\ 
\frac{d\theta }{dt}
\end{array}
\right) +  \label{12veiculo01}
\end{equation}

$\qquad \qquad \qquad 
\begin{array}{c}
\left[ 
\begin{array}{cc}
K_{2}+K_{1} & K_{2}.l_{2}-K_{1}.l_{1} \\ 
K_{2}.l_{2}-K_{1}.l_{1} & K_{2}.l_{2}^{2}+K_{1}.l_{1}^{2}
\end{array}
\right] \left( 
\begin{array}{c}
Z \\ 
\theta 
\end{array}
\right) =\left( 
\begin{array}{c}
K_{1}.h_{1} \\ 
K_{2}.h_{2}
\end{array}
\right) 
\end{array}
$

Para o sistema de equa\c{c}\~{o}es diferenciais acima o amortecimento dos
pneus foi desprezado, $Ct_{1}=Ct_{2}=0$, e os valores de $K_{1}$ e $K_{2}$
s\~{a}o derivados da associa\c{c}\~{a}o em s\'{e}rie das molas que
representam as molas da suspens\~{a}o e a rigidez dos pneus, ver equa\c{c}%
\~{a}o \ref{Ride rate}.

\begin{equation}
K_{i}=\frac{Ks_{i}.Kt_{i}}{Ks_{i}+Kt_{i}}  \label{Ride rate}
\end{equation}

Onde o \'{i}ndice $i$ assume valor $1$ para a suspens\~{a}o dianteira e
valor $2$ para a suspens\~{a}o traseira.

Os modos de vibra\c{c}\~{a}o do sistema de dois graus de liberdade n\~{a}o
amortecido, $Cs_{2}=Cs_{1}=0$, descrito pelo sistema de equa\c{c}\~{o}es \ref
{12veiculo01} podem ser obtidos utilizando a solu\c{c}\~{a}o do sistema de
equa\c{c}\~{o}es diferenciais homogeneas apresentado a seguir. 
\begin{equation}
\begin{array}{cc}
Ms & 0 \\ 
0 & Jy
\end{array}
. 
\begin{array}{c}
\frac{dZ^{2}}{d^{2}t} \\ 
\frac{d\theta ^{2}}{d^{2}t}
\end{array}
+ 
\begin{array}{cc}
K_{2}+K_{1} & K_{2}.l_{2}-K_{1}.l_{1} \\ 
K_{2}.l_{2}-K_{1}.l_{1} & K_{2}.l_{2}^{2}+K_{1}.l_{1}^{2}
\end{array}
. 
\begin{array}{c}
Z \\ 
\theta
\end{array}
= 
\begin{array}{c}
0 \\ 
0
\end{array}
\label{12veiculo02}
\end{equation}

Do sistema de equa\c{c}\~{o}es \ref{12veiculo02} acima nota-se que quando\ $%
K_{2}.l_{2}-K_{1}.l_{1}=0$\ as duas equa\c{c}\~{o}es do sistema se tornam
independentes,ou seja os modos de ''bounce'' e ''pitch'' se desacoplam
completamente.

O sistema de equa\c{c}\~{o}es diferenciais \ref{12veiculo02} \ possui quatro
raizes sendo duas delas imagin\'{a}rias e duas reais, as raizes reais
representam as frequ\^{e}ncias naturais do sistema. Resolvendo o sistema de
equa\c{c}\~{o}es \ref{12veiculo02}, obtem-se as seguintes frequ\^{e}ncias
naturais.

\begin{eqnarray}
\varpi _{1} &=&\sqrt{\frac{\left( \frac{K_{2}+K_{1}}{Ms}+\frac{%
K_{2}.l_{2}^{2}+K_{1}.l_{1}^{2}}{Jy}\right) }{2}+\sqrt{\frac{\left( \frac{%
K_{2}+K_{1}}{Ms}-\frac{K_{2}.l_{2}^{2}+K_{1}.l_{1}^{2}}{Jy}\right) ^{2}}{4}+%
\frac{\left( \frac{K_{2}.l_{2}-K_{1}.l_{1}}{Ms}\right) ^{2}.Ms}{Jy}}} 
\nonumber \\
\varpi _{2} &=&\sqrt{\frac{\left( \frac{K_{2}+K_{1}}{Ms}+\frac{%
K_{2}.l_{2}^{2}+K_{1}.l_{1}^{2}}{Jy}\right) }{2}-\sqrt{\frac{\left( \frac{%
K_{2}+K_{1}}{Ms}-\frac{K_{2}.l_{2}^{2}+K_{1}.l_{1}^{2}}{Jy}\right) ^{2}}{4}+%
\frac{\left( \frac{K_{2}.l_{2}-K_{1}.l_{1}}{Ms}\right) ^{2}.Ms}{Jy}}}
\label{12naturalfreq}
\end{eqnarray}

\bigskip Para determinar a posi\c{c}\~{a}o dos centros de repercuss\~{a}o da
carroceira em rela\c{c}\~{a}o ao CG \'{e} necess\'{a}rio determinar para
cada uma das frequ\^{e}ncias naturais a raz\~{a}o $\frac{Z}{\theta }$. Da
solu\c{c}\~{a}o do sistema de equa\c{c}\~{o}es diferenciais \ref{12veiculo02}%
, tem-se:

\begin{eqnarray}
\left( \frac{Z}{\theta }\right) _{\varpi _{1}} &=&-\frac{\left(
K_{2}.l_{2}-K_{1}.l_{1}\right) }{\left( K_{2}+K_{1}-\varpi
_{1}^{2}.Ms\right) }  \nonumber \\
\left( \frac{Z}{\theta }\right) _{\varpi _{2}} &=&-\frac{\left(
K_{2}.l_{2}^{2}+K_{1}.l_{1}^{2}-\varpi _{2}^{2}.Jy\right) }{\left(
K_{2}.l_{2}-K_{1}.l_{1}\right) }  \label{12veiculo centros}
\end{eqnarray}

\bigskip A dist\^{a}ncia de um dos dois centros de repercuss\~{a}o ser\'{a}
maior que a dist\^{a}ncia entre eixos, fazendo com que o centro de
repercuss\~{a}o se encontre fora do ve\'{i}culo e caracterizando
predominantemente o modo de ''bounce'', o outro centro se encontrar\'{a}
entre os dois pontos de fixa\c{c}\~{a}o da suspens\~{a}o \`{a} carroceria e
caracterizar\'{a} predominantemente o modo de ''pitch''. A figura \ref
{modes_bounce and Pitch} mostra de uma forma esquem\'{a}tica a localiza\c{c}%
\~{a}o dos centros de repercuss\~{a}o e os modos de ''bounce'' e\ ''pitch''.

\FRAME{fhFU}{3.2474in}{2.3237in}{0pt}{\Qcb{Modos de vibrar de um modelo 1/2
ve\'{i}culo com dois graus de liberdade.}}{\Qlb{modes_bounce and Pitch}}{%
12mode01.jpg}{\special{language "Scientific Word";type
"GRAPHIC";maintain-aspect-ratio TRUE;display "USEDEF";valid_file "F";width
3.2474in;height 2.3237in;depth 0pt;original-width 9.3331in;original-height
6.6668in;cropleft "0";croptop "1";cropright "1";cropbottom "0";filename
'12mode01.jpg';file-properties "XNPEU";}}

\bigskip Os par\^{a}metros sugeridos por Maurice Olley, que datam da
d\^{e}cada de 30, para o projeto de autom\'{o}veis com boas
caracter\'{i}sticas de ''ride'' n\~{a}o poderiam deixar de ser citados.
Segundo Gillespie \cite{Gillespie} , embora os m\'{e}todos utilizados para
determinar o n\'{i}vel conforto do autom\'{o}vel utilizado nos experimentos
tenham sido estrictamente subjetivos, as recomenda\c{c}\~{o}es de Olley
ainda hoje s\~{a}o utilizadas nos projetos de autom\'{o}veis. Os
crit\'{e}rios de Olley s\~{a}o:\footnote{%
Os crit\'{e}rios de Olley apresentados neste trabalho foram extraidos do
livro '' Fundamentals of Vehicle Dynamics '' de Thomas D. Gillespie.}

1)A rigidez da suspens\~{a}o dianteira deve ser $30\%$ menor que a rigidez
da suspens\~{a}o traseira, ou o centro da mola deve estar no m\'{i}nimo $%
6,5\%$ da dist\^{a}ncia entre eixos al\'{e}m do C.G.

De acordo com Gillespie \cite{Gillespie}, como a distribui\c{c}\~{a}o de
massa entre os eixos dos autom\'{o}veis \'{e} praticamente igual, este
primeiro crit\'{e}iro de Olley incorrer\'{a} em uma maior frequ\^{e}ncia de
resson\^{a}ncia para a suspens\~{a}o traseira. Como os movimentos de pitch
s\~{a}o considerados mais desagrad\'{a}veis que os de bounce, \'{e} at\'{e}
intuitivo concluir porque a primeira recomenda\c{c}\~{a}o de Olley \'{e}
seguida at\'{e} hoje. Para ilustrar o conceito utilizaremos um exemplo
apresentado por Gillespie \cite{Gillespie}.

Considerando um ve\'{i}culo que se encontra com uma eleva\c{c}\~{a}o,\
''lombada'', na pista. O tempo decorrido entre a perturba\c{c}\~{a}o da
suspens\~{a}o dianteira e a traseira, considerando que o autom\'{o}vel se
movimenta com uma velocidade constante $V$ para a frente e que a
dist\^{a}ncia entre eixos do autom\'{o}vel \'{e} $L$, ser\'{a}:

\[
t=\frac{L}{V} 
\]

A figura \ref{12velsusp} mostra as oscila\c{c}\~{o}es das suspens\~{o}es
dianteiras e traseiras devido a uma excita\c{c}\~{a}o do tipo descrito
acima. Pode-se notar que logo depois que a suspens\~{a}o dianteira passa
sobre a eleva\c{c}\~{a}o o ve\'{i}culo est\'{a} na pior condi\c{c}\~{a}o de
''pitching'', indicado pelos pontos A\ e B\ da figura \ref{12velsusp}.

\FRAME{fhFU}{3.2785in}{2.4647in}{0pt}{\Qcb{Ocila\c{c}\~{o}es de um
ve\'{i}culo passando por um lombada.}}{\Qlb{12velsusp}}{12velsusp.jpg}{%
\special{language "Scientific Word";type "GRAPHIC";maintain-aspect-ratio
TRUE;display "USEDEF";valid_file "F";width 3.2785in;height 2.4647in;depth
0pt;original-width 3.9998in;original-height 3in;cropleft "0";croptop
"1";cropright "1";cropbottom "0";filename '12velsusp.jpg';file-properties
"XNPEU";}}

Com uma maior rigidez da suspens\~{a}o traseira, maior frequ\^{e}ncia,
depois de uma ocila\c{c}\~{a}o e meia aproximadamente as suspens\~{o}es
dianteiras e traseiras estar\~{a}o se movimentando em fase, ou seja, a
carroceria estar\'{a} se movimentado em ''bounce'' at\'{e} que o movimento
seja totalmente amortecido.

2)As frequ\^{e}ncias de ''pitch e bounce'' devem ser pr\'{o}ximas: a
frequ\^{e}ncia de ''bounce'' deve ser menor que $1,2$ vezes a frequ\^{e}ncia
de ''pitch''.

De acordo com Genta \cite{Genta}, este segundo crit\'{e}rio \'{e} facilmente
preenchido pelos autom\'{o}veis modernos, por\'{e}m um problema pode
aparecer se a frequ\^{e}ncia de ''pitch'' for muito maior que a
frequ\^{e}ncia de ''bounce'', como pode acontecer quando o \'{I}ndice
din\^{a}mico for menor que 1 (geralmente ocorre em ve\'{i}culos com grandes
dist\^{a}ncias entre eixos e com as rodas bem pr\'{o}ximas dos extremos da
carroceria)

3)Nenhuma das frequ\^{e}ncias deve ser maior que $1,3Hz$.

4)A frequ\^{e}ncia de ''roll'' \ deve ser aproximadamente igual \`{a}s
frequ\^{e}ncias de ''pitch'' e de ''bounce''.

De uma forma geral , um autom\'{o}vel com o \'{I}ndice din\^{a}mico \
pr\'{o}ximo da unidade \ e cumprindo as recomenda\c{c}\~{o}es de Olley \'{e}
considerado um autom\'{o}vel com boas propriedades de ''ride comfort''. O
completo desacoplamento dos modos de ''pitch e Bounce'', $%
K_{2}.l_{2}-K_{1}.l_{1}=0$ (ver equa\c{c}\~{a}o \ref{12veiculo01}),\'{e}
indesej\'{a}vel pois as oscila\c{c}\~{o}es podem ser bastante irregulares.
Em contrapartida, o acoplamento entre esses dois modos de vibra\c{c}\~{a}o
\'{e} desej\'{a}vel j\'{a} que este evita grandes oscila\c{c}\~{o}es no modo
de ''pitch''\cite{Genta}.

Em casos onde a frequ\^{e}ncia natural de ''pitch'' \'{e} muito maior que a
frequ\^{e}ncia natural de ''bounce'' o conforto vibracional pode ser
afetado.Para controlar as frequ\^{e}ncias de ''bounce'' e ''pitch''
independentemente, sem mudar a posi\c{c}\~{a}o das rodas, as propriedades de
in\'{e}rcia da carroceria e sem utilizar sistemas de suspens\~{a}o ativa, as
suspens\~{o}es dianteiras e traseiras devem ser interconectadas. Se as
suspens\~{o}es forem interconectadas por um sistema com mola e amortecedor
que se opoem aos movimentos de ''pitch'' esse modo pode ser controlado
independentemente do modo de bounce. A figura \ref{2CV_figure} mostra um
exemplo da utiliza\c{c}\~{a}o de um sistema de suspens\~{o}es
interconectadas \cite{Genta}.

\FRAME{fhFU}{3.3849in}{0.9063in}{0pt}{\Qcb{Esquema da suspens\~{a}o
utilizada no Citro\"{e}n 2CV.}}{\Qlb{2CV_figure}}{slo2cvsusp.gif}{\special%
{language "Scientific Word";type "GRAPHIC";maintain-aspect-ratio
TRUE;display "USEDEF";valid_file "F";width 3.3849in;height 0.9063in;depth
0pt;original-width 8.2079in;original-height 2.1767in;cropleft "0";croptop
"1";cropright "1";cropbottom "0";filename 'slo2CVsusp.gif';file-properties
"XNPEU";}}

\section{A percep\c{c}\~{a}o do conforto}

Ao definir um autom\'{o}vel como sendo um autom\'{o}vel confot\'{a}vel os
par\^{a}metros utilizados para esse veredicto s\~{a}o muito mais subjetivos
do que se possa imaginar. De maneira geral pode-se dizer que a percep\c{c}%
\~{a}o do conforto de um autom\'{o}vel passa por um conjunto de
par\^{a}metros tacteis, visuais e aud\'{i}veis.Par\^{a}metros esses que
muitas vezes n\~{a}o se percebem separadamente e que se separados talvez
n\~{a}o causem desconforto, ou n\~{a}o produzam o mesma sensa\c{c}\~{a}o
agrad\'{a}vel.

Segundo Gillespie \cite{Gillespie}, um n\'{i}vel geral de conforto pode ser
influenciado pelo projeto dos bancos e a sua adapta\c{c}\~{a}o aos
ocupantes, a temperatura interna da cabine, a ventila\c{c}\~{a}o, o espa\c{c}%
o interno e muitos outros fatores. Alguns desses fatores, vibra\c{c}\~{o}es
e ru\'{i}dos por exemplo, podem ser medidos objetivamente, enquanto outros,
como a adapta\c{c}\~{a}o dos ocupantes \`{a} textura dos tecidos dos bancos,
est\'{a} fortemente ligada a an\'{a}lises subjetivas.

Do ponto de vista dos fatores (grandezas) mensur\'{a}veis que comp\~{o}e o
conforto de um autom\'{o}vel, o chamado conforto vibracional \'{e} um dos
mais explorados. Alguns tipos de movimento podem ser fontes de prazer ou
satisfa\c{c}\~{a}o e propiciam um sensa\c{c}\~{a}o de bem-estar e conforto,
entretanto o estudo do conforto vibracional est\'{a} focado nos movimentos
que causam desprazer, incomodo e desconforto.

A toler\^{a}ncia do ser humano \`{a}s vibra\c{c}\~{o}es foi e continua sendo
objeto de v\'{a}rios estudos, estudos esses conduzidos por profissionais de
diversas especialidades e com focos bastante diversos.

\subsection{Toler\^{a}ncia humana \`{a}s vibra\c{c}\~{o}es}

Inicialmente os estudos da sensibilidade humana a vibra\c{c}\~{o}es foram
relacionados ao ato de caminhar, nesta condi\c{c}\~{a}o um adulto que
desenvolve entre 70 e 90 passos por minuto (aproximadamente 1Hz.), e o
tronco desloca-se verticalmente aproximadamente 25 mm. Acreditava-se
ent\~{a}o que as oscila\c{c}\~{o}es do ve\'{i}culo deveriam ser mantidas
segundo estes par\^{a}metros.

Atualmente existem estudos mais aprofundados e espec\'{i}ficos da
sensibilidade humana \`{a} vibra\c{c}\~{o}es na posi\c{c}\~{a}o
sentada.Algumas normas foram elaboradas como a ISO2631 [ citar] e estudos
como os de Griffin \cite{Griffin} e outros indicam que a nossa sensibilidade
est\'{a} relacionada \`{a} frequ\^{e}ncia de vibra\c{c}\~{a}o,a amplitude\ e
tempo de exposi\c{c}\~{a}o.

A toler\^{a}ncia a determindas formas, e dire\c{c}\~{o}es de vibra\c{c}%
\~{o}es tamb\'{e}m est\~{a}o relacionadas a posi\c{c}\~{a}o do corpo (em
p\'{e}, sentado ou deitado), a parte do corpo que recebe a vibra\c{c}\~{a}o
e a v\'{a}rios outros par\^{a}metros. De acordo com Griffin \cite{Griffin},
a postura de uma pessoa sentada ou em p\'{e} pode determinar grandes mudan\c{%
c}as na toler\^{a}ncia a vibra\c{c}\~{o}es, uma pequena mudan\c{c}a na
orienta\c{c}\~{a}o das costas ou na inclina\c{c}\~{a}o da cabe\c{c}a pode
causar mudan\c{c}as substanciais nas vibra\c{c}\~{o}es transmitidas
atrav\'{e}s da coluna para a cabe\c{c}a.

\'{E} conveniente considerar a resposta humana a vibra\c{c}\~{o}es do corpo
envolvendo cinco efeitos separados: degrada\c{c}\~{a}o do conforto,
interfer\^{e}ncia com atividades, \emph{impaired helth}, percep\c{c}\~{a}o
de vibra\c{c}\~{o}es de baixa magnitude e a ocorr\^{e}ncia de \emph{motion
sickness}.

Ao se realizarem ensaios para estudos da toler\^{a}ncia do corpo humano a
vibra\c{c}\~{o}es na posi\c{c}\~{a}o sentada muitos estudiosos utilizaram os
bancos (deform\'{a}veis, complacentes, n\~{a}o r\'{i}gidos) que afetam as
vibra\c{c}\~{o}es que alcan\c{c}am o corpo. Como cada banco possui
propriedades din\^{a}micas diferentes estabeleceu-se que \'{e} preferivel a
utiliza\c{c}\~{a}o de bancos r\'{i}gidos sem encosto para a determina\c{c}%
\~{a}o experimental dos contornos do conforto vibracional.\cite{Griffin} Em
outros casos recomenda-se fazer a medi\c{c}\~{a}o pr\'{o}ximo da fixa\c{c}%
\~{a}o do banco ao assoalho e utilizar a transmissibilidade do banco para
avaliar a vibra\c{c}\~{a}o que poderia estar chegando ao ocupante do
ve\'{i}culo, embora esse procedimento n\~{a}o leve em considera\c{c}\~{a}o a
itera\c{c}\~{a}o entre a resposta do banco e do passageriro, j\'{a} que o
comportamento din\^{a}mico do banco \'{e} influenciado pela resposta do
ocupante.

A baixas frequ\^{e}ncias, abaixo de 1Hz ou 2 Hz dependendo da dire\c{c}%
\~{a}o da vibra\c{c}\~{a}o e da orienta\c{c}\~{a}o do corpo, as for\c{c}as
que atuam no corpo s\~{a}o praticamente proporcionais as acelera\c{c}\~{o}es
a que o corpo est\'{a} sujeito e o movimento do corpo todo est\'{a} em fase
com o movimento da fonte de vibra\c{c}\~{a}o. Com o aumento da
frequ\^{e}ncia de excita\c{c}\~{a}o o disconforto ficar\'{a} caracterizado
pela resson\^{a}ncia de de diferentes partes do corpo (abd\^{o}men, cabe\c{c}%
a, ombros, t\'{o}rax, etc.).

Em experimentos realizados por Griffin em pessoas sentadas em assentos
rigidos e sem encosto, vibra\c{c}\~{o}es horizontais em frequ\^{e}ncias na
faixa entre 1 e 2Hz foram apontadas como as mais desconfort\'{a}veis. Nesses
experimentos constatou-se que nessa faixa de frequ\^{e}ncia a vibra\c{c}%
\~{a}o do assento \'{e} mais facilmente transmitida para as partes
superiores do corpo. A presen\c{c}a do encosto, de forma a apoiar as costas
do ocupante do assento, aumenta a sensa\c{c}\~{a}o de desconforto na faixa
de 3 a 6 Hz para vibra\c{c}\~{o}es na dire\c{c}\~{a}o x.

Os movimentos induzidos em ve\`{i}culos n\~{a}o podem ser considerados como
vibra\c{c}\~{o}es puramente rotacionais da bacia do ocupante. A posi\c{c}%
\~{a}o do centro de rota\c{c}\~{a}o n\~{a}o pode ser definida facilmente,
mas s\~{a}o comumente abaixo do plano do assento. Isso significa que o
ocupante n\~{a}o estar\'{a} somente sujeito a vibra\c{c}\~{o}es de rota\c{c}%
\~{a}o, mas tamb\'{e}m estar\'{a} sujeito a movimentos de transla\c{c}\~{a}o.

Os estudos de Parsons and Griffin (1982)\cite{Griffin} mostram que
movimentos de ''roll'' puro causam mais desconforto do que movimentos de
puro ''pitch'' e que por sua vez causam mais desconforto do que puro ''yaw''.

Em movimentos de ''pitch'' o sujeito que tem as coxas apoiada sobre o banco
se sente menos confort\'{a}vel do que aquele que as tem os p\'{e}s apoiados
no assoalho e as coxas levantadas em consequ\^{e}ncia.

A vibra\c{c}\~{a}o do encosto para as costas dos bancos dos autom\'{o}veis
na horizontal pode ser, em determinadas frequ\^{e}ncias, a causa dooinante
do desconforto causado pela transmiss\~{a}o de vibra\c{c}\~{o}es para o
corpo do ocupante.

Ocupantes de assentos de autom\'{o}veis que possuem boa capacidade de isolar
as vibra\c{c}\~{o}es de alta frequ\^{e}ncia indicam que as vibra\c{c}\~{o}es
dos apoios dos p\'{e}s dominam a sensa\c{c}\~{a}o de desconforto a essas
frequ\^{e}ncias.

De acordo com Griffin\cite{Griffin}, uma s\'{e}rie de estudos sobre conforto
vibracional (ride comfort) foram realizados pela MIRA (Motor Industry
Research Association) entre os anos de 1960 e 1970. Em estudos inciais foram
utilizados 7 ve\'{i}culos e 12 condutores e as opini\~{o}es dos condutores
foram comparadas com medidas de vibra\c{c}\~{o}es feitas na dire\c{c}\~{a}o
vertical do assento e na cabe\c{c}a dos condutores. As vibra\c{c}\~{o}es da
cabe\c{c}a n\~{a}o foram consideradas como um bom par\^{a}metro para avaliar
o conforto vibracional de um autom\'{o}vel, mas as vibra\c{c}\~{o}es no
assento, o tamanho do ve\'{i}culo, o peso do ve\'{i}culo e o seu pre\c{c}o
foram os fatores subjetivos considerados como determinantes para um
autom\'{o}vel ser confort\'{a}vel.

A otimiza\c{c}\~{a}o dos assentos envolve a minimiza\c{c}\~{a}o da
transmissibilidade \ de vibra\c{c}\~{o}es atrav\'{e}s dos mesmos e ao mesmo
tempo a maximiza\c{c}\~{a}o do conforto est\'{a}tico, ou seja que a sua
deflex\~{a}o ao sentar-se n\~{a}o seja t\~{a}o grande.

A resposta em frequ\^{e}ncia da maioria dos assentos dos autom\'{o}veis
mostra uma frequ\^{e}ncia de resson\^{a}ncia pr\'{o}xima a 4Hz. Essa
frequ\^{e}ncia \'{e} influenciada pela imped\^{a}ncia do ocupante, e depende
da magnitude da excita\c{c}\~{a}o e de outros fatores\cite{Griffin}. Alguns
assentos possuem mecanismos de suspens\~{a}o com baixa rigidez que
geralmente possuem frequ\^{e}ncias de resson\^{a}ncia da ordem de 2Hz e
podem isolar vibra\c{c}\~{o}es verticais na faixa de 4 Hz.

Dos \ v\'{a}rios experimentos realizados ao longo de v\'{a}rios anos e
utilizando-se um sem n\'{u}mero de m\'{e}todos para a an\'{a}lise dos dados
de ensaio, a faixa entre 4 e 12 Hz \'{e} tida considerada a mais cr\'{i}tica
na toler\^{a}ncia vibracional vertical para o corpo humano. Isto ocorre em
grande parte devido a resson\^{a}ncia da cavidade abdominal que situa-se
pr\'{o}ximo a 5Hz.

A norma ISO2631 [citar] sugere uma s\'{e}rie de gr\'{a}ficos que apresentam
a r.m.s da acelera\c{c}\~{a}o que causa, em um determinado intervalo de
tempo, uma redu\c{c}\~{a}o da efici\^{e}ncia f\'{i}sica, a figura \ref
{ISO2631-01} a seguir \'{e} um exemplo desse tipo de gr\'{a}fico. Do
gr\'{a}fico apresentado na figura \ref{ISO2631-01} nota-se que a faixa de
frequ\^{e}ncia que mais afeta um ser humano sentado \'{e} a faixa entre 4 e
8 Hz segundo a norma ISO2631[citar].

\FRAME{fhFU}{8.5932cm}{6.8842cm}{0pt}{\Qcb{R.m.s da acelera\c{c}\~{a}o em
rela\c{c}\~{a}o a frequ\^{e}ncia, que causa redu\c{c}\~{a}o da
efici\^{e}ncia f\'{i}sica para uma pessoa sentada (ISO 2631).}}{\Qlb{%
ISO2631-01}}{graph03_genta.jpg}{\special{language "Scientific Word";type
"GRAPHIC";maintain-aspect-ratio TRUE;display "USEDEF";valid_file "F";width
8.5932cm;height 6.8842cm;depth 0pt;original-width 8.3333in;original-height
6.6668in;cropleft "0";croptop "1.0002";cropright "1";cropbottom "0";filename
'graph03_genta.jpg';file-properties "XNPEU";}}

\chapter{An\'{a}lise de Sensibilidade}

A chamada an\'{a}lise de sensibilidade surgiu, recentemente, como um
poderoso procedimento para a otimiza\c{c}\~{a}o de sistemas; \'{e} o estudo
do grau de influ\^{e}ncia da altera\c{c}\~{a}o de determinados
par\^{a}metros de projeto nas caracter\'{i}sticas do mesmo que \'{e}
caracterizado por um funcional.

A an\'{a}lise de sensibilidade, atrav\'{e}s de formula\c{c}\~{o}es
matem\'{a}ticas, apresenta a depend\^{e}ncia entre a varia\c{c}\~{a}o dos
par\^{a}metros de projeto (propriedades do material, geometria, etc.) e as
respostas do sistema projetado medidas em termos de deslocamentos, acelera\c{%
c}\~{o}es, peso pr\'{o}prio, custo, etc..

O principal objetivo da an\'{a}lise de sensibilidade \'{e} o c\'{a}lculo do
gradiente de fun\c{c}\~{o}es ou de um funcional em rela\c{c}\~{a}o aos
par\^{a}metros que determinam as caracter\'{i}sticas do sistema.

\section{O problema de otimiza\c{c}\~{a}o de sistemas}

Em qualquer formula\c{c}\~{a}o para a sele\c{c}\~{a}o de par\^{a}metros
\'{o}timos de um sistema , \'{e} necess\'{a}rio identificar adequadamente:

1. as vari\'{a}veis de projeto(par\^{a}metros) que descrevem o sistema;

2. a fun\c{c}\~{a}o custo ou fun\c{c}\~{a}o objetivo que ser\'{a} otimizada;
e

3. as restri\c{c}\~{o}es que garantam uma performance segura do sistema.

As vari\'{a}veis de projeto sujeitas a altera\c{c}\~{o}es s\~{a}o definidas
como um vetor $p=\{p_{e}\}$, onde $e=1,2,3...,D$ e $D$ \'{e} o n\'{u}mero
total dessas vari\'{a}veis; atrav\'{e}s da mudan\c{c}a dos valores dos
componentes deste vetor podem ser definidos diferentes sistemas e dentre
eles o \'{o}timo.

A chamada fun\c{c}\~{a}o custo ou fun\c{c}\~{a}o objetivo, $G(p)$, \'{e}
utilizada para definir as grandezas a serem otimizadas, podendo ser um
escalar ou uma fun\c{c}\~{a}o vetorial.

Normalmente as vari\'{a}veis de projeto $p$ est\~{a}o limitadas inferior e
superiormente, por restri\c{c}\~{o}es de desigualdade. Outros tipos de restri%
\c{c}\~{a}o tamb\'{e}m s\~{a}o utilizadas, as chamadas restri\c{c}\~{o}es de
igualdade, como exemplo, podem ser citadas as equa\c{c}\~{o}es de
equil\'{i}brio.

De posse de todos os dados que caracterizam o problema, as vari\'{a}veis de
projeto $p$, bem definidas, a fun\c{c}\~{a}o objetivo $G(p)$, o n\'{u}mero
de restri\c{c}\~{o}es de igualdade $Gc(p)=0,c=1,2,3...,C^{e}$, e o
n\'{u}mero de restri\c{c}\~{o}es de desigualdade $Gc(p)\leqslant 0$, $%
c=C^{e}+1,...,C$ caracteriza-se o problema de otimiza\c{c}\~{a}o por:

\begin{center}
Encontrar o vetor vari\'{a}vel de projeto $p=\{p_{e}\}$ , onde $e=1,2,3...,D$%
, que minimiza a fun\c{c}\~{a}o objetivo $G(p)$ de tal\textbf{\ }maneira que
p satisfa\c{c}a as restri\c{c}\~{o}es $Gc(p)=0,c=1,2,3...,C^{e}$, e $%
Gc(p)\leqslant 0$, $c=C^{e}+1,...,C$.
\end{center}

Para a solu\c{c}\~{a}o deste tipo de problema t\^{e}m-se que considerar que:

1. todas as fun\c{c}\~{o}es s\~{a}o cont\'{i}nuas e diferenciaveis segundo
Gateaux ou Fr\'{e}chet;

2. as fun\c{c}\~{o}es objetivo e de restri\c{c}\~{o}es (tipicamente
apresentadas implicita ou explicitamente em termos das vari\'{a}veis de
projeto $p$) podem ser calculadas uma vez que o vetor vari\'{a}veis de
projeto \'{e} especificado.

\textit{Como exemplo, seja uma estrutura cuja resposta \'{e} governada por
um n\'{u}mero finito de vari\'{a}veis denominadas graus de liberdade.
Considerando um comportamento linear el\'{a}stico do material e que a
estrutura trabalha dentro do campo dos pequenos deslocamentos, tem-se que:}

\textit{1. a configura\c{c}\~{a}o deformada da estrutura se confunde com a
configura\c{c}\~{a}o inicial;}

\textit{2. a resposta \'{e} dada por um sistema linear de equa\c{c}\~{o}es;}

\textit{3. o princ\'{i}pio de superposi\c{c}\~{a}o \'{e} v\'{a}lido. \ \ \ \
\ \ \ \ \ \ \ \ \ \ \ \ \ \ \ \ \ \ \ \ \ \ \ \ \ \ \ \ \ \ \ \ \ \ \ \ \ \
\ \ \ \ \ \ \ \ \ \ \ \ \ \ \ \ \ \ \ \ \ \ \ \ \ \ \ \ \ \ \ \ \ \ \ \ \ \
\ \ \ \ \ \ \ \ \ \ \ \ \ \ \ \ \ \ \ \ \ \ \ \ \ \ \ \ \ \ \ \ \ \ \ \ \ \
\ \ \ \ \ \ \ \ \ \ \ \ \ \ \ \ \ \ \ \ \ \ \ \ \ \ \ \ \ \ \ \ \ \ \ \ \ \
\ \ \ \ \ \ \ \ \ O modelo mec\^{a}nico com o qual analisamos as estruturas
reticuladas e as estruturas aporticadas \'{e} por sua natureza discreto, ou
seja, depende de um n\'{u}mero finito de graus de liberdade. N\~{a}o \'{e} o
que acontece nos modelos mec\^{a}nicos aplicados a cascas, placas e
s\'{o}lidos tridimensionais que s\~{a}o meios cont\'{i}nuos.}

\textit{Neste \'{u}ltimo caso, utilizam-se m\'{e}todos aproximados tais como
o M\'{e}todo dos Elementos Finitos e transforma-se a estrutura inicialmente
cont\'{i}nua em uma discreta, cujos graus de liberdade est\~{a}o associados
a deslocamentos nodais generalizados. Dessa maneira, a partir dessa aproxima%
\c{c}\~{a}o, a an\'{a}lise da resposta da estrutura pode ser tratada como a
das estruturas discretas.}

\textit{Considerando que somente cargas est\'{a}ticas s\~{a}o aplicadas,
aqui designadas por }$F\in \Re ^{N}$\textit{, diz-se que }$u$\textit{\
equilibra }$F$\textit{\ se o seguinte sistema de equa\c{c}\~{o}es for
satisfeito: } 
\begin{equation}
\medskip Ku=F  \label{eq1}
\end{equation}
\textit{Em an\'{a}lise estrutural, }$K$\textit{\ \'{e} chamada de matriz
rigidez global, }$F$\textit{\ \'{e} o vetor de cargas nodais equivalentes e }%
$u$\textit{\ \'{e} o vetor de deslocamentos nodais.}

\textit{A matriz de rigidez depende das caracter\'{i}sticas do material e de
par\^{a}metros geom\'{e}tricos que definem a forma dos corpos tais como:
espessura para as placas e cascas e se\c{c}\~{a}o transversal para o caso de
vigas, arcos, etc..}

\textit{Os projetistas trabalham com este conjunto de par\^{a}metros,
escolhendo alguns deles como vari\'{a}veis de projeto que s\~{a}o
representados por }$p\in \Re ^{m}$\textit{, onde }$m$\textit{\ \'{e} o
n\'{u}mero de vari\'{a}veis de projeto adotado pelos projetistas.}

\textit{Geralmente as for\c{c}as nodais equivalentes s\~{a}o fun\c{c}\~{a}o
de }$p$\textit{\ (por exemplo, quando se considera o peso pr\'{o}prio de uma
viga, }$F$\textit{\ depender\'{a} da densidade do material e das
caracter\'{i}sticas geom\'{e}tricas da viga.) Portanto, a Eq.\ref{eq1} pode
ser reescrita da seguinte maneira: } 
\begin{equation}
\medskip K(p)u=F(p)  \label{eq2}
\end{equation}
\textit{A partir da equa\c{c}\~{a}o acima, os projetistas podem calcular u
para cada valor das vari\'{a}veis de projeto }$p$\textit{\ resolvendo o
sistema de equa\c{c}\~{o}es lineares.}

\textit{Com base nos dados obtidos os projetistas podem definir o valor das
vari\'{a}veis de projeto para obter o melhor desempenho. Para isto,
definem-se crit\'{e}rios que permitem comparar os diferentes projetos.
Geralmente s\~{a}o definidas fun\c{c}\~{o}es de custo, ou fun\c{c}\~{o}es
objetivo, que designaremos por }$G(p,u)$\textit{. Se tomarmos como fun\c{c}%
\~{a}o objetivo o peso total da estrutura, o melhor projeto consiste em
determinar }$p$\textit{\ de tal maneira que minimize o peso da estrutura e
que satisfa\c{c}a o sistema de equa\c{c}\~{o}es Eq.\ref{eq2}. Al\'{e}m das
restri\c{c}\~{o}es impostas pela Eq.\ref{eq2} existir\~{a}o outras restri\c{c%
}\~{o}es tais como normas, c\'{o}digos de projeto, processos de fabrica\c{c}%
\~{a}o, custo, etc.. Normalmente as vari\'{a}veis de projeto }$p$\textit{\
est\~{a}o limitadas inferior e superiormente, de tal maneira que: } 
\[
p_{i}\leqslant p\leqslant p^{s} 
\]
\textit{onde } 
\[
\medskip p_{i},p^{s}\in \Re ^{m} 
\]
\textit{Existir\~{a}o, tamb\'{e}m, restri\c{c}\~{o}es provenientes de limita%
\c{c}\~{o}es nos deslocamentos, nas deforma\c{c}\~{o}es, nas tens\~{o}es,
etc.. Desta maneira , tem-se que }$C(p,u)\leqslant 0$\textit{\ , }$C\in \Re
^{q}$\textit{, onde }$q$\textit{\ representa o n\'{u}mero total de restri\c{c%
}\~{o}es.}

\textit{Resumindo, o problema de otimiza\c{c}\~{a}o estrutural pode ser
descrito da seguinte maneira:}

\textit{Determinar }$p\in X_{ad}$\textit{\ tal que }$P:=\arg
\inf_{X_{ad}}G(p,u)$\textit{,}

\textit{onde }$X_{ad}=\{p\in \Re ^{m};p_{i}\leqslant p\leqslant
p^{s};C(p,u)\leqslant 0;K(p).u=F(p)\}$

\subsection{A an\'{a}lise de sensibilidade}

\textit{Da mesma maneira que foi visto na parte introdut\'{o}ria, a formula%
\c{c}\~{a}o matem\'{a}tica da an\'{a}lise de sensibilidade ser\'{a} tratada
do ponto de vista da otimiza\c{c}\~{a}o estrutural, j\'{a} que \'{e} um
assunto de pressuposto conhecimento.}

\textit{Considerando que a geometria e as caracter\'{i}sticas da estrutura
est\~{a}o descritas em termos dos par\^{a}metros de projeto, a matriz
rigidez generalizada global e o vetor carregamento s\~{a}o fun\c{c}\~{o}es
dos par\^{a}metros de projeto, ou seja:}

\[
\mathit{K}_{g}\mathit{=K}_{g}\mathit{(p)} 
\]

\[
\mathit{\medskip F}_{g}\mathit{=F}_{g}\mathit{(p)} 
\]

\textit{Reescrevendo a Eq.\ref{eq2} em fun\c{c}\~{a}o dos par\^{a}metros
globais, tem-se:}

\begin{equation}
\mathit{\medskip K}_{g}\mathit{(p)u}_{g}\mathit{=F}_{g}\mathit{(p)}
\label{eq3}
\end{equation}

\textit{onde o vetor }$p=[p_{1},...,p_{m}]^{T}$\textit{\ \'{e} o vetor das
vari\'{a}veis de projeto e vari\'{a}veis que localizam n\'{o}s selecionados
na estrutura e }$u_{g}$\textit{\ \'{e} o vetor dos deslocamentos nodais.
\'{E} pressuposto que as condi\c{c}\~{o}es cinematicamente admiss\'{i}veis
(condi\c{c}\~{o}es de contorno e condi\c{c}\~{o}es de interface) n\~{a}o
s\~{a}o fun\c{c}\~{o}es expl\'{i}citas de projeto.}

\textit{Uma vez que a matriz rigidez generalizada global e o vetor
carregamento s\~{a}o dependentes das vari\'{a}veis de projeto, as formas
bilinear de energia e linear do carregamento da equa\c{c}\~{a}o de
equil\'{i}brio variacional \cite{choi} podem ser escritas como:}

\begin{center}
$a_{p}(u_{g},\overline{v}_{g})=\overline{v}_{g}^{T}K_{g}(p)u_{g}\hspace{0.5in%
}$\textit{ou\hspace{0.5in} }$a_{p}(u_{g},\overline{v}_{g})=K_{g}(p)u_{g}%
\cdot \overline{v}_{g}$

$l_{p}(\overline{v}_{g})=\overline{v}_{p}^{T}F_{g}(p)\hspace*{0.5in}$\textit{%
ou}$\hspace{0.5in}l_{p}(\overline{v}_{g})=F_{g}(p)\cdot \overline{v}_{g}$
\end{center}

\textit{reescrevendo a Eq.\ref{eq3} utilizando a nota\c{c}\~{a}o acima,
chega-se a: } 
\begin{eqnarray}
\overline{v}_{g}^{T}K_{g}(p)u_{g} &=&\overline{v}_{p}^{T}F_{p}(p)
\label{eq4} \\
\medskip a_{p}(u_{g},\overline{v}_{g}) &=&l_{p}(\overline{v}_{g})  \nonumber
\end{eqnarray}

\textit{Deve-se lembrar que existe uma \'{u}nica solu\c{c}\~{a}o }$u_{g}$%
\textit{\ para a Eq.\ref{eq4}. Uma vez que essa equa\c{c}\~{a}o depende
explicitamente das vari\'{a}veis de projeto, \'{e} \'{o}bvio que as solu\c{c}%
\~{a}o ug tamb\'{e}m \'{e} dependente dessas vari\'{a}veis, ou seja:}

\begin{equation}
\mathit{\medskip u}_{g}\mathit{=u}_{g}\mathit{(p)}  \label{eq5}
\end{equation}

\textit{Na maioria dos problemas de projeto estrutural, as fun\c{c}\~{o}es
objetivo devem ser minimizadas, sujeitas \`{a}s restri\c{c}\~{o}es de
tens\~{a}o, deslocamento e outras vari\'{a}veis de projeto. Considere agora
uma fun\c{c}\~{a}o geral que possa representar qualquer uma das altera\c{c}%
\~{o}es no comportamento da estrutura, escrita na seguinte forma:}

\begin{equation}
\mathit{\medskip \psi =\psi (p,u}_{g}\mathit{(p))}  \label{eq6}
\end{equation}

\textit{A depend\^{e}ncia dessa fun\c{c}\~{a}o em rela\c{c}\~{a}o ao projeto
d\'{a}-se de dois modos: (1) uma depend\^{e}ncia expl\'{i}cita das
vari\'{a}veis de projeto; e (2) uma depend\^{e}ncia impl\'{i}cita
atrav\'{e}s da solu\c{c}\~{a}o de }$u_{g}$\textit{\ das equa\c{c}\~{o}es de
estado. O objetivo da an\'{a}lise de sensibilidade \'{e} determinar a total
influ\^{e}ncia de tais fun\c{c}\~{o}es no projeto, ou seja, determinar }$%
d\psi /dp$\textit{.}

\textit{A quest\~{a}o que surge no momento \'{e}: dado que a fun\c{c}\~{a}o }%
$\psi $\textit{\ \'{e} diferenci\'{a}vel em todos os seus argumentos, a sua
influ\^{e}ncia nos par\^{a}metros de projeto }$d\psi /dp$\textit{\
tamb\'{e}m o \'{e}? Se a solu\c{c}\~{a}o }$u_{g}$\textit{\ das equa\c{c}%
\~{o}es de estado \'{e} diferenci\'{a}vel em rela\c{c}\~{a}o \`{a}s
vari\'{a}veis de projeto, como pode ser calculada a derivada de }$\psi $%
\textit{\ em rela\c{c}\~{a}o \`{a}s vari\'{a}veis de projeto?}

\textit{\bigskip }

\subsection{Diferenciabilidade da vari\'{a}vel dependente.}

Considerando que as condi\c{c}\~{o}es de contorno foram aplicadas
inicialmente de forma a eliminar as vari\'{a}veis de estado dependentes, ou
seja, impedem os deslocamentos de corpo r\'{i}gido, pode-se reescrever a equa%
\c{c}\~{a}o Eq.\ref{eq3} na forma a seguir:

\begin{equation}
\medskip K(p)u=F(p)  \label{eq7}
\end{equation}
onde $K(p)$ \'{e} a matriz rigidez global modificada pelas condi\c{c}\~{o}es
de contorno e $F(p)$ a carga modificada pelas condi\c{c}\~{o}es de contorno.
Sendo a matriz global reduzida positiva definida por defini\c{c}\~{a}o,
considerando que todos os par\^{a}metros das matrizes $K(p)$ e $F(p)$
s\~{a}o $n$ vezes continuamente diferenci\'{a}veis em rela\c{c}\~{a}o aos
respectivos par\^{a}metros de projeto. O teorema da fun\c{c}\~{a}o
impl\'{i}cita garante que a solu\c{c}\~{a}o $u=u(p)$ tamb\'{e}m \'{e} $n$
vezes continuamente diferenci\'{a}vel. Portanto a quest\~{a}o anterior a
respeito da diferenciabilidade de $u$ em rela\c{c}\~{a}o \`{a}s
vari\'{a}veis de projeto est\'{a} solucionada. Entretanto, o problema do
c\'{a}lculo das derivadas totais da fun\c{c}\~{a}o $\psi $ da Eq.\ref{eq6}
em rela\c{c}\~{a}o \`{a}s vari\'{a}veis de projeto ainda deve ser
solucionado. A seguir, apresenta-se dois m\'{e}todos para o c\'{a}lculo
dessas derivadas.

\subsection{M\'{e}todo de diferencia\c{c}\~{a}o direta}

Utilizando a regra da cadeia para a diferencia\c{c}\~{a}o e a nota\c{c}%
\~{a}o matricial para representar as derivadas, a derivada total de $\psi $
em rela\c{c}\~{a}o a $p$ pode ser calculada como:

\begin{equation}
\medskip \frac{d\psi }{dp}=\frac{\partial \psi }{\partial p}+\frac{\partial
\psi }{\partial u}\frac{du}{dp}  \label{eq8}
\end{equation}

Derivando ambos os lados da Eq.\ref{eq7} em rela\c{c}\~{a}o a $p$, tem-se:

\begin{equation}
\medskip K(p)\frac{du}{dp}=-\frac{\partial }{\partial p}(K(p)\widetilde{u})+%
\frac{\partial F(p)}{\partial p}  \label{eq9}
\end{equation}

onde o s\'{i}mbolo ($\symbol{126}$) indica que a vari\'{a}vel deve ser
mantida constante durante o processo de diferencia\c{c}\~{a}o. Uma vez que a
matriz $K(p)$ \'{e} n\~{a}o-singular, a Eq.\ref{eq9} pode ser resolvida para 
$du/dp$ de forma que:

\begin{equation}
\medskip \frac{du}{dp}=K^{-1}(p)[\frac{\partial F(p)}{\partial p}-\frac{%
\partial }{\partial p}(K(p)\widetilde{u})]  \label{eq10}
\end{equation}

Esse resultado pode ser ent\~{a}o substitu\'{i}do na Eq.\ref{eq8}, obtendo:

\begin{equation}
\medskip \frac{d\psi }{dp}=\frac{\partial \psi }{\partial p}+\frac{\partial
\psi }{\partial u}K^{-1}(p)\frac{\partial }{\partial p}[F(p)-K(p)\widetilde{u%
})]  \label{eq11}
\end{equation}

A utilidade da Eq.\ref{eq11} \'{e} question\'{a}vel dado que em aplica\c{c}%
\~{o}es real\'{i}sticas o c\'{a}lculo direto de $K^{-1}(p)$ \'{e}
impratic\'{a}vel. Duas alternativas podem ser utilizadas para superar essa
dificuldade. Na primeira, a Eq.\ref{eq9} pode ser resolvida numericamente
para $du/dp$ e substitu\'{i}da na Eq.\ref{eq8} para obter-se o resultado
desejado. Isso \'{e} conhecido como o M\'{e}todo da Diferencia\c{c}\~{a}o
Direta, que tem sido largamente empregado em otimiza\c{c}\~{a}o estrutural.
A outra alternativa \'{e} conhecida como M\'{e}todo Adjunto, que ser\'{a}
discutido a seguir.

\subsection{M\'{e}todo Adjunto (M\'{e}todo da vari\'{a}vel adjunta)}

Um enfoque alternativo \'{e} definir uma vari\'{a}vel adjunta $\lambda $
sendo:

\begin{equation}
\medskip \lambda \equiv [\frac{\partial \psi }{\partial u}%
K^{-1}(p)]^{T}=K^{-1}(p)\frac{\partial \psi ^{T}}{\partial u}  \label{eq12}
\end{equation}

onde a simetria da matriz $K$ foi empregada. Ao inv\'{e}s de avaliar $%
\lambda $ diretamente a partir da Eq.\ref{eq12}, que envolve $K^{-1}(p)$,
ambos os lados da Eq.\ref{eq12} podem ser multiplicados pela matriz $K(p)$ e
obter assim a seguinte equa\c{c}\~{a}o adjunta em $\lambda $ :

\begin{equation}
\medskip K(p)\lambda =\frac{\partial \psi ^{T}}{\partial u}  \label{eq13}
\end{equation}

Essa \'{u}ltima equa\c{c}\~{a}o pode ser resolvida para $\lambda $ e o seu
resultado substitu\'{i}do, utilizando a Eq.\ref{eq12} , na Eq.\ref{eq11} a
fim de obter-se:

\begin{equation}
\medskip \frac{d\psi }{dp}=\frac{\partial \psi }{\partial p}+\lambda ^{T}[%
\frac{\partial F(p)}{\partial p}-\frac{\partial }{\partial p}(K(p)\widetilde{%
u})]  \label{eq14}
\end{equation}

Uma forma muito mais conveniente para o c\'{a}lculo derivativo \'{e}:

\begin{equation}
\medskip \frac{d\psi }{dp}=\frac{\partial \psi }{\partial p}+\frac{\partial 
}{\partial p}[\widetilde{\lambda }^{T}F(p)-\widetilde{\lambda }^{T}K(p)%
\widetilde{u}]  \label{eq15}
\end{equation}

\subsection{T\'{i}tulo da sub se\c{c}\~{a}o}

Texto da sub se\c{c}\~{a}o\newpage

\chapter{\protect\bigskip Implementa\c{c}\~{a}o do Modelo}

\chapter{Apresenta\c{c}\~{a}o e An\'{a}lise dos Resultados}

Para a an\'{a}lise dos resultados obtidos no estudo realizado, a an\'{a}lise
das respostas em frequ\^{e}ncia das acelera\c{c}\~{o}es verticais da
carroceira do autom\'{o}vel foi a escolhida por se tratar de um dos
par\^{a}metros mais importantes no estudo do conforto vibracional de um
autom\'{o}vel. Especialmente a an\'{a}lise no dom\'{i}nio da frequ\^{e}ncia
por facilitar a an\'{a}lise dos resultados e por ser largamente utilizada
para o estudo de sistemas din\^{a}micos.

A an\'{a}lise espectral foi utilizada para reprsentar a resposta em
frequ\^{e}ncia da acelera\c{c}\~{a}o vertical no C.G. da carroceria do
ve\'{i}culo modelado. As PSDs foram obtidas utilizando o Software Microcal
Origin. A PSD foi filtrada com um filtro retangular\footnote{%
Conhecido na literatura como '' Rectangular Window Function''.} que
geralmente representa de forma mais precisa as amplitudes se comparado com
os demais filtros \cite{ADAMS Manual}.

\bigskip O Software ADAMS, que foi utilizado para a simula\c{c}\~{a}o,

\section{\protect\bigskip An\'{a}lise Modal}

A an\'{a}lise modal permite que se identifiquem bem os modos de vibrar e as
frequ\^{e}ncias correspondentes a esses modos, e complementa e auxilia a
an\'{a}lise espectral uma vez que se sabe qual \'{e} o modo excitado a em
cada frequ\^{e}ncia.\ A an\'{a}lise modal foi de grande utilidade para
ajustar e corrigir alguns par\^{a}metros do modelo, como posi\c{c}\~{o}es e
tipos de restri\c{c}\~{o}es, para fazer com que o modelo fosse capaz de
representar os modos de interesse para o estudo. A falta de um modelamento
mais completo para o pneu fez com que o ajuste para a representa\c{c}\~{a}o
de alguns do modos naturais fosse muito trabalhosa.

Os modos naturais do modelo mostram um primeiro panorama do que se observou
posteriormente na an\'{a}lise espectral. A tabela \ref{ResultsModos} mostra
a varia\c{c}\~{a}o das frequ\^{e}ncias naturais dos principais modos de
interesse do modelo em fun\c{c}\~{a}o da varia\c{c}\~{a}o do di\^{a}metro da
barra estabilizadora.

Observa-se que a medida que aumenta o di\^{a}metro da barra estabilizadora \
dianteira as frequ\^{e}ncias dos modos relacionados ou dependentes da
suspens\~{a}o dianteira aumentam.

As figuras a seguir apresentam os principais modos de vibrar do modelo.

\begin{equation}
\begin{tabular}{|c|c|c|c|c|c|}
\hline
Modo/Di\^{a}m. & $16[mm]$ & $18[mm]$ & $20[mm]$ & $22[mm]$ & $24[mm]$ \\ 
\hline
\textit{Pitch} & $1,0693Hz$ & $1,0775Hz$ & $1,0830Hz$ & $1,0865Hz$ & $%
1,0888Hz$ \\ \hline
\textit{Bounce} & $1,4241Hz$ & $1,5610Hz$ & $1,7112Hz$ & $1,8660Hz$ & $%
2,0183Hz$ \\ \hline
\textit{Wheel Hop} (Diant. em fase) & $10,3405Hz$ & $10,7131Hz$ & $11,1891Hz$
& $11,7768Hz$ & $12,4831Hz$ \\ \hline
\textit{Wheel Hop} (Diant. fora de fase) & $10,8385Hz$ & $11,1740Hz$ & $%
11,5998Hz$ & $12,1243Hz$ & $12,7562Hz$ \\ \hline
\textit{Wheel Hop} (Tras. em fase) & $11,0120Hz$ & $11,0194Hz$ & $11,0254Hz$
& $11,0302Hz$ & $11,0339Hz$ \\ \hline
\textit{Wheel Hop} (Tras. fora de fase) & $38,6924Hz$ & $38,7228Hz$ & $%
38,7479Hz$ & $38,7675Hz$ & $38,7827Hz$ \\ \hline
\end{tabular}
\label{ResultsModos}
\end{equation}
\bigskip \FRAME{fhFU}{13.9705cm}{10.3549cm}{0pt}{\Qcb{1%
%TCIMACRO{\UNICODE{0xba} }%
%BeginExpansion
${{}^o}$%
%EndExpansion
modo da carroceria, apresenta-se predominantemente como o modo de\textit{\
Pitch}, com centro de repercuss\~{a}o localizado na suspens\~{a}o dianteira.}%
}{\Qlb{Result:pitchmode}}{pitch_mode.bmp}{\special{language "Scientific
Word";type "GRAPHIC";maintain-aspect-ratio TRUE;display "USEDEF";valid_file
"F";width 13.9705cm;height 10.3549cm;depth 0pt;original-width
10.6666in;original-height 8.0004in;cropleft "0";croptop "0.9865";cropright
"0.9998";cropbottom "0";filename 'pitch_mode.bmp';file-properties "XNPEU";}}

\FRAME{fbhFU}{5.4907in}{4.1243in}{0pt}{\Qcb{2%
%TCIMACRO{\UNICODE{0xba} }%
%BeginExpansion
${{}^o}$%
%EndExpansion
modo da carroceria apresenta-se predominantemente como modo de \textit{Bounce%
}, com centro de repercuss\~{a}o atr\'{a}s da suspens\~{a}o dianteira. }}{%
\Qlb{Result:bouncemode}}{bounce_mode.bmp}{\special{language "Scientific
Word";type "GRAPHIC";maintain-aspect-ratio TRUE;display "USEDEF";valid_file
"F";width 5.4907in;height 4.1243in;depth 0pt;original-width
10.6666in;original-height 8.0004in;cropleft "0";croptop "1";cropright
"1";cropbottom "0";filename 'Bounce_mode.bmp';file-properties "XNPEU";}}

\FRAME{ftbpFU}{5.5737in}{4.1866in}{0pt}{\Qcb{Modo de vibra\c{c}\~{a}o das
massas n\~{a}o suspensas dianteiras oscilando em fase sobre as molas da
suspens\~{a}o dianteira. (\textit{Wheel Hop})}}{\Qlb{Result:WHFFemode}}{%
front_wheelhope_fase.bmp}{\special{language "Scientific Word";type
"GRAPHIC";maintain-aspect-ratio TRUE;display "USEDEF";valid_file "F";width
5.5737in;height 4.1866in;depth 0pt;original-width 10.6666in;original-height
8.0004in;cropleft "0";croptop "1";cropright "1";cropbottom "0";filename
'front_WheelHope_fase.bmp';file-properties "XNPEU";}}

\FRAME{fhFU}{5.6152in}{4.2177in}{0pt}{\Qcb{Modo de vibra\c{c}\~{a}o das
massas n\~{a}o suspensas dianteiras oscilando fora de fase sobre as molas da
suspens\~{a}o dianteira. (\textit{Wheel Hop})}}{\Qlb{Result:WHFOFemode}}{%
front_wheelhope_outfase01.bmp}{\special{language "Scientific Word";type
"GRAPHIC";maintain-aspect-ratio TRUE;display "USEDEF";valid_file "F";width
5.6152in;height 4.2177in;depth 0pt;original-width 10.6666in;original-height
8.0004in;cropleft "0";croptop "1";cropright "1";cropbottom "0";filename
'front_WheelHope_outfase01.bmp';file-properties "XNPEU";}}

\FRAME{fhFU}{5.5322in}{4.1554in}{0pt}{\Qcb{Modo de vibra\c{c}\~{a}o das
massas n\~{a}o suspensas traseiras oscilando em fase sobre as molas da
suspens\~{a}o traseira. (\textit{Wheel Hop}). }}{\Qlb{Result:WHRFemode}}{%
rear_wheelhope_fase.bmp}{\special{language "Scientific Word";type
"GRAPHIC";maintain-aspect-ratio TRUE;display "USEDEF";valid_file "F";width
5.5322in;height 4.1554in;depth 0pt;original-width 10.6666in;original-height
8.0004in;cropleft "0";croptop "1";cropright "1";cropbottom "0";filename
'Rear_WheelHope_fase.bmp';file-properties "XNPEU";}}

\FRAME{fhFU}{5.5737in}{4.1866in}{0pt}{\Qcb{Modo de vibra\c{c}\~{a}o das
massas n\~{a}o suspensas traseiras oscilando fora de fase sobre as molas da
suspens\~{a}o traseira. (\textit{Wheel Hop})}}{\Qlb{Result:WHROFemode}}{%
rear_wheelhope_outfase1.bmp}{\special{language "Scientific Word";type
"GRAPHIC";maintain-aspect-ratio TRUE;display "USEDEF";valid_file "F";width
5.5737in;height 4.1866in;depth 0pt;original-width 10.6666in;original-height
8.0004in;cropleft "0";croptop "1";cropright "1";cropbottom "0";filename
'Rear_WheelHope_outfase1.bmp';file-properties "XNPEU";}}

\section{\protect\bigskip}

\bigskip

\bigskip

\section{An\'{a}lise Espectral}

A seguir s\~{a}o apresentados os gr\'{a}ficos das PSDs da acelera\c{c}\~{a}o
vertical da carroceria para as seis configur\c{c}\~{o}es ensaiadas
utilizando o sinais de pista de asfalto lisa e irregular. Os gr\'{a}ficos 
\ref{ResultsLisaPSDtodos} e \ref{ResultsLisaPSDtodos01} mostram de forma
geral o comportamento da resposta da carroceria \`{a}s excita\c{c}\~{o}es
provenientes das irregularidades de uma pista de asfalto liso, com o
autom\'{o}vel trafegando a 100 km/h.

Em primeira an\'{a}lise, observa-se um aumento da transmissibilidade das
vibra\c{c}\~{o}es para a carroceria com o aumento do di\^{a}metro da barra
estabilizadora dianteira. Esse comportamento j\'{a} era previsto uma vez que
a an\'{a}lise modal mostrou o aumeto das frequ\^{e}ncias de resson\^{a}ncia
dos modos de \textit{bounce} e \textit{wheel hop} da suspens\~{a}o
dianteira, ver tabela \ref{ResultsModos}.

Pode-se observar nos gr\'{a}ficos \ref{ResultsLisaPSDtodos} e \ref
{ResultsLisaPSDtodos01} que al\'{e}m do aumento da amplitude das acelera\c{c}%
\~{o}es h\'{a} um sens\'{i}vel deslocamento das frequ\^{e}ncias de
resson\^{a}ncia com o aumento do di\^{a}metro da barra estabilizadora. Entre 
$1,3Hz $ e $3Hz$ observa-se um consider\'{a}vel aumento das acelera\c{c}%
\~{o}es da carroceria com aumento do di\^{a}metro da barra estabilizadora,
essa regi\~{a}o compreende a faixa de frequ\^{e}ncia do modo de\textit{\
bounce}.

\FRAME{fhFU}{6.0079in}{4.3846in}{0pt}{\Qcb{PSDs da acelera\c{c}\~{a}o
vertical da carroceria variando o di\^{a}metro da barra estabilizadora da
suspens\~{a}o dianteira. Pista de asfalto liso - 100 km/h.}}{\Qlb{%
ResultsLisaPSDtodos}}{psdlisa01.bmp}{\special{language "Scientific
Word";type "GRAPHIC";maintain-aspect-ratio TRUE;display "USEDEF";valid_file
"F";width 6.0079in;height 4.3846in;depth 0pt;original-width
5.9482in;original-height 4.3336in;cropleft "0";croptop "1";cropright
"1";cropbottom "0";filename 'PSDlisa01.bmp';file-properties "XNPEU";}}

\bigskip \FRAME{fhFU}{6.0079in}{4.3846in}{0pt}{\Qcb{PSDs da acelera\c{c}%
\~{a}o vertical da carroceria variando o di\^{a}metro da barra
estabilizadora da suspens\~{a}o dianteira. Pista de asfalto liso - 100 km/h.}%
}{\Qlb{ResultsLisaPSDtodos01}}{psdlisa02.bmp}{\special{language "Scientific
Word";type "GRAPHIC";maintain-aspect-ratio TRUE;display "USEDEF";valid_file
"F";width 6.0079in;height 4.3846in;depth 0pt;original-width
5.9482in;original-height 4.3336in;cropleft "0";croptop "1";cropright
"1";cropbottom "0";filename 'PSDlisa02.bmp';file-properties "XNPEU";}}

\FRAME{fhFU}{6.0079in}{4.3846in}{0pt}{\Qcb{PSDs da acelera\c{c}\~{a}o
vertical da carroceria variando o di\^{a}metro da barra estabilizadora da
suspens\~{a}o dianteira. Pista de asfalto liso - 100 km/h.}}{\Qlb{%
ResultsLisaPSDtodos02}}{psdlisa03.bmp}{\special{language "Scientific
Word";type "GRAPHIC";maintain-aspect-ratio TRUE;display "USEDEF";valid_file
"F";width 6.0079in;height 4.3846in;depth 0pt;original-width
5.9482in;original-height 4.3336in;cropleft "0";croptop "1";cropright
"1";cropbottom "0";filename 'PSDlisa03.bmp';file-properties "NPEU";}}

\bigskip \FRAME{fhFU}{6.186in}{4.3846in}{0pt}{\Qcb{PSDs da acelera\c{c}%
\~{a}o vertical da carroceria variando o di\^{a}metro da barra
estabilizadora da suspens\~{a}o dianteira.}}{\Qlb{ResultsIrregPSDtodos}}{%
psdirreg01.bmp}{\special{language "Scientific Word";type
"GRAPHIC";maintain-aspect-ratio TRUE;display "USEDEF";valid_file "F";width
6.186in;height 4.3846in;depth 0pt;original-width 6.1246in;original-height
4.3336in;cropleft "0";croptop "1";cropright "1";cropbottom "0";filename
'PSDIrreg01.bmp';file-properties "XNPEU";}}

\section{\protect\bigskip An\'{a}lise de Sensibilidade}

\bigskip

\chapter{Conclus\~{o}es e Recomenda\c{c}\~{o}es}

\bigskip

\QTP{appendix}
Rotinas para gera\c{c}\~{a}o do sinal de pista.

%TCIMACRO{
%\TeXButton{indent}{\parindent=10mm%
%}}%
%BeginExpansion
\parindent=10mm%
%
%EndExpansion

O sinais utilizados para a simula\c{c}\~{a}o do modelo foram gerados
utilizando o software Visual Basic. A seguir s\~{a}o apresentadas as rotinas
geradas para os dois sinais de pista utlizados neste trabalho.

\section{Perfis de Pista}

\subsection{Perf\'{i}l para pista de asfalto liso}

Sub pista01()

'Mauro B Biasizzo - 22.08.2001 - S\~{a}o Jos\'{e} dos Campos

'Rotina para a gera\c{c}\~{a}o de sinal de pista.

'Gera um Time History da eleva\c{c}\~{a}o do perfil da pista com fases
aleat\'{o}rias.

'A vari\'{a}veis para a obten\c{c}\~{a}o da PSD da pista seguem a classifica%
\c{c}\~{a}o do MIRA.

'Rotina utilizada para a gera\c{c}\~{a}o do sinal de pista lisa com
ve\'{i}culo a 100 km/h.

Dim Wf() As Double ' Vetor das frequencias que o sinal da pista

Dim v1() As Double ' Vetor dos numeros de onda

Dim Gh1() As Double ' Vetor da PSD da elevacao das irregularidades da pista

Dim Amp1() As Double ' Vetor das amplitudes da PSD

Dim T1() As Double ' Vetor do tempo para a construcao do sinal

Dim Hdd() As Double ' Vetor do sinal da pista para a roda dianteira direita

Dim Hde() As Double ' Vetor do sinal da pista para a roda dianteira esquerda

Dim Htd() As Double ' Vetor do sinal da pista para a roda traseira direita

Dim Hte() As Double ' Vetor do sinal da pista para a roda traseira esquerda

Dim FaseE() As Double ' Vetor do sinal da pista t = 0

Dim FaseD() As Double ' Vetor do sinal da pista t = 0

Pi = 3.1416

L = 2.36 ' Distancia entre eixos

V = 100 ' Velocidade do automovel em km/h

V = V / 3.6 ' Velocidade do automovel em m/s

T = 40 ' Tempo em segundos

Hf = 126 ' Frequencia mais alta que compoe o sinal da pista em rad/s (20Hz)

n = Round((Hf * T) / (2 * Pi)) ' Numero de termos da serie

' Gerador do vetor das frequencias que compoe o sinal da pista

For i = 1 To n

ReDim Preserve Wf(i)

Wf(i) = 2 * Pi * i / T

ReDim Preserve v1(i) 'Convertendo o vetor Wf de rad/seg para numero de onda
v1(ciclos/m)

v1(i) = Wf(i) / (2 * Pi * V)

Next i

' Dados para a PSD da pista

G0 = 0.000014 ' Coeficiente de Rugosidade da pista para v0 (m\symbol{94}%
3/Ciclos)

w1 = 1.945 ' Fator de ajuste da PSD

w2 = 1.36 ' Fator de ajuste da PSD

v0 = 1 / (2 * Pi) ' Numero de onda que marca o ponto de descontinuidade da
PSD

' Geracao da PSD da elevacao da pista

For i = 1 To n

ReDim Preserve Gh1(i)

If v1(i) \TEXTsymbol{<}= v0 Then

Gh1(i) = G0 * ((v1(i) / v0) \symbol{94}(-w1))

Else

Gh1(i) = G0 * ((v1(i) / v0) \symbol{94}(-w2))

End If

Next i

d1 = ((1 / T) / V)

' Calculo da amplitude do sinal aleatorio

For i = 1 To n

ReDim Preserve Amp1(i)

Amp1(i) = Sqr(2 * Gh1(i) * d1)

Next i

dt = 1 / (4 * n / T) ' Passo de Tempo (segundos)

j = Round(T / dt) ' Determinacao do numero de elementos do vetor do tempo

' Construcao do vetor do tempo para a geracao do sinal

For i = 1 To j

ReDim Preserve T1(i)

T1(i) = i * dt

Next i

' Criando dois vetores de Zeros

For i = 1 To j

ReDim Preserve Hdd(i)

ReDim Preserve Hde(i)

Hdd(i) = 0

Hde(i) = 0

Next i

For r = 1 To j

For g = 1 To n

ReDim Preserve Hdd(r)

ReDim Preserve Hde(r)

ReDim Preserve FaseE(r)

ReDim Preserve FaseD(r)

FaseE(r) = Rnd(2 * Pi) ' Gerador de fases aleat\'{o}rias para o lado
esquerdo da pista

FaseD(r) = Rnd(2 * Pi) ' Gerador de fases aleat\'{o}rias para o lado direito
da pista

Hdd(r) = Hdd(r) + (Amp1(g) * 1000 * Sin(Wf(g) * T1(r) + FaseD(r)))

Hde(r) = Hde(r) + (Amp1(g) * 1000 * Sin(Wf(g) * T1(r) + FaseE(r)))

Next g

Next r

'Constru\c{c}\~{a}o do sinal para as rodas traseiras

Te = L / V 'Tempo entre a passagem do eixo dinateiro e traseiro pelo mesmo
ponto da pista

np = Round(Te / dt)

For i = np + 1 To j

ReDim Preserve Htd(i)

ReDim Preserve Hte(i)

Htd(i) = Hdd(i - np)

Hte(i) = Hde(i - np)

Next i

' Check do gerador do vetor de frequencias,numeros de onda, amplitude etc

For i = 1 To n

Range(''C'' + CStr(i + 1)).Value = Wf(i)

Range(''D'' + CStr(i + 1)).Value = v1(i)

Range(''E'' + CStr(i + 1)).Value = Gh1(i)

Range(''F'' + CStr(i + 1)).Value = Amp1(i)

Next i

' Check do gerador do vetor do tempo

For i = 1 To j

Range(''G'' + CStr(i + 1)).Value = T1(i)

Range(''H'' + CStr(i + 1)).Value = Hdd(i)

Range(''I'' + CStr(i + 1)).Value = Hde(i)

Range(''J'' + CStr(i + 1)).Value = Htd(i)

Range(''K'' + CStr(i + 1)).Value = Hte(i)

Next i

Range(''C1'').Value = ''Freq''

Range(''D1'').Value = ''\# Onda''

Range(''E1'').Value = ''PSD''

Range(''F1'').Value = ''Ampl''

Range(''G1'').Value = ''Tempo''

Range(''H1'').Value = ''ElevDD\_''

Range(''I1'').Value = ''ElevDE\_''

Range(''J1'').Value = ''ElevTD\_''

Range(''K1'').Value = ''ElevTE\_''

' Check dos valores entrados no programa

Range(''A1'').Value = n

Range(''A2'').Value = T

Range(''A3'').Value = Hf

Range(''A4'').Value = V

Range(''A5'').Value = d1

Range(''A6'').Value = dt

Range(''A7'').Value = j

Range(''A8'').Value = v0

Range(''A9'').Value = np

Range(''A10'').Value = G0

Range(''A11'').Value = w1

Range(''A12'').Value = w2

End Sub

\subsection{Perfil da pista para asfalto irregular}

Sub pista02()

'Mauro B Biasizzo - 22.08.2001 - S\~{a}o Jos\'{e} dos Campos

'Rotina para a gera\c{c}\~{a}o de sinal de pista.

'Gera um Time History da eleva\c{c}\~{a}o do perfil da pista com fases
aleat\'{o}rias.

'A vari\'{a}veis para a obten\c{c}\~{a}o da PSD da pista seguem a classifica%
\c{c}\~{a}o do MIRA.

'Rotina utilizada para a gera\c{c}\~{a}o do sinal de pista irregular com
ve\'{i}culo a 40 km/h.

Dim Wf() As Double ' Vetor das frequencias que o sinal da pista

Dim v1() As Double ' Vetor dos numeros de onda

Dim Gh1() As Double ' Vetor da PSD da elevacao das irregularidades da pista

Dim Amp1() As Double ' Vetor das amplitudes da PSD

Dim T1() As Double ' Vetor do tempo para a construcao do sinal

Dim Hdd() As Double ' Vetor do sinal da pista para a roda dianteira direita

Dim Hde() As Double ' Vetor do sinal da pista para a roda dianteira esquerda

Dim Htd() As Double ' Vetor do sinal da pista para a roda traseira direita

Dim Hte() As Double ' Vetor do sinal da pista para a roda traseira esquerda

Dim FaseE() As Double ' Vetor do sinal da pista t = 0

Dim FaseD() As Double ' Vetor do sinal da pista t = 0

Pi = 3.1416

L = 2.36 ' Distancia entre eixos

V = 40 ' Velocidade do automovel em km/h

V = V / 3.6 ' Velocidade do automovel em m/s

T = 40 ' Tempo em segundos

Hf = 126 ' Frequencia mais alta que compoe o sinal da pista em rad/s (20Hz)

n = Round((Hf * T) / (2 * Pi)) ' Numero de termos da serie

' Gerador do vetor das frequencias que compoe o sinal da pista

For i = 1 To n

ReDim Preserve Wf(i)

Wf(i) = 2 * Pi * i / T

ReDim Preserve v1(i) 'Convertendo o vetor Wf de rad/seg para numero de onda
v1(ciclos/m)

v1(i) = Wf(i) / (2 * Pi * V)

Next i

' Dados para a PSD da pista

G0 = 0.00009 ' Coeficiente de Rugosidade da pista para v0 (m\symbol{94}%
3/Ciclos)

w1 = 2.14 ' Fator de ajuste da PSD

w2 = 1.428 ' Fator de ajuste da PSD

v0 = 1 / (2 * Pi) ' Numero de onda que marca o ponto de descontinuidade da
PSD

' Geracao da PSD da elevacao da pista

For i = 1 To n

ReDim Preserve Gh1(i)

If v1(i) \TEXTsymbol{<}= v0 Then

Gh1(i) = G0 * ((v1(i) / v0) \symbol{94}(-w1))

Else

Gh1(i) = G0 * ((v1(i) / v0) \symbol{94}(-w2))

End If

Next i

d1 = ((1 / T) / V)

' Calculo da amplitude do sinal aleatorio

For i = 1 To n

ReDim Preserve Amp1(i)

Amp1(i) = Sqr(2 * Gh1(i) * d1)

Next i

dt = 1 / (4 * n / T) ' Passo de Tempo (segundos)

j = Round(T / dt) ' Determinacao do numero de elementos do vetor do tempo

' Construcao do vetor do tempo para a geracao do sinal

For i = 1 To j

ReDim Preserve T1(i)

T1(i) = i * dt

Next i

' Criando dois vetores de Zeros

For i = 1 To j

ReDim Preserve Hdd(i)

ReDim Preserve Hde(i)

Hdd(i) = 0

Hde(i) = 0

Next i

For r = 1 To j

For g = 1 To n

ReDim Preserve Hdd(r)

ReDim Preserve Hde(r)

ReDim Preserve FaseE(r)

ReDim Preserve FaseD(r)

FaseE(r) = Rnd(2 * Pi) ' Gerador de fases aleat\'{o}rias para o lado
esquerdo da pista

FaseD(r) = Rnd(2 * Pi) ' Gerador de fases aleat\'{o}rias para o lado direito
da pista

Hdd(r) = Hdd(r) + (Amp1(g) * 1000 * Sin(Wf(g) * T1(r) + FaseD(r)))

Hde(r) = Hde(r) + (Amp1(g) * 1000 * Sin(Wf(g) * T1(r) + FaseE(r)))

Next g

Next r

'Constru\c{c}\~{a}o do sinal para as rodas traseiras

Te = L / V 'Tempo entre a passagem do eixo dinateiro e traseiro pelo mesmo
ponto da pista

np = Round(Te / dt)

For i = np + 1 To j

ReDim Preserve Htd(i)

ReDim Preserve Hte(i)

Htd(i) = Hdd(i - np)

Hte(i) = Hde(i - np)

Next i

' Check do gerador do vetor de frequencias,numeros de onda, amplitude etc

For i = 1 To n

Range(''C'' + CStr(i + 1)).Value = Wf(i)

Range(''D'' + CStr(i + 1)).Value = v1(i)

Range(''E'' + CStr(i + 1)).Value = Gh1(i)

Range(''F'' + CStr(i + 1)).Value = Amp1(i)

Next i

' Check do gerador do vetor do tempo

For i = 1 To j

Range(''G'' + CStr(i + 1)).Value = T1(i)

Range(''H'' + CStr(i + 1)).Value = Hdd(i)

Range(''I'' + CStr(i + 1)).Value = Hde(i)

Range(''J'' + CStr(i + 1)).Value = Htd(i)

Range(''K'' + CStr(i + 1)).Value = Hte(i)

Next i

Range(''C1'').Value = ''Freq''

Range(''D1'').Value = ''\# Onda''

Range(''E1'').Value = ''PSD''

Range(''F1'').Value = ''Ampl''

Range(''G1'').Value = ''Tempo''

Range(''H1'').Value = ''ElevDD\_''

Range(''I1'').Value = ''ElevDE\_''

Range(''J1'').Value = ''ElevTD\_''

Range(''K1'').Value = ''ElevTE\_''

' Check dos valores entrados no programa

Range(''A1'').Value = n

Range(''A2'').Value = T

Range(''A3'').Value = Hf

Range(''A4'').Value = V

Range(''A5'').Value = d1

Range(''A6'').Value = dt

Range(''A7'').Value = j

Range(''A8'').Value = v0

Range(''A9'').Value = np

Range(''A10'').Value = G0

Range(''A11'').Value = w1

Range(''A12'').Value = w2

End Sub

\QTP{bibsection}
Bibliografia

\begin{thebibliography}{99}
\bibitem{Alex Don}  Alexander, Don. (1991). Performance Handling. 1${{}^{a}}$
ed. MBI Publishing Company. NY. USA

\bibitem{Bastow}  Bastow, D; Howard, G.P. (1993). Car Suspension and
Handling. 3${{}^{a}}$ ed. SAE.

\bibitem{Darling}  Darling, J.; Hickson, L. R. (1998). An Experimental Study
of a Prototype Active Anti-Roll Suspension System. Vehicle System Dynamics
Supplement 29. Pp.309-329

\bibitem{choi}  Design Sensitivity Analysis of Structural Systems, Haug,
E.J., CHOI, K.K., Komkov, V. , 1986, Academic Press Inc. Orlando, Florida,
U.S.A.

\bibitem{Gillespie}  Gillespie, T. D. (1992). Fundamentals of Vehicle
Dynamics. SAE

\bibitem{Neto}  Neto, A C; Prado, M. (1995). Prot\'{o}tipo Virtual: Um Novo
Conceito no Desenvolvimento de Projetos Mec\^{a}nicos. SAE Thecnical Papers
Series. 982938.

\bibitem{Pollone}  Pollone, G.(1970). Il Veicolo - Libreria Editrice
Universitaria Levrotto \& Bella - 3${{}^{a}}$ Ed. - Torino - Italia

\bibitem{Reimpell}  Reimpell, J. ; Stoll, H. (1996). The Automotive Chassis:
Engineering Principles. SAE.

\bibitem{Sayers00}  Sayers, M.W.; Han, D. (1996). A Generic Multibody
Vehicle Model for Simulating Handling and Braking. Vehicle System Dynamics
Supplement 25. Pp.599-613.

\bibitem{Wallentowitz}  Wallentowitz, H.(1998). Analysis of Dynamic Driving
Control Systems (DDC) on a full Vehicle in ADAMS. http\TEXTsymbol{\backslash}%
\TEXTsymbol{\backslash}www.ika.rwth-aachen.de.

\bibitem{Wunsche}  Wunsche, T. and Muhr, K ; Biecker, K; Schnaubelt L. :
Side Load Springs as a Solution to Minimize Adverse Side Loads Acting on the
McPherson Strut. SAE paper 940862.

\bibitem{Montiglio}  Montiglio, Mauro.(1996). Ride Comfort del Veicolo -
Veicolo, Modulo R3. Master in Car Engineering - 3%
%TCIMACRO{\UNICODE{0xaa} }%
%BeginExpansion
${{}^a}$%
%EndExpansion
Edizione. Centro Richerche FIAT. Modena, Italia.

\bibitem{Genta}  Genta, G.''Motor Vehicle Dynamics'' ,World Scientific
Publishing, London, 1997.

\bibitem{ElBeheiry and Karnopp}  E.M.ElBeheiry ; D.C.Karnopp (1996) Journal
of Sound and Vibration, 189(5), 547-564. Optimal Control of Vehicle Random
Vibration with Constrained Suspension Deflection.

\bibitem{X.P.Lu}  X.P.Lu ; L. Segel. Vehicular Energy Losses Associated with
the Transversal of an Uneven Road. IAVSD Extensive Summaries.

\bibitem{X.P.Lu01}  X.P.Lu.1985 American Society for Testing and Materials
Proceeding of Symposium on Measuring Road Roughness and Its Effect On User
Cost and Comfort, 143-161. Effects of Road Roughness on Vehicular Rolling
Resistance.

\bibitem{Sayers01}  M.W.Sayers 1985 American Society of Mechanical Engineers
Proceeding of Symposium on Simulation Control of Ground Vehicles and
Transportation Systems, 113-129. Characteristic Power Spectral Density
Functions for Vertical and Roll Components of Road Roughness.

\bibitem{Sayers02}  Sayers,M.W.; Karamihas, S.M. Interpretation of Road
Roughness Profile Data. Prepared for Federal Highway Administration. June
1996.

\bibitem{Gillespie and Sayers}  Gillespie,T.D.;Sayers, M.W. e Segel, L.
Calibration of Response-Type Road Roughness Measuring Systems.Transportation
Research Board. National Reserch Council. Washington, D.C. 1980.

\bibitem{Tamboli and Joshi}  Tamboli, J.A.; Joshi, S.G.Optimum Design of a
Passive Suspension System of a Vehicle Subjected to Actual Random Road
Excitations. Journal of Sound and Vibration, 1999, 292(2), 193-205.

\bibitem{Kozin}  Kozin, F.; Bogdanoff, J.L. On the Statistical Analysis of
the Motion of Some Simple Two-Dimensional Linear Vehicles Moving on a Random
Track. Int. J. Mech. Sci. Pergamon Press Ltd. 1960. Vol. 2, pp. 168-178.
Great Britain.

\bibitem{Dodds}  Dodds, C.J.; Robson, J.D. The Description of Road Surface
Roughness, Journal of Sound and Vibration. 1973, 31 (2), pp.175-183.

\bibitem{Schlesinger}  Schlesinger, A. Laboratory assessment of ride quality
of a vehicle using a limited number of actuators.IMechE paper C466/043/93.
1993, pp.195-198.

\bibitem{Alonso}  Alonso, J. Manuel.(1997). Tecnolog\'{i}as Avanzadas del
Autom\'{o}vil.2%
%TCIMACRO{\UNICODE{0xaa}}%
%BeginExpansion
${{}^a}$%
%EndExpansion
Ed. Editorial Paraninfo. Madrid, Espa\~{n}a.

\bibitem{Jolly}  Jolly,A.(1983).Study of ride comfort using a nonlinear
mathematical model of a vehicle suspension. Int.J. of Vehicle Design, Vol.4,
no.3, pp.233-244. Printed in U.K.

\bibitem{Thomson}  Thomson,W.T.\ Theory of Vibration with Applications, 5th
edition, Prentice Hall, New Jersey . EUA.1998.

\bibitem{Ewins}  Ewins,D.J. Modal Testing: Theory an Practice.Research
Studies Press LTD. Tauton, Somerset, England.1995.

\bibitem{Inman}  Inman,D.J. Engineering Vibration.Prentice Hall, New Jersey
. EUA.1996.

\bibitem{Pipes}  Pipes, L.A. Applied Mathematics for Engineers and
Physicists.2nd edition, McGraw-Hill Book Company, Inc. New York, USA. 1958

\bibitem{Griffin}  Griffin, M.J. Handbook of Human Vibration, 1st Edition,
Academic Press,San Diego, California, USA.1996.

\bibitem{Satchell}  Satchell, Terry L. The Design of Trailing Twist Axels.
International Congress and Exposition, Cobo Hall,\ Detroit, Michigan.SAE
paper n%
%TCIMACRO{\UNICODE{0xba}}%
%BeginExpansion
${{}^o}$%
%EndExpansion
.810420, February 23-27, 1981. USA.

\bibitem{Salmon}  Salmon, C.G., Jonhson, J.E., \textit{Steel Structures
Design and Behaivor}, Harper \& Row, New York, 1980.

\bibitem{ADAMS Manual}  Using ADAMS/Post Processor - Manipulating Curve Data.
\end{thebibliography}

\end{document}
